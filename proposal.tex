\documentclass[doublespace]{styles/rackham-thesis}

\usepackage[linktocpage=true]{hyperref}
\usepackage[utf8]{inputenc}
\usepackage{amsmath, amssymb, amsthm, xfrac}
\usepackage{booktabs}
\usepackage{lipsum}
\usepackage{flafter}

\documentclass{styles/thesis-umich}
\usepackage[margin=1in]{geometry}
\usepackage{mathptmx}\renewcommand{\ttdefault}{cmtt}
\usepackage{graphicx}

%\usepackage{microtype}
\usepackage{xspace}
\usepackage[compact]{titlesec}
\usepackage{color}
\definecolor{azure}{rgb}{0.16, 0.32, 0.75}
%\usepackage[%
%    breaklinks=true,colorlinks=true,urlcolor=azure,
%    citecolor=black,pdftex]{hyperref}
%\thispagestyle{empty}\pagestyle{plain}
\usepackage{url}\urlstyle{rm}

\usepackage{amsmath,amssymb,amsbsy}

\usepackage[margin=0pt]{caption}
\usepackage{algorithm, algorithmic}
\usepackage{graphicx}
\usepackage{tikz}
\usepackage{pifont}% http://ctan.org/pkg/pifont
\usepackage{color}
\usepackage{verbatim}
%\usepackage[margin=1in]{geometry}
%\usepackage{watermark}
\usepackage{booktabs}
\usepackage{listings}
\usepackage{cite}
\usepackage{wasysym}
\usepackage{paralist}
\usepackage{balance}
\usepackage{tabularx}
\usepackage[protrusion=true,expansion=true,kerning]{microtype}
\usepackage{multirow}
\usepackage{placeins}
\usepackage{pgfplots}
\usepackage{wrapfig}
\usepackage{colortbl}
\usepackage{adjustbox}
\usepackage{bigstrut}
%\usepackage{subcaption}
\newcommand{\specialcell}[2][c]{%
  \begin{tabular}[#1]{@{}c@{}}#2\end{tabular}}

\newcommand{\twolinecell}[2][r]{%
  \begin{tabular}[#1]{@{}c@{}}#2\end{tabular}}

\newcommand{\twollinecell}[2][r]{%
  \begin{tabular}[#1]{@{}l@{}}#2\end{tabular}}


\newcommand*{\refname}{Bibliography}

\newcolumntype{P}[1]{>{\centering\arraybackslash}p{#1}}

\usepackage{float}

\newcommand{\ind}{\hspace*{1em}}
\newif\ifex\extrue % for extended version
\newif\ifredact\redactfalse
\newcommand{\re}[1]{{#1}}
\newcommand{\subpar}[1]{\medskip\noindent\textsl{#1}\enspace}
\newcommand{\update}[1]{{\color{black}#1}\xspace}
\newcommand{\mycaption}[2]{\caption{\textbf{#1.} #2}}
\newcommand{\hostfont}[1]{\texttt{#1}}
\newcommand{\starttls}{STARTTLS\@\xspace}
\newcommand{\cmark}{\ding{51}}
\newcommand{\xmark}{\color{red}{\ding{53}}}

\usepackage{compactbib}
%\usepackage[compress]{cite}
%\renewcommand{\bibliorefname}[1]{\subsection*{#1}}
\renewcommand{\paragraph}[1]{\medskip\noindent\textbf{#1}\,\,\,}

\newcommand{\todo}[1]{{\color{red}{\textbf{\em [TODO: #1]}}}\xspace}
\newcommand{\TODO}[1]{\todo{#1}}

% Math Helpers
\newcommand{\bbZ}{\mathbb{Z}}

\newtheorem{theorem}{Theorem}
\newcommand{\FF}{\ensuremath{\mathbb{F}}}
\newcommand{\QQ}{\ensuremath{\mathbb{Q}}}


\begin{document}

\title{Using Large-Scale Empirical Methods to Understand Fragile Cryptographic Ecosystems}
\author{David Adrian}
\degree{Doctor of Philosophy}
\department{Computer Science and Engineering}
\committee{
  Professor J.~Alex Halderman, Chair\\
  Research Professor Peter Honeyman\\
  Associate Professor Chris Peikert\\
  Assistant Professor Florian Schaub\\
}

\maketitle

% !TEX root = proposal.tex

\dedication{
To my grandfather, who asked me to work hard and be successful.
}

\acknowledgments{
%\vspace{20pt}
%\setlength{\parindent}{2em}

There are many people to thank; these people will be thanked in the final
dissertation. Until then, I'd like to thank my advisor, J.~Alex Halderman,
for putting up with my unique path into and out of the program, both as a
Master's and a Ph.D.\. student. I've learned tremendous amount over the last
five years, and I will never forget my time as your student. I'd also like to
thank Peter Honeyman for helping push me towards the door, and the rest of my
commitee---Chris Peikert and Florian Schaub---for agreeing to be a part of
this.

The work in this dissertation was supported in part by the National Science
Foundation and the Alfred P. Sloan Foundation, with additional support from
the Mozilla Foundation, Supermicro, Google, Cisco, and the Morris Wellman
Professorship.

}


\tableofcontents
\listoffigures
\listoftables
\listofappendices

\abstract{
Abstract data types are good.    
}


\chapter{Introduction}
% !TEX root ../proposal.tex

Large-scale empirisicm, enabled by Internet-wide scanning, provides missing
insight into the security of cryptography used on the Internet. 

% What insight is "missing"? What are the other methods?

We gain additional insight into cryptography from large-scale empiricism.

The Internet is a large, distributed system with a diverse set of clients and
servers~\cite{something}. Compatibility with legacy clients remains a top priority 
for many operators~\cite{something} and protocol designers~\cite{something}.

% High-level questions that empiricism helps with. (TLS 1.3 compatibility woes?)
%
% List of insights that help with this.

We can use that insight to improve the overall security of the Internet, today.

We can use that insight to better inform the design of the Internet in the future.

The latest version of TLS 1.3~\cite{rfc8446} was drafted over the course of \TK years, with countermeasures
in place for both known cryptographic vulnerabilites in earlier version of TLS, as well as with updated negotation and parameter selection processes, designed to prevent deployment failures that had traditionally been considered outside of the scope of standards.

We can use it alongside analysis of individual components. (With our power combined!)

Using large-scale empiriscism requires solving engineering challenges.

\section{Large-scale Empiricism}

% Some people study cryptograpy usages in individual programs~\cite{most-dangerous-code-2012}.
% 
% API design helps.
% 
% But ultimately, what is happening? Are we impacting the security on the web as a whole?
% 
% But what about ecosystems?
% 
% Study usage across ecosystems.
% 
% Usage is configuration.
% 
% Design experiments to glean information into configuration.
% 
% Talk about configuration of the ecosystem beyond just configuration of hosts (cross-protocol correlation).

Security research often involves searching for new classes of
vulnerabilites~\cite{something}, or identifying vulnerabilites within
existing systems~\cite{something}. Other research has concerned identifying
misuse of security-critical APIs~\cite{gutmann-lessons}, including systematic
misuse among large swaths of library code
users~\cite{most-dangerous-code-2012}. While improving API design and
secure-by-default programming models certainly improve the security behavior
surrounding network communication for many programs, it does not provide
insight into the security behavior of Internet ecosystems themselves.

Large-scale empiricism enables us to understand the security behavior
ecosystems, rather than \TK. To understand the behavior of users of cryptographic applications, 

\subsection{Cryptography in the TLS Ecosystem}

Configuration of web servers.

Logjam.

Diffie-Hellman. Subgroup.

RFC 5114

\subsection{Unexpected Interactions in Cryptography at Scale}

Beyond examining individual protocols at the global perspective, new issues can
be identified by looking across multiple similar protocols. Although it is
tempting to consider the security of any given system in isolation, complex
interactions between systems necessarily impacts security.

At the most-basic level, cross-protocol key-reuse links the security of any two
protocols together in key-compromise scenarios. When the same server private
key is used for both a mail server and a web server, compromising the often
less-secured mail server effectively compromises the web
server~\cite{mail-2015}.

Looking past simply key reuse, name reuse across protocols utilizing TLS opens
additional attack vectors. Any service with a certificate chaining to a
publicly trusted root that shares a Common Name or Subject Alternative Name
with a web server, or that covers the web address under a wildcard name, can be
used to impersonate the web server if the key is compromised.

Furthermore, specific vulnerabilites in the TLS protocol and implementations
can be utilized in a cross-protocol context to attack users of a web service
using an unpatched ``forgotten-about'' server, such as a mail server, without
even explicitly compromising the private key of the service. The best example
of this is the DROWN vulnerability~\cite{drown-2016}, in which the mere
existence of an SSLv2 host that shared a key with a TLS host enabled decryption
of otherwise secure TLS connections using modern cryptography.

\paragraph{Cross-protocol interactions between TLS and SSLv2}

SSLv2

DROWN

Generalize DROWN to cross-protocol key reusage

\paragraph{Export Cryptography}
Don't weaken cryptography.

Takeaways from Black Hat talk.

\section{Engineering Challenges}

To do science, we have to collect data.

To collect the data, we have to build things.

Determining what to build and how to build is also difficult.

Engineering is very closely related to methodology, which is where the science is.

\subsection{Speed}

Assume we're using scanning.

TLS-Attacker is slow.

Performance matters, especially once you start cross-correlating.

Discovery and collection separate. (ZMap vs ZGrab)

\subsection{Data Processing}

Beyond collecting data, you actually have to process it.

Straightforward for one-off studies.

What if you want longitudinal data?

Take-away This all requires work.


\chapter{Improving Measurement}
\label{chapter:zippier}
% !TEX root = ../../../proposal.tex

\newcommand*{\ZippyPaper}{papers/zippy/paper}
\newcommand*{\ZippyFigures}{papers/zippy/figures}
%%ethertex-once:https://davidadrian.org/.well-known/usenix.sty
\documentclass[letterpaper,twocolumn,10pt]{article}
\usepackage{usenix,epsfig,footnote,url}\urlstyle{rm}
\usepackage[T1]{fontenc}
\usepackage[%
    breaklinks=true,colorlinks=true,linkcolor=black,%
     citecolor=black,urlcolor=black,bookmarks=true,bookmarksopen=false,%
    pdfauthor={David Adrian, Zakir Durumeric, Gulshan Singh, and J. Alex Halderman},
    pdftitle={Zippier ZMap: Internet-Wide Scanning at 10 Gbps}
    ,pdftex]{hyperref}
\usepackage{amsmath, amssymb, amsfonts, amsthm}
\usepackage{mathptmx}\renewcommand{\ttdefault}{cmtt}
\usepackage[margin=0pt,font=small]{caption}
\usepackage{graphicx}
\usepackage{booktabs}
\usepackage{balance}
\usepackage{cite}
\usepackage[protrusion=true,expansion=true,kerning]{microtype}
\SetExtraKerning % Add space around emdashes
   { encoding = *,
        font = * }
   { \textemdash  = {120,120} }

% Fix ugly USENIX subsection headings (AH 3/08)
\makeatletter
\renewcommand{\section}{\@startsection {section}{1}{\z@}%
                                   {-3.5ex plus-1ex minus -.2ex}%
                                   {2.3ex plus.2ex}%
                                   {\normalfont\large\bfseries}}
\renewcommand{\subsection}{\@startsection{subsection}{2}{\z@}%
                                     {-2.5ex plus-.7ex minus -.2ex}%
                                     {1.5ex plus .2ex}%
                                     {\normalfont\fontsize{11}{12.5}\bfseries}}
\makeatother

% Paragraph and subpar
\renewcommand{\paragraph}[1]{\medskip\noindent\textbf{#1}\quad}
\newcommand{\subpar}[1]{\medskip\noindent\textsl{#1}\enspace}

% Stop URLs from hyphenating after  "http:" (AH 12/08)
\def\UrlBreaks{\do-\do\.\do\@\do\\\do\!\do\_\do\|\do\;\do\>\do\]%
 \do\)\do\,\do\?\do\'\do+\do\=\do\#}
\def\UrlBigBreaks{\do\:\do\/}%

% TODO, TK, etc. (AH 4/12)
\usepackage{xspace}
\newcommand{\todo}[1]{{\color{red}{\textbf{\em [TODO: #1]}}}\xspace}
\newcommand{\TODO}[1]{\todo{#1}}
\newcommand{\tk}{{\color{red}{\bf TK}}\xspace}
\newcommand{\TK}{\tk}
\newcommand{\comment}[1]{\relax} % comment out text
\newcommand{\xcite}[1]{\relax} % comment out citation

\newif\ifweb\webtrue
\ifweb
    \usepackage{watermark}
    \thiswatermark{\parbox{\textwidth}{\vskip30pt\centering
    \vspace*{-20pt}%
     This paper appeared in \emph{Proceedings of the 8th {USENIX} Workshop on Offensive Technologies} (WOOT '14), August~2014.\\
    \vskip6pt
    \rule[\baselineskip]{\textwidth}{0.5pt}
    }}
\fi

\begin{document}
\pagenumbering{arabic}
\thispagestyle{empty}

% Some math stuff
\newcommand{\bbZ}{\mathbb{Z}}

%don't want date printed
\date{}

\title{\Large\bf Zippier ZMap: Internet-Wide Scanning at 10\,Gbps}

\author{
{\rm David Adrian, Zakir Durumeric, Gulshan Singh, and J.\,Alex Halderman}\smallskip\\
University of Michigan\\
\small\rm\{davadria,\,zakir,\,gulshan,\,jhalderm\}@umich.edu
}
 

\maketitle


\begin{abstract}
We introduce optimizations to the ZMap network scanner that achieve a 10-fold
increase in maximum scan rate. By parallelizing address generation,
introducing an improved blacklisting algorithm, and using zero-copy NIC
access, we drive ZMap to nearly the maximum throughput of
10~gigabit~Ethernet, almost 15 million probes per second. With these changes,
ZMap can comprehensively scan for a single TCP port across the entire public
IPv4 address space in 4.5~minutes given adequate upstream bandwidth. We
consider the implications of such rapid scanning for both defenders and
attackers, and we briefly discuss a range of potential
applications.\looseness=1
\end{abstract}

\section{Introduction}
\label{sec:introduction}

In August 2013, we released ZMap, an open-source network scanner designed to
quickly perform Internet-wide network surveys~\cite{zmap-2013}. From a single
machine, ZMap is capable of scanning at 1.44~million packets per second
(Mpps), the theoretical limit of gigabit Ethernet. At this speed, ZMap can
complete a scan targeting one TCP port across the entire public IPv4 address
space in under 45~minutes---a dramatic improvement compared to
weeks~\cite{zmap-2013} or months~\cite{ssl-observatory-2010} required using Nmap. Yet
even at gigabit linespeed, ZMap does not utilize the full bandwidth of the
fastest readily available connections: 10\,GigE uplinks are now offered by
Amazon~EC2~\cite{amazon-10g} and at a growing number of research
institutions. \looseness=-1

In this paper, we scale ZMap to 10\,GigE speeds by introducing a series of
performance enhancements. These optimizations allow scanning speeds that
provide higher temporal resolution when conducting Internet-wide surveys and
make it possible to quickly complete complex multipacket studies.

Scanning at 10\,GigE linespeed necessitates sending nearly 15~Mpps
continuously. For single-packet probes such as SYN scans, this allows only
200 cycles per probe on a 3\,GHz core. An L2 cache miss might incur a cost of
almost 100 cycles, so it essential to make efficient use of both CPU and
memory. In order to generate and transmit packets at this rate, we introduce
modifications that target the three most expensive per-probe operations in
ZMap:
\begin{enumerate}
  \item \emph{Parallelized address generation.}\quad
    ZMap uses a multiplicative cyclic group to iterate over a random permutation
    of the address space, but this becomes a bottleneck at multigigabit speeds.
    We implement a mutex-free sharding mechanism that spreads address generation
    across multiple threads and cores.
  \item \emph{Optimized address constraints.}\quad
    Responsible scanning requires honoring requests from networks that opt out,
    but over time this can result in large and complex blacklists. We develop an
    optimized address constraint data structure that allows ZMap to efficiently
    cycle through allowed targets.
  \item \emph{Zero-copy packet transmission.}\quad
    ZMap sends Ethernet frames using a raw socket, which avoids the kernel's
    TCP/IP stack but still incurs a per-packet context switch. We switch to using
    the PF\_RING Zero Copy (ZC) interface, which bypasses the kernel and reduces
    memory bandwidth.
\end{enumerate}

These enhancements enable ZMap to scan at 14.23~Mpps, 96\% of the theoretical
limit of 10\,GigE\@. In order to confirm these performance gains, we
completed a full scan of the IPv4 address space in 4m29s---to our knowledge,
the fastest Internet-wide scan yet reported.

The ability to scan at 10\,GigE speeds creates new opportunities for security
researchers. It allows for truer snapshots of the state of the Internet by
reducing error due to hosts that move or change during the scan. Likewise, it
enables more accurate measurement of time-critical phenomena, such as
vulnerability patching in the minutes and hours after public disclosure. On
the other hand, it raises the possibility that attackers could use 10\,GigE
to exploit vulnerabilities with alarming speed.

\section{Related Work}
\label{sec:relatedwork}

Many network scanning tools have been
introduced~\cite{scanrand,unicornscan,masscan-10g,nmap,zmap-2013}, although until
recently most were designed for scanning small networks. One of the most
popular is Nmap~\cite{nmap}, a highly capable network exploration tool. Nmap
is well suited for vertical scans of small networks or individual hosts, but
the original ZMap implementation outperformed it on horizontal Internet-wide
scans by a factor of 1300~\cite{zmap-2013}. Our enhancements to ZMap improve its
performance by another factor of ten.

ZMap is not the first Internet-wide scanner to use PF\_RING to send at speeds
greater than 1~Gbps. Masscan, released in September 2013, also utilizes
PF\_RING and claims the ability to scan at 25~Mpps using dual 10\,GigE
ports---84\% of the theoretical limit of dual 10\,GigE~\cite{masscan-10g}. We
present a more detailed comparison to Masscan in
Section~\ref{sec:masscan-comparison}. While the Masscan team did not have the
facilities to perform live network tests at rates higher than
100,000~pps~\cite{masscan-10g}, we report what we believe is the first
Internet-wide scan conducted at 10\,GigE speeds.

\section{Performance Optimizations}
\label{sec:optimizations}

ZMap achieves this performance based on a series of architectural choices
that are geared towards very large, high-speed scans~\cite{zmap-2013}. It avoids
per-connection state by embedding tracking information in packet fields that
will be echoed by the remote host, using an approach similar to
SYN~cookies~\cite{bernstein1996syn}. It eschews timeouts and simplifies flow
control by scanning according to a random permutation of the address space.
Finally, it avoids the OS's TCP/IP stack and writes raw Ethernet frames.

This architecture allows ZMap to exceed gigabit Ethernet linespeed on
commodity hardware, but there are several bottlenecks that prevent it from
fully reaching 10\,GigE speeds. ZMap's address generation is CPU intensive
and requires a global lock, adding significant overhead. Blacklisting ranges
of addresses is expensive and scales poorly. Sending each packet requires a
context switch and unnecessary copies as packets are passed from userspace to
the kernel and then to the NIC~\cite{multi-core-network-2010}. We implement
optimizations that reduce each of these bottlenecks.

\subsection{Address Generation Sharding}

Address generation in ZMap is designed to achieve two goals. First, it avoids
flooding destination networks by ordering targets according to a pseudorandom
permutation of the address space. Second, it enables statistically valid
sampling of the address space.
% by ensuring that prefixes of the permutation have good statistical
% randomness. However, on a per-probe basis, address generation is one of the
% most costly operations.

ZMap iterates over a multiplicative group of integers modulo $p$ that
represent 32-bit IPv4 addresses. By choosing $p$ to be $2^{32} + 15$, the
smallest prime larger than $2^{32}$, we guarantee that the group
$(\bbZ/p\bbZ)^{\times}$ is cyclic and that it covers the full IPv4 address
space. ZMap derives a new random primitive root $g$ for each scan in order to
generate new permutation of the address space. The scanner starts at a random
initial address $a_{0}$ and calculates $a_{i+1} = g \cdot a_{i} \bmod p$ to
iterate through the permutation. The iteration is complete when $a_{i+1}$
equals $a_{0}$.

The most expensive part of this scheme is the modulo operation, which must be
performed at every step of the iteration. Unfortunately, the modulo operation
cannot currently be performed by multiple threads at once, because each
address in the permutation is dependent on the previous---calculating the
next address requires acquiring a lock over the entire iterator state.

To remove this bottleneck and efficiently distribute address generation over
multiple cores, we extend ZMap to support sharding. In the context of ZMap, a
shard is a partition of the IPv4 address space that can be iterated over
independently from other shards; assigning one shard to each thread allows
for independent, mutex-free execution. Each shard contains a disjoint subset
of the group, with the union of all the shards covering the entire group.

To define $n$ shards, we choose an initial random address $a_0$ and assign
each sequential address $a_j$ in the permutation to shard $j \bmod n$. To
implement this, we initialize shards $1 \dots n$ with starting addresses
$a_0,\dots,a_{n-1}$, which can be efficiently calculated as $a_0 \cdot
g^{0,\dots,n-1}$. To iterate, we replace $g$ with $g^n$, which ``skips
forward'' in the permutation by $n$ elements at each step. Each shard
computes $a_{i+1} = a_i \cdot g^n \bmod p$ until reaching its shard specific
ending address $a_{e_j}$. For example, if there were three shards, the first
would scan
$\{ {a_0,\; a_3=g^3 \cdot a_0,\; a_6 = g^3 \cdot a_3,\dots,\; a_{e_1}} \}$,
second
$\{ {a_1,\; a_4=g^3 \cdot a_4,\; a_7 = g^3 \cdot a_4,\dots,\; a_{e_2}} \}$, 
and third 
$\{ {a_2,\; a_5=g^3 \cdot a_0,\; a_8 = g^3 \cdot a_5,\dots,\; a_{e_3}} \}$. 
We illustrate the process in Figure~\ref{fig:sharding}.

After pre-calculating the shard parameters, we only need to store three
integers per shard: the starting address $a_0$, the ending address $a_e$, and
the current address $a_i$. The iteration factor $g^n$ and modulus $p$ are the
same for all shards. Each thread can then iterate over a single shard
independently of the other threads, and no global lock is needed to determine
the next address to scan. Multiple shards can operate within the same ZMap
process as threads (the configuration we evaluate in this paper), or they can
be split across multiple machines in a distributed scanning mode.

\begin{figure}\centering
\includegraphics[width=\linewidth]{\ZippyFigures/shard_diagram.pdf}
\caption{\textbf{Sharding Visualization}\,---\,%
This is a configuration with three shards ($n = 3$). Shards $0,1,2$ are
initialized with starting addresses $a_0,a_1,a_2$. Each arrow represents
performing $a_i \cdot g^3$, a step forward by three elements in the
permutation.
}
\label{fig:sharding}
\end{figure}

\paragraph{Benchmarks}
To measure the impact of sharding in isolation from our other enhancements,
we conducted a series of scans, each covering a 1\% sample of the IP address
space, using our local blacklist file and a 10\,GigE uplink. Without
sharding, the average bandwidth utilization over 10~scans was 1.07 Gbps; with
sharding, the average increased to 1.80 Gbps, an improvement of 68\%.

\subsection{Blacklisting and Whitelisting}
 
ZMap address constraints are used to limit scans to specific areas of the
network (whitelisting) or to exclude particular address ranges
(blacklisting), such as IANA reserved allocations~\cite{iana}. Blacklisting
can also be used to comply with requests from network operators who want to
be excluded from receiving probe traffic. Good Internet citizenship demands
that ZMap users honor such requests, but after many scans over a prolonged
time period, a user's blacklist might contain hundreds of excluded prefixes.
 
Even with complicated address constraints, ZMap must be able to efficiently
determine whether any given IP address should be part of the scan. To support
10\,GigE linespeed, we implemented a combination tree- and array-based data
structure that can efficiently manipulate and query allowed addresses.

The IPv4 address space is modeled as a binary tree, where each node
corresponds to a network prefix. For example, the root represents 0.0.0.0/0,
and its children, if present, represent 0.0.0.0/1 and 128.0.0.0/1. Each
\emph{leaf} is colored either white or black, depending on whether or not the
corresponding prefix is allowed to be scanned. ZMap constructs the tree by
sequentially processing whitelist and blacklist entries that specify CIDR
prefixes. For each prefix, ZMap sets the color of the corresponding leaf,
adding new nodes or pruning the tree as necessary.

Querying whether an address may be scanned involves walking the tree,
beginning with the most significant bit of the address, until arriving at a
leaf and returning the color. However, a slightly different operation is used
during scanning. To make efficient use of the pseudorandom permutation
described above, we determine the number of allowed addresses $n$ (which may
be much smaller than the address space if a small whitelist is specified) and
select a permutation of approximately the same size. We then map from this
permutation of $1,\ldots,n$ to allowed addresses $a_1,\ldots,a_n$. Each node
in the tree maintains the total number of allowed addresses covered by its
descendants, allowing us to efficiently find the $i$th allowed address using
a simple recursive procedure.
%The ability to find specific addresses allows ZMap to utilize smaller cyclic
%groups to iterate over a small number of disjoint regions of address instead
%of iterating over the entire address space.\looseness=-1
 
As a further optimization, after the tree is constructed, we assemble a list
of /20 prefixes that are entirely allowed and reassign the address indices so
that these prefixes are ordered before any other allowed addresses. We then
use an array of these prefixes to optimize address lookups. If there are $m$
/20 prefixes that are allowed, then the first $m\cdot2^{12}$ allowed
addresses can be returned using only an array lookup, without needing to
consult the tree. The /20 size was determined empirically as a trade off
between lookup speed and memory usage.
 
\subsection{Zero-Copy NIC Access}
\label{sec:zc}
 
Despite ZMap's use of raw Ethernet sockets, sending each probe packet is an
expensive operation, as it involves a context switch for the \texttt{sendto}
system call and requires the scan packet to be transferred through kernel
space to the NIC~\cite{pfring-original, ten-gig-commodity}. Even with our
other enhancements, the high cost of these in-kernel operations prevented
ZMap from reaching above 2\,Gbps. To reduce these costs, we reimplemented
ZMap's network functionality using the PF\_RING ZC
interface~\cite{pfring-zc}. PF\_RING ZC allows userspace code to bypass the
kernel and have direct ``zero-copy'' access to the NIC\@, making it possible
to send packets without any context switches or wasted memory bandwidth.
 
To boost ZMap to 10\,GigE speeds, we implemented a new probe transmission
architecture on top of PF\_RING\@. This new architecture uses multiple
\emph{packet creation} threads that feed into a single \emph{send} thread. We
found that using more than one send thread for PF\_RING decreased the
performance of ZMap, but that a single packet creation thread was not fast
enough to reach line speed. By decoupling packet creation from sending, we
are able to combine the parallelization benefits of sharding with the speed
of PF\_RING\@.

In the original version of ZMap, multiple send threads each generated and
sent packets via a thread-specific raw Ethernet socket. We modify thread
responsibilities such that each packet creation thread iterates over one
address generation shard and generates and queues the packets. In a tight
loop, each packet generation loop calculates the next index in the shard,
finds the corresponding allowed IP address using the address constraint tree,
and creates an addressed packet in the PF\_RING ZC driver's memory. The
packet is added to a per-thread single-producer, single-consumer packet
queue. The send thread reads from each packet queue as packets come
available, and sends them over the wire using PF\_RING\@.

To determine the optimal number of packet creation threads, we performed a
series of tests, scanning for 50 seconds using 1--6 packet creation threads,
and measured the send rate. We find the optimal number of threads corresponds
with assigning one per physical core.
% Our results are show in Figure~\ref{fig:threads}.

\if0
\begin{figure}\centering
\includegraphics[width=\linewidth]{threads_vs_send.pdf}
\caption{\textbf{Threads vs.\ Scan Rate}\,---\,%
The send rate scales with the number of threads until reaching five threads,
at which point one thread is assigned per physical CPU. As threads begin to
share the same CPU, the send rate peaks. Using six threads results in an
identical send rate to five threads.
}
\label{fig:threads}
\end{figure}
\fi

\if0
\paragraph{Benchmarks}
We compared zero-copy NIC access to ZMap's traditional packet sending
architecture. In each case we also enabled the enhancements discussed in
Section~\ref{sec:TK} and Section~\ref{sec:TK} and conducted repeated scans of
covering 2\% samples of the public IPv4 address space. The average
transmission rate improved from 2.6~Mpps to 14.23~Mpps, a 447\% improvement.
This is 96\% of the theoretical limit of 10\,GigE,
14.88~Mpps~\cite{ten-gig-commodity}.
\fi

\begin{table}\centering
    \begin{tabular}{lrr}
    \toprule
    \textbf{Scan Rate} & \textbf{Hit Rate} & \textbf{Duration} \\ \midrule
    1.44 Mpps ($\approx$1\,GigE)                 & 1.00  & 42:08               \\ 
    3.00 Mpps                  & 0.99  & 20:47              \\ 
    4.00 Mpps                  & 0.97  & 15:38              \\
    14.23 Mpps ($\approx$10\,GigE)                 &  0.63 &  4:29               \\ % raw is 0.675
    \bottomrule
    \end{tabular}
    % 0.675
\caption{\textbf{Performance of Internet-wide Scans}\,---\,%
We show the scan rate, the normalized hit rate, and the scan duration (m:s)
for complete Internet-wide scans performed with optimized ZMap.}
\label{tbl:fullhitrate}
\end{table}

\section{Evaluation}
\label{sec:evaluation}

We performed a series of experiments to characterize the behavior of scanning
at speeds greater than 1~Gbps. In our test setup, we completed a full scan of
the public IPv4 address space in 4m29s on a server with a 10\,GigE uplink.
However, at full speed the number of scan results (the hit rate) decreased by
37\% compared to a scan at 1~Gbps, due to random packet drop. We find that we
can scan at speeds of up to 2.7~Gbps before seeing a substantial drop in hit
rate.

We performed the following measurements on a Dell PowerEdge R720 with two
Intel Xeon E5-2690 2.9~GHz processors (8 physical cores each plus
hyper-threading) and 128~GB of memory running Ubuntu 12.04.4~LTS and the
3.2.0-59-generic Linux kernel. We use a single port on a Intel X540-AT2
(rev~01) 10\,GigE controller as our scan interface, using the PF\_RING-aware
\texttt{ixgbe} driver bundled with PF\_RING 6.0.1. We configured ZMap to use
one send thread, one receive thread, one monitor thread, and five packet
creation threads.

We used a 10\,GigE network connection at the University of Michigan Computer
Science and Engineering division connected directly to the building uplink,
an aggregated $2\times 10$\,GigE channel. Beyond the 10\,GigE connection, the
only special network configuration arranged was static IP addresses. We note
that ZMap's performance may be different on other networks depending on local
congestion and upstream network conditions.\looseness=1

We performed all of our experiments using our local blacklist file. Our
blacklist, which eliminates non-routable address space and networks that have
requested exclusion from scanning~\cite{state-of-scanning-2014}, consists of over
1,000 entries of various-sized network blocks. It results in 3.7~billion
allowed addresses---with almost all the excluded space consisting of IANA
reserved allocations.

\subsection{Hit-rate vs.\ Scan-rate}

In our original ZMap study, we experimented with various scanning speeds up
to gigabit Ethernet line speed (1.44~Mpps) and found no significant effect on
the number of results ZMap found~\cite{zmap-2013}. In other words, from our
network, ZMap did not appear to miss any results when it ran faster up to
gigabit speed.

In order to determine whether hit-rate decreases with speeds higher than
1~Gigabit, we performed 50~second scans at speeds ranging from 0.1--14~Mpps.
We performed 3~trials at each scan rate. As can be seen in
Figure~\ref{fig:hitrate}, hit-rate begins to drop linearly after 4~Mpps. At
14~Mpps (close to 10\,GigE linespeed), the hit rate is 68\% of the hit rate
for a 1\,GigE scan. However, it is not immediately clear why this packet drop
is occurring at these higher speeds---are probe packets dropped by the
network, responses dropped by the network, or packets dropped on the scan
host due to ZMap?

\begin{figure}[t]\centering
\includegraphics[height=2.1in]{\ZippyFigures/norm_avg_hitrate.pdf}
\caption{\textbf{Hit-rate vs.\ Scan-rate}\,---\,%
ZMap's hit rate is roughly stable up to a scan rate of 4~Mpps, then declines
linearly. This drop off may be due to upstreudegrm network congestion. Even using
PF\_RING, Masscan is unable to achieve scan rates above 6.4~Mpps on the same
hardware and has a much lower hit rate.}
\label{fig:hitrate}
\end{figure}

\begin{figure}[t]\centering
\includegraphics[height=2.1in]{\ZippyFigures/avg_count_recv_succs.pdf}
\caption{\textbf{Response Rate During Scans}\,---\,%
This graph shows the rate of incoming SYN-ACKs during 50-second scans. The
peaks at the end (after sending finishes) at rates above 7~Mpps indicate that
many responses are being dropped and retransmitted before being recorded by
ZMap.}
\label{fig:recvrate}
\end{figure}

We first investigate whether response packets are being dropped by ZMap or
the network. In the original ZMap work, we found that 99\% of hosts respond
within 1~second~\cite{zmap-2013}. As such, we would expect that after 1~second,
there would be negligible responses. However, as can be seen in
Figure~\ref{fig:recvrate}, there is an unexpected spike in response packets
after sending completes at 50~seconds for scans at 10 and 14~Mpps. This spike
likely indicates that response packets are being dropped by our network, NIC,
or ZMap, as destination hosts will resend SYN-ACK packets for more than one
minute if an ACK or RST packet is not received.

In order to determine whether the drop of response packets is due to ZMap
inefficiencies or upstream network congestion, we performed a secondary scan
in which we split the probe generation and address processing onto separate
machines. The send machine remained the same. The receive machine was an HP
ProLiant DL120 G7, with an Intel Xeon E3-1230 processor (4 cores with
hyperthreading) and 16~GB of memory, running Ubuntu 12.04.4~LTS and the
3.5.0-52-generic Linux kernel.

As we show in Figure~\ref{fig:twomachines}, this spike does not occur when
processing response packets on a secondary server---instead it closely
follows the pattern of the slower scans. This indicates that ZMap is locally
dropping response packets. However, the split setup received only 4.3\% more
packets than the single machine---not enough to account for the 31.7\%
difference between a 14~Mpps and a 1~Mpps scan. If a large number of response
packets were dropped due to network congestion, we would not have observed an
immediate drop in responses---likely indicating that the root cause of the
decreased hit-rate is dropped probe packets.

It is not immediately clear where probe packets are dropped---it is possible
that packets are dropped locally by PF\_RING, are dropped by local routers
due to congestion, or that we are overwhelming destination networks. PF\_RING
records locally dropped packets, which remained zero throughout our scans,
which indicates that packets are not being dropped locally. In order to
locate where packet drop is occurring on our network, we calculated the drop
rate per AS and found little AS-level correlation for packets dropped by the
10\,GigE scans, which suggests that random packet drop is occurring close to
our network rather than at particular distant destination networks.

\begin{figure}\centering
\includegraphics[width=\linewidth]{\ZippyFigures/split_avg_count_recv_succs.pdf}
\caption{\textbf{Comparing One and Two Machines}\,---\,%
If we scan at 14~Mpps and use separate machines for the sending and receiving
tasks, the spike in the SYN-ACK rate at 50~s disappears, indicating that
fewer packets are dropped with the workload spread over two machines.
However, overall the two machine configuration received only 4.3\% more
responses than with one machine, which suggests that network packet loss
accounts for the majority of the drop off at higher scan rates.}
\label{fig:twomachines}
\end{figure}

\subsection{Complete Scans}

We completed a full Internet-wide scan, allowing ZMap to operate at its full
scan rate. This scan achieved an average 14.23~Mpps---96\% of the theoretical
limit of 10\,GigE, completing in 4~minutes, 29~seconds and achieving a hit
rate that is 62.5\% of that from a 1\,GigE scan. We show a comparison to
lower speed scans in Table~\ref{tbl:fullhitrate}. As we discussed in the
previous section, this decrease is likely due to local network congestion,
which results in dropped probe packets. However, more investigation is
deserved in order to understand the full dynamics of high-speed scans.

\subsection{Comparison to Masscan}
\label{sec:masscan-comparison}

\begin{figure*}\centering
\hfill
\includegraphics[height=2.5in]{\ZippyFigures/masscan_randomness.pdf}
\hfill\hfill\includegraphics[height=2.5in]{\ZippyFigures/zmap_randomness.pdf}
\hfill\strut
\caption{\textbf{Address Randomization Comparison}\,---\,%
These plots depict the first 1000 addresses of an Internet-wide scan selected
by Masscan (\emph{left}) and ZMap (\emph{right}), with the first and second
octets mapped to the $x$ and~$y$ coordinates. ZMap's address randomization is
CPU intensive but achieves better statistical properties than the cheaper
approach used by Masscan, enabling valid sampling. We enhanced ZMap to
distribute address generation across multiple cores.}
\label{fig:randomy}
\end{figure*}

Masscan advertises the ability to emit probes at 25~Mpps using PF\_RING and
two 10\,GigE adapters, each configured with two RSS queues---84\% of
linespeed for dual 10\,GigE and 166\% of linespeed for a single 10\,GigE
adapter~\cite{masscan-10g}. We benchmarked ZMap and Masscan using the
Xeon~E3-1230 machine described above. In our experiments, we found that
Masscan was able to send at a peak 7.4~Mpps using a single-adapter
configuration with two RSS queues, 50\% of 10\,GigE linespeed. On the same
hardware, ZMap is capable of reaching a peak 14.1~Mpps. While Masscan may be
able to achieve a higher maximum speed using multiple adapters, ZMap is able
to fully saturate a 10\,GigE uplink with a single adapter.

Masscan uses a custom Feistel network to ``encrypt'' a monotonically
increasing index to generate a random permutation of the IPv4 address
space~\cite{masscan-rng}. While this is computation cheaper than using a
cyclic group, this technique results in poor statistical properties, which we
show in Figure~\ref{fig:randomy}. This has two consequences: first, it is not
suitable for sampling portions of the address space, and second, there is
greater potential for overloading destination networks. This could explain
the discrepency in Figure~\ref{fig:hitrate} if Masscan targeted a less
populated subnet.

Masscan and ZMap use a similar sharding approach to parallelize address
generation and distribute scans. Both programs ``count off'' addresses into
shards by staggering the offsets of the starting position of each shard
within the permutation and iterating a fixed number of steps through each of
their permutations. In ZMap, this is implemented by replacing the iteration
factor $g$ with $g^n$. In Masscan, this is simply a matter of incrementing
the monotonically increasing index by more than one.


\section{Applications}
\label{sec:discussion}

In this section, we consider applications that could benefit from 10\,GigE
scanning and remark on the implications of high-speed scanning for defenders
and attackers.

Scanning at faster rates reduces the blur introduced from hosts changing IP
addresses by decreasing the number of hosts that may be doubly counted during
longer scans. This also increases the ability to discover hosts that are only
online briefly. Thus, the ability to complete scans in minutes allows
researchers to more accurately create a snapshot of the Internet at a given
moment.

Similarly, the increased scan rate enables researchers to complete
high-resolution scans when measuring temporal effects. For example, while
researchers were able to complete comprehensive scans for the recent
Heartbleed Vulnerability every few hours~\cite{heartbleed-2014}, many sites
were patched within the first minutes after disclosure. The ability to scan
more rapidly could help shed light on patching behavior within this critical
initial period.

\begin{figure*}
\centering
\includegraphics[width=0.75\linewidth]{\ZippyFigures/graph_image.png}
\caption{\textbf{10\,GigE   Scan Traffic}\,---\,%
An Internet-wide scan at full 10\,GigE speed dwarfed all other traffic at the
university during this 24~hour period. At 14.23~Mpps, a single machine
running ZMap generated 4.6~Gbps in outgoing IP traffic and scanned the entire
public IPv4 address space in 4m29s. The massive increase in outbound traffic
appears to have caused elevated packet drop. Notable smaller spikes are due
to earlier experiments.}
\label{fig:traffic}
\end{figure*}

Faster scan rates also allow for a variety of new scanning-related
applications that require multiple packets, including quickly completing
global trace routes or performing operating system fingerprinting.
Furthermore, the advancement of single-port scanning can be utilized to
quickly perform scans of a large number of ports, allowing scanning all
privileged ports on a /16 in under 5~seconds and \emph{all} ports in
5~minutes, assuming the attacker has sufficient bandwidth to the target.

The most alarming malicious potential for 10\,GigE scanning lies in its
ability to find and exploit vulnerabilities \emph{en masse} in a very short
time. Durumeric et~al.\ found that attackers began scanning for the
Heartbleed vulnerability within 22~hours of its
disclosure~\cite{heartbleed-2014}. While attackers have utilized botnets and
worms in order to complete distributed scans for vulnerabilities, recent
work~\cite{state-of-scanning-2014} has shown that attackers are now also using
ZMap, Masscan, and other scanning technology from bullet-proof hosting
providers in order to find vulnerable hosts. The increase in scan rates could
allow attackers to complete Internet-wide vulnerability scans in minutes as
10\,GigE becomes widely available.\looseness=-1

\section{Future Work}

We demonstrated that it is possible to perform Internet-wide scans at
10\,GigE linespeed, but, at least from our institutional network, we are
unable to sustain the expected hit rate as scanning approaches this packet
rate. Further investigation is needed to understand this effect and profile
ZMap's performance on other networks. One important question is whether the
drop off is caused by nearby network bottlenecks (which might be reduced with
upgraded network hardware) or whether it arises because such rapid scanning
induces congestion on many distant networks---which would represent an
inherent limit on scan speed. It is also possible that there are a small
number of remote bottlenecks that cause the observed drop in hit rate at high
speeds. In that case, identifying, profiling, and removing these bottlenecks
could improve performance.

40\,GigE hardware currently exists, and 100\,GigE is under
development~\cite{hundred-gig}. As these networks become more widely
available, it may be desirable to optimize and scale Internet-wide scanning
to even higher speeds.

\section{Conclusion}

In this work, we introduced enhancements to the ZMap Internet scanner that
enable it to scan at up to 14.2~Mpps. The three modifications we
present---sharding, optimized address constraints, and integration with
PF\_RING ZC---enable scanning at close to 10\,GigE linespeed. These
modifications are available now on experimental ZMap branches and will be
merged into mainline ZMap.

With these enhancements, we are able to complete a scan of the public IPv4
address space in 4m29s. However, despite having a well provisioned upstream
network, coverage in our experiments drops precipitously when scanning faster
than 4~Mpps. While further research is needed to better characterize and
reduce the causes of this drop off, it may be related to specific conditions
on our network.

As faster network infrastructure becomes more widely available, 10\,GigE
scanning will enable powerful new applications for both researchers and
attackers.

%\section*{Acknowledgments}
%
%The authors thank the exceptional sysadmins at the University of Michigan for
%their help and support throughout this project. This research would not have
%been possible without Kevin Cheek, Chris Brenner, Laura Fink, Paul Howell,
%Don Winsor, and others from ITS, CAEN, and DCO\@. We are grateful to Michael
%Bailey for numerous productive discussions, to Luca Deri and ntop for
%providing a PF\_RING license, and to the many contributors to the ZMap open
%source project. We also thank Denis Bueno, Jakub Czyz, Henry Fanson, Pat
%Pannuto, and Eric Wustrow.
%
%This material is based upon work supported by the National Science Foundation
%under Grant No.~CNS-1255153 and No.~CNS-0964545. Any opinions, findings, and
%conclusions or recommendations expressed in this material are those of the
%authors and do not necessarily reflect the views of the National Science
%Foundation.

%% !TEX root = ../../../proposal.tex

{\footnotesize\bibliographystyle{abbrv}
\balance
\bibliography{woot}}
\end{document}



\chapter{Measuring Diffie-Hellman}
\label{chapter:subgroup}
\input{papers/subgroup/paper/subgroup}
\chapter{Measuring Export-Grade Key Exchange}
\label{chapter:logjam}
\newcommand*{\LogjamPaper}{papers/logjam/paper}
\newcommand*{\LogjamFigures}{papers/logjam/figures}
% !TEX root = ../../../proposal.tex

This chapter is adapted from a joint publication that originally appeared in
the proceedings of the 22nd ACM Conference on Computer and Communications
Security (CCS~'15)~\cite{logjam-2015}.
%\begin{abstract}
We investigate the security of Diffie-Hellman key exchange as used in popular
Internet protocols and find it to be less secure than widely believed. First,
we present Logjam, a novel flaw in TLS that lets a man-in-the-middle
downgrade connections to ``export-grade'' Diffie-Hellman. To carry out this
attack, we implement the number field sieve discrete log algorithm. After a
week-long precomputation\footnote{\small Except where otherwise noted, the
experimental data and network measurements for this chapter were obtained in
early 2015.} for a specified 512-bit group, we can compute arbitrary discrete
logs in that group in about a minute. We find that 82\% of vulnerable servers
use a single 512-bit group, allowing us to compromise connections to 7\% of
Alexa Top Million HTTPS sites. In response, major browsers have changed to
reject short groups.
%\looseness=-1

We go on to consider Diffie-Hellman with 768- and 1024-bit groups. We
estimate that even in the 1024-bit case, the computations are plausible given
nation-state resources. A small number of fixed or standardized groups are
used by millions of servers; performing precomputation for a single 1024-bit
group would allow passive eavesdropping on 18\% of popular HTTPS sites, and a
second group would allow decryption of traffic to 66\% of IPsec VPNs and 26\%
of SSH servers. A close reading of published NSA leaks shows that the
agency's attacks on VPNs are consistent with having achieved such a break. We
conclude that moving to stronger key exchange methods should be a priority
for the Internet community.
%\looseness=-1
%\end{abstract}

\section{Introduction}

Diffie-Hellman key exchange is a popular cryptographic algorithm that allows
Internet protocols to agree on a shared key and negotiate a secure
connection. It is fundamental to protocols such as HTTPS, SSH, IPsec, SMTPS,
and other protocols that rely on TLS\@. Many protocols use Diffie-Hellman to
achieve \emph{perfect forward secrecy}, the property that a compromise of the
long-term keys used for authentication does not compromise sessions keys for
past connections. We examine how Diffie-Hellman is commonly implemented and
deployed with common protocols and find that, in practice, it frequently
offers less security than widely believed.

There are two reasons for this. First, a surprising number of servers use
weak Diffie-Hellman parameters or maintain support for obsolete 1990s-era
``export-grade'' crypto. More critically, the common practice of using
standardized, hard-coded, or widely shared Diffie-Hellman parameters has the
effect of dramatically reducing the cost of large-scale attacks, bringing
some within range of feasibility today.

The current best technique for attacking Diffie-Hellman relies on
compromising one of the private exponents ($a$, $b$) by computing the
discrete logarithm of the corresponding public value ($g^a \bmod p$, $g^b
\bmod p$). With state-of-the-art number field sieve algorithms, computing a
single discrete log is more difficult than factoring an RSA modulus of the
same size. However, an adversary who performs a large precomputation for a
prime $p$ can then quickly calculate arbitrary discrete logs in that group,
amortizing the cost over all targets that share this parameter. Although this
fact is well known among mathematical cryptographers, it seems to have been
lost among practitioners deploying cryptosystems. We exploit it to
obtain the following results:

\paragraph{Active attacks on export ciphers in TLS}
We introduce Logjam, a new attack on TLS by which a man-in-the-middle
attacker can downgrade a connection to export-grade cryptography. This attack
is reminiscent of the FREAK attack~\cite{freak-attack-2015} but applies to
the ephemeral Diffie-Hellman ciphersuites and is a TLS protocol flaw rather
than an implementation vulnerability. We present measurements that show that
this attack applies to 8.4\% of Alexa Top Million HTTPS sites and 3.4\% of
all HTTPS servers that have browser-trusted certificates.

To exploit this attack, we implemented the number field sieve discrete log
algorithm and carried out precomputation for two 512-bit Diffie-Hellman
groups used by more than 92\% of the vulnerable servers. This allows us to
compute individual discrete logs in about a minute. Using our discrete log
oracle, we can compromise connections to over 7\% of Top Million HTTPS sites.
Discrete logs over larger groups have been computed before~\cite{dlp180},
but, as far as we are aware, this is the first time they have been exploited
to expose concrete vulnerabilities in real-world systems.
%\looseness=-1


\begin{figure*}[ht]
\centering\includegraphics[width=\linewidth]{\LogjamFigures/nfs}

\caption{\textbf{Number field sieve for discrete log}\,---\,%
This algorithm consists of a precomputation stage that depends only on the
prime $p$ and a descent stage that computes individual logarithms. With
sufficient precomputation, an attacker can quickly break any Diffie-Hellman
instances that use a particular $p$.
}
\label{fig:nfs}
\end{figure*}

\paragraph{Risks from common 1024-bit groups}
We explore the implications of precomputation attacks for 768- and 1024-bit
groups, which are widely used in practice and still considered secure. We
estimate the computational resources necessary to compute discrete logs in
groups of these sizes, concluding that 768-bit groups are within range of
academic teams, and 1024-bit groups may plausibly be within range of
nation-state adversaries. In both cases, individual logarithms can be quickly
computed after the initial precomputation.

We then examine evidence from published Snowden documents that suggests NSA
may already be exploiting 1024-bit Diffie-Hellman to decrypt VPN traffic. We
perform measurements to understand the implications of such an attack for
popular protocols, finding that an attacker who could perform precomputations
for ten 1024-bit groups could passively decrypt traffic to about 66\% of IKE
VPNs, 26\% of SSH servers, and 24\% of popular HTTPS sites.

\paragraph{Mitigations and lessons}
In response to the Logjam attack, mainstream browsers have implemented a more
restrictive policy on the size of Diffie-Hellman groups they accept, and
Chrome has discontinued support for finite field key exchanges. We further
recommend that TLS servers disable export-grade cryptography and carefully
vet the Diffie-Hellman groups they use. In the longer term, we advocate that
protocols migrate to elliptic curve Diffie-Hellman.

\section{Diffie-Hellman Cryptanalysis}
\label{sec:dl}

Diffie-Hellman key exchange was the first published public-key
algorithm~\cite{new-directions-in-crypto-1976}. In the simple case of prime
groups, Alice and Bob agree on a prime $p$ and a generator $g$ of a
multiplicative subgroup modulo $p$. Then each generates a random private
exponent, $a$ and $b$. Alice sends $g^a \bmod p$, Bob sends $g^b \bmod p$,
and each computes a shared secret $g^{ab} \bmod p$. While there is also a
Diffie-Hellman exchange over elliptic curve groups, we address only the ``mod
$p$'' case.

The security of Diffie-Hellman is not known to be equivalent to the discrete
logarithm problem, but computing discrete logs remains the best known
cryptanalytic attack. An attacker who can find the discrete log $x$ from $y =
g^x \bmod p$ can easily find the shared secret.

Textbook descriptions of discrete log can be misleading about the
computational tradeoffs, for example by optimizing for computing a
\emph{single} discrete log. In fact, as illustrated in Figure~\ref{fig:nfs},
a single large precomputation on $p$ can be used to efficiently break
\emph{all} Diffie-Hellman exchanges made with that prime.

Diffie-Hellman is typically implemented with prime fields and large group
orders. In this case, the most efficient discrete log algorithm is the number
field sieve
(NFS)~\cite{discrete-log-nfs-1993,virtual-logarithms-2005,nfs-prime-field-2003}.
The algorithm has four stages with different computational properties. The
first three steps are only dependent on the prime $p$ and comprise most of
the computation.

First is \emph{polynomial selection}, in which one finds a polynomial $f(z)$
defining a number field $\QQ[z]/f(z)$ for the computation. This parallelizes
well and is only a small portion of the runtime.

In the second stage, \emph{sieving}, one factors ranges of integers and
number field elements in batches to find many relations of elements, all of
whose prime factors are less than some bound $B$ (called $B$-smooth). Sieving
parallelizes well, but is computationally expensive, because we must search
through and attempt to factor many elements.

In the third stage, \emph{linear algebra}, we construct a large, sparse
matrix consisting of the coefficient vectors of prime factorizations we have
found. This stage can be parallelized in a limited fashion, and produces a
database of logarithms which are used as input to the final stage.

The final stage, \emph{descent}, actually deduces the discrete log of the
target $y$. We re-sieve until we find a set of relations that allow us to
write the logarithm of $y$ in terms of the logarithms in the precomputed
database. Crucially, descent is the only NFS stage that involves $y$ (or
$g$), so polynomial selection, sieving, and linear algebra can be done once
for a prime $p$ and reused to compute the discrete logs of many targets.

The numerous parameters of the algorithm allow some flexibility to reduce
time on some computational steps at the expense of others. For example,
sieving more will result in a smaller matrix, making linear algebra cheaper,
and doing more work in the precomputation makes the final descent step
easier.

\paragraph{Standard primes}
Generating safe primes\footnote{\small An odd prime $p$ is safe when
$(p-1)/2$ is prime.} can be computationally burdensome, so many
implementations use standardized Diffie-Hellman parameters. A prominent
example is the Oakley groups~\cite{rfc2412}, which give ``safe'' primes of
length 768 (Oakley Group 1), 1024 (Oakley Group 2), and 1536 (Oakley Group
5). These groups were published in 1998 and have been used for many
applications since, including IKE, SSH, Tor, and OTR\@.

When primes are of sufficient strength, there seems to be no
disadvantage to reusing them.  However, widespread reuse of
Diffie-Hellman groups can convert attacks that are at the limits of an
adversary's capabilities into devastating breaks, since it allows the
attacker to amortize the cost of discrete log precomputation among
vast numbers of potential targets.

\section{Attacking TLS}
\label{sec:attacking-tls}

TLS supports Diffie-Hellman as one of several possible key exchange
methods, and prior to public disclosure of the attack, about two-thirds of popular HTTPS sites supported it, most
commonly using 1024-bit primes.  However, a smaller number of servers
also support legacy ``export-grade'' Diffie-Hellman using 512-bit
primes that are well within reach of NFS-based
cryptanalysis. Furthermore, for both normal and export-grade
Diffie-Hellman, the vast majority of servers use a handful of common
groups.

In this section, we exploit these facts to construct a novel attack against
TLS\@, which we call the Logjam attack. First, we perform NFS precomputations
for the two most popular 512-bit primes on the web, so that we can quickly
compute the discrete log for any key exchange message that uses one of them.
Next, we show how a man-in-the-middle, so armed, can attack connections
between popular browsers and any server that allows export-grade
Diffie-Hellman, by using a TLS protocol flaw to downgrade the connection to
export-strength and then recovering the session key. We find that this attack
with our precomputations can compromise connections to about 7.8\% of HTTPS
servers among Alexa Top Million domains.

\begin{table}[t]
	\centering\small
	\begin{tabular}{lll}
          \toprule
          Source  & Popularity & Prime \\
          \midrule
          Apache   & 82\% & \tt 9fdb8b8a004544f0045f1737d0ba2e0b\\
                   &      & \tt 274cdf1a9f588218fb435316a16e3741\\
                   &      & \tt 71fd19d8d8f37c39bf863fd60e3e3006\\
                   &      & \tt 80a3030c6e4c3757d08f70e6aa871033\smallskip\\
          mod\_ssl & 10\% & \tt d4bcd52406f69b35994b88de5db89682\\
                   &      & \tt c8157f62d8f33633ee5772f11f05ab22\\
                   &      & \tt d6b5145b9f241e5acc31ff090a4bc711\\
                   &      & \tt 48976f76795094e71e7903529f5a824b\smallskip\\
          (\emph{others\/}) & \ 8\% & (463~distinct primes) \\
          \bottomrule
	\end{tabular}
    \caption{\textbf{Top 512-bit DH primes for TLS}\,---\,%
        8.4\% of Alexa Top~1M HTTPS domains allow \dheexp{}, of which 92.3\% use
        one of the two most popular primes, shown here.
    }
    \label{tab:export-primes}
\end{table}

\subsection{TLS and Diffie-Hellman}

The TLS handshake begins with a negotiation to determine the crypto
algorithms used for the session. The client sends a list of supported
ciphersuites (and a random nonce $cr$) within the \textsf{ClientHello}
message, where each ciphersuite specifies a key exchange algorithm and other
primitives. The server selects a ciphersuite from the client's list and
signals its selection in a \textsf{ServerHello} message (containing a random
nonce $sr$).

TLS specifies ciphersuites supporting multiple varieties of Diffie-Hellman.
Textbook Diffie-Hellman with unrestricted strength is called ``ephemeral''
Diffie-Hellman, or \dhe{}, and is identified by ciphersuites that begin with
\texttt{TLS\_DHE\_*}.\footnote{\small New ciphersuites that use elliptic
curve Diffie-Hellman (\ecdhe{}) are gaining in popularity, but we focus
exclusively on the traditional prime field variety.} In \dhe{}, the server is
responsible for selecting the Diffie-Hellman parameters. It chooses a group
$(p,g)$, computes $g^b$, and sends a \textsf{ServerKeyExchange} message
containing a signature over the tuple $(cr, sr, p, g, g^b)$ using the
long-term signing key from its certificate. The client verifies the signature
and responds with a \textsf{ClientKeyExchange} message containing $g^a$.

To ensure agreement on the negotiation messages, and to prevent downgrade
attacks, each party computes the TLS master secret from $g^{ab}$ and
calculates a MAC of its view of the handshake transcript. These MACs are
exchanged in a pair of \textsf{Finished} messages and verified by the
recipients.

\paragraph{Export-grade Diffie-Hellman}
To comply with 1990s-era U.S. export restrictions on cryptography, SSL 3.0
and TLS 1.0 supported reduced-strength \dheexp{} ciphersuites that were
restricted to primes no longer than 512 bits. In all other respects,
\dheexp{} protocol messages are identical to \dhe{}. The relevant export
restrictions are no longer in effect, but many servers maintain support for
backwards compatibility.

To understand how HTTPS servers in the wild use Diffie-Hellman, we modified
the ZMap~\cite{zmap-2013} toolchain to offer \dhe{} and \dheexp{}
ciphersuites and scanned TCP/443 on both the full public IPv4 address space
and the Alexa Top~1M domains. The scans took place in March 2015. Of 539,000
HTTPS sites among Top~1M domains, we found that 68.3\% supported \dhe{} and
8.4\% supported \dheexp{}. Of 14.3~million IPv4 HTTPS servers with
browser-trusted certificates, 23.9\% supported \dhe{} and 4.9\% \dheexp{}.

\iffalse
\begin{table}[t]
	\centering\small
	\begin{tabular}{rllrl}
    \toprule
	Fraction & Source & Year & Bits & Prime \\
    \midrule
0.8255 & Apache 2.2 & 2005 & 512 & \texttt{9fdb8b8a}$\ldots$\texttt{aa871033} \\
0.0997 & mod\_ssl & 1999 & 512 & \texttt{d4bcd524}$\ldots$\texttt{9f5a824b}\\
0.0414 & IKE & 2000 & 2048 & \texttt{fff}$\ldots$\texttt{c90fdaa2}$\ldots$\texttt{fff} \\
0.0069 & JDK & 2003 & 512 & \texttt{fca682ce}$\ldots$\texttt{37592e17} \\
0.0012 & (unknown)& --- & 512 & \texttt{acc8149e}$\ldots$\texttt{67ec1505} \\
\midrule
0.0253 & \multicolumn{4}{c}{\emph{other primes}} \\
    \bottomrule
	\end{tabular}
    \label{tab:export-primes}
\end{table}
\fi

While the TLS protocol allows servers to generate their own Diffie-Hellman
parameters, just two 512-bit primes account for 92.3\% of Alexa Top~1M
domains that support \dheexp{} (Table~\ref{tab:export-primes}), and 92.5\% of
all servers with browser-trusted certificates that support \dheexp{}. The
most popular 512-bit prime was hard-coded into many versions of Apache; the
second most popular is the \texttt{mod\_ssl} default for \dheexp{}.

\subsection{Active Downgrade to Export-Grade DHE}
\label{sec:dhead}

\begin{figure}[t]
\includegraphics[width=\linewidth]{\LogjamFigures/mitm-dhe-export}
    \caption{\textbf{The Logjam attack}\,---\,%
    A man-in-the-middle can force TLS clients to use export-strength DH with
    any server that allows \dheexp{}. Then, by finding the 512-bit discrete
    log, the attacker can learn the session key and arbitrarily read or
    modify the contents. $\AData^{fs}$ refers to False Start application data
    that some TLS clients send before receiving the server's
    \textsf{Finished} message.}
    \label{fig:mitm-export}
\end{figure}

Given the widespread use of these primes, an attacker with the ability to
compute discrete logs in 512-bit groups could efficiently break \dheexp{}
handshakes for about 8\% of Alexa Top~1M HTTPS sites, but modern browsers
never negotiate export-grade ciphersuites. To circumvent this, we show how an
attacker can downgrade a regular \dhe{} connection to use a \dheexp{} group,
and thereby break both the confidentiality and integrity of application data.

The attack, which we call Logjam, is depicted in Figure~\ref{fig:mitm-export}
and relies on a flaw in the way TLS composes \dhe{} and \dheexp{}. When a
server selects \dheexp{} for a handshake, it proceeds by issuing a signed
\textsf{ServerKeyExchange} message containing a 512-bit $p_{512}$, but the
structure of this message is identical to the message sent during standard
\dhe{} ciphersuites. Critically, the signed portion of the server's message
fails to include any indication of the specific ciphersuite that the server
has chosen. Provided that a client offers \dhe{}, an active attacker can
rewrite the client's \textsf{ClientHello} to offer a corresponding \dheexp{}
ciphersuite accepted by the server and remove other ciphersuites that could
be chosen instead. The attacker rewrites the \textsf{ServerHello} response to
replace the chosen \dheexp{} ciphersuite with a matching non-export
ciphersuite and forwards the \textsf{ServerKeyExchange} message to the client
as is. The client will interpret the export-grade tuple $(p_{512}, g, g^b)$
as valid \dhe{} parameters chosen by the server and proceed with the
handshake. The client and server have different handshake transcripts at this
stage, but an attacker who can compute $b$ in close to real time can then
derive the master secret and connection keys to complete the handshake with
the client.

There are two remaining challenges in implementing this active downgrade
attack. The first is to compute individual discrete logs in close to real
time, and the second is to delay handshake completion until the discrete log
computation has had time to finish.

\subsection{512-bit Discrete Log Computations}
\label{subsec:512bit-dl-computation}

We modified CADO-NFS~\cite{cado-nfs-2.3} to implement the number field sieve
discrete log algorithm and applied it to the top two \dheexp{} primes shown
in Table~\ref{tab:export-primes}. Precomputation took 7~days for each prime,
after which computing individual logarithms requires a median of 70~seconds.

\paragraph{Precomputation}
As illustrated in Figure~\ref{fig:nfs}, the precomputation phase includes the
polynomial selection, sieving, and linear algebra steps. For this
precomputation, we deliberately sieved more than strictly necessary. This
enabled two optimizations: first, with more relations obtained from sieving,
we eventually obtain a larger database of known logarithms, which makes the
descent faster. Second, more sieving relations also yield a smaller linear
algebra step, which is desirable because sieving is much easier to
parallelize than linear algebra.
%\looseness=-1

For the polynomial selection and sieving steps, we used idle time on
2000--3000 CPU cores in parallel. Polynomial selection ran for about 3~hours
(7,600 core-hours). Sieving ran for 15~hours (21,400 core-hours). This
sufficed to collect 40\,M relations of which 28\,M were unique, involving
15\,M primes of at most 27~bits.

From this data set, we obtained a square matrix with 2.2\,M rows and columns,
with 113 nonzero coefficients per row on average. We solved the corresponding
linear system on a 36-node cluster using the block Wiedemann
algorithm~\cite{coppersmith-block-wiedemann-1994,thome-block-wiedemann-2002}.
Using unoptimized code, the computation finished in 120 hours (60,000
core-hours).

The experiment above was done with CADO-NFS in early 2015. As of 2017,
release 2.3 of CADO-NFS~\cite{cado-nfs-2.3} performs 20\% faster for sieving,
and drastically faster for linear algebra, since 9,000 core-hours suffice to
solve the same linear system on the same hardware. In total, the wall-clock
time for each precomputation was slightly over one week in 2015, and is
reduced to about two days with current hardware and more recent software.

\paragraph{Descent}
Once this precomputation was finished, we were able to run the final descent
step to compute individual discrete logs in about a minute. We implemented
the descent calculation in a mix of Python and C\@. On average, computing
individual logarithms took about 70~seconds, but the time varied from 34 to
206 seconds on a server with two 18-core Intel Xeon E5-2699 CPUs. For
purposes of comparison, a single 512-bit RSA factorization using the CADO-NFS
implementation takes about 4 days of wall-clock time on the computer used for
the descent\cite{cado-nfs-2.3}.

% reference: revision 580f1f1 of CADO-NFS with tasks.sieve.qmin=15470309
% on a catrel node:
% Total cpu/elapsed time for entire factorization: 1.06608e+07/351570
% 351570/3600/24 = 4.07

\subsection{Active Attack Implementation}
The main challenge in performing this attack is to compute the shared secret
$g^{ab}$ before the handshake completes in order to forge a \textsf{Finished}
message from the server. With our descent implementation, the computation
takes an average of 70 seconds, but there are several ways an attacker can
work around this delay:

\paragraph{Non-browser clients}
Different TLS clients impose different time limits, after which they kill the
connection. Command-line clients such as \texttt{curl} and \texttt{git} have
long or no timeouts, and we can hijack their connections without difficulty.

\paragraph{TLS warning alerts}
Web browsers tend to have shorter timeouts, but we can keep their connections
alive by sending TLS warning alerts, which are ignored by the browser but
reset the handshake timer. For example, this allows us to keep Firefox TLS
connections alive indefinitely.

\paragraph{Ephemeral key caching}
Many TLS servers do not use a fresh value $b$ for each connection, but
instead compute $g^b$ once and reuse it for multiple negotiations. For
example, F5 BIG-IP load balancers will reuse $g^b$ by default. Microsoft
Schannel caches $g^b$ for two hours---this setting is hard-coded. For these
servers, an attacker can compute the discrete log of $g^b$ from one
connection and use it to attack later handshakes.

\paragraph{TLS False Start}
Even when clients enforce shorter timeouts and servers do not reuse values
for $b$, the attacker can still break the confidentiality of user requests
that use TLS False Start. Recent versions of Chrome, Internet Explorer, and
Firefox implement False Start, but their policies on when to enable it vary.
Firefox 35, Chrome 41, and Internet Explorer (Windows 10) send False Start
data with \dhe{}\@. In these cases, a man-in-the-middle can record the
handshake and decrypt the False Start payload at leisure.

\section{Nation-State Threats to DH}
\label{sec:nationstate}

The previous sections demonstrate the existence of practical attacks against
Diffie-Hellman key exchange as currently used by TLS\@. However, these
attacks rely on the ability to downgrade connections to export-grade crypto.
In this section we address the following question: how secure is
Diffie-Hellman in broader practice, as used in other protocols that do not
suffer from downgrade, and when applied with stronger groups?

\begin{table*}
\centering\small
\begin{tabular}{rrrrrrl@{}}
\toprule
  & \multicolumn{2}{c}{Sieving} & \multicolumn{2}{c}{Linear Algebra} & \multicolumn{1}{c}{Descent} \\
 \cmidrule(r){2-3} \cmidrule(r){4-5} \cmidrule(){6-6}
 & $\log_2B$ & core-years  & rows & core-years  & core-time   \\
\midrule
% timings below from CADO-NFS revision 580f1f1 on a catrel node
% with tasks.sieve.qmin=15470309
% Info:Lattice Sieving: Total CPU time: 9.53745e+06s -> 0.30 cpu year
% Info:Filtering - Merging: Merged matrix has 4146413 rows and total weight 704890615 (170.0 entries per row on average)
% Info:Linear Algebra: Krylov: WCT time 21706.76
% Info:Linear Algebra: Lingen CPU time 4834.58, WCT time 218.43
% Info:Linear Algebra: Mksol: WCT time 11599.56
% Linear Algebra: total WCT = 33524.75
% thus cpu <= 32*WCT <= 1.072792e6 (32 threads) <= 0.03 cpu year
\multicolumn{1}{r}{RSA-512} &
29 & 0.3 &  4.2M & 0.03 &  
& \scriptsize\quad Timings with default CADO-NFS parameters.\\
\multicolumn{1}{r}{DH-512} &
27& 2.5 &  2.2M & 1.1 &  10\,mins
& \scriptsize\quad For the computations in this paper; may be suboptimal.\\
\cmidrule{1-7}
\multicolumn{1}{r}{RSA-768} & 
37& 800 & 250M & 100 &
& \scriptsize\quad Est.\ based on~\cite{rsa768} with less sieving.  \\
%\multicolumn{1}{r}{DH-768} &
%  35 & 8,000 & 150M & 28,500 & 2\,days
%& \scriptsize\quad Est.\ based on~\cite{rsa768,dlp180} and our own experiments. \\
\multicolumn{1}{r}{DH-768} &
36 & 4,000 & 24M & 920 & 43\,hours
    & \scriptsize\quad Data from~\cite[Table 1]{Kleinjung2017}.\\
\cmidrule{1-7}
\multicolumn{1}{r}{RSA-1024} &
42& $\approx$1,000,000  & $\approx$8.7B & $\approx$120,000 & 
& \scriptsize\quad Crude estimate based on complexity formula. \\
%\multicolumn{1}{r}{DH-1024} &
%   40 & 10,000,000 & 5.2B & 35,000,000 & 30\,days
%& \scriptsize\quad Est.\ based on complexity formula and our experiments. \\
\multicolumn{1}{r}{DH-1024} &
   40 & $\approx$5,000,000 & $\approx$0.8B & $\approx$1,100,000 & 30\,days
& \scriptsize\quad Crude estimate based on formula and our experiments. \\
\bottomrule
\end{tabular}

\caption{\textbf{Estimating costs for factoring and discrete log}.
For sieving, we give one important parameter, which is the number of bits of the
smoothness bound
{\tt B}.
For linear algebra, all costs for DH are for
safe primes; for DSA primes with $q$ of 160 bits, this should be
divided by 6.4 for 1024 bits, 4.8 for 768 bits, and 3.2 for 512 bits.}

\label{tab:costs}
\end{table*}



To answer this question we must first examine how the number field sieve for
discrete log scales to 768- and 1024-bit groups. As we argue below, 768-bit
groups in relatively widespread use are now within reach for academic
computational resources. Additionally, performing precomputations for a small
number of 1024-bit groups is plausibly within the resources of nation-state
adversaries. The precomputation would likely require special-purpose
hardware, but would not require any major algorithmic improvements. In light
of these results, we examine several standard Internet security
protocols---IKE, SSH, and TLS---to determine their vulnerability. Although
the cost of the precomputation for a 1024-bit group is several times higher
than for an RSA key of equal size, a one-time investment could be used to
attack millions of hosts, due to widespread reuse of the most common
Diffie-Hellman parameters. Finally, we apply this new understanding to a set
of recently published documents to evaluate the hypothesis that the National
Security Agency has {\em already} implemented such a capability.

\subsection{Scaling NFS to 768- and 1024-bit DH}

Estimating the cost for discrete log cryptanalysis at larger key sizes is far
from straightforward due to the complexity of parameter tuning. We attempt
estimates up to 1024-bit discrete log based on the existing literature and
our own experiments but further work is needed for greater confidence. We
summarize all the costs, measured or estimated, in Table~\ref{tab:costs}.

\paragraph{DH-768: Completed in 2016}
At the time of disclosure, the latest discrete log record was a 596-bit
computation. Based on that work, and on prior experience with the 768-bit
factorization record in 2009~\cite{factor-rsa-768}, we made the conservative
prediction that it was possible, as explained in~\S\ref{sec:dl}, to put more
computational effort into sieving for the discrete log case than for
factoring, so that the linear algebra step would run on a slightly smaller
matrix. This led to a runtime estimate of around 35,000 core-years, most of
which was spent on linear algebra.

This estimate turned out be overly conservative, for several reasons. First,
there have been significant improvements in our software implementation
(see~\S\ref{subsec:512bit-dl-computation}). In addition, our estimate did not
use the Joux-Lercier alternative polynomial selection
method~\cite[\S2.1]{nfs-prime-field-2003}, which is specific to discrete
logs. For 768-bit discrete logs, this polynomial selection method leads to a
significantly smaller computational cost.

In 2016, Kleinjung et al.\ completed a 768-bit discrete log
computation~\cite{compute-dlog-768}. While this is a massive computation on
the academic scale, a computation of this size has likely been within reach
of nation-states for more than a decade. This data is mentioned in
Table~\ref{tab:costs}.

\paragraph{DH-1024: Plausible with nation-state resources}
Experimentally extrapolating sieving parameters to the 1024-bit case is
difficult due to the tradeoffs between the steps of the algorithm and their
relative parallelism. The prior work proposing parameters for factoring a
1024-bit RSA key is thin and we resort to extrapolating from asymptotic
complexity. For the number field sieve, the complexity is
$\exp\big((k+o(1))(\log N)^{1/3}(\log\log N)^{2/3}\big),$ where $N$ is the
integer to factor or the prime modulus for discrete log and $k$ is an
algorithm-specific constant. This formula is inherently imprecise, since the
$o(1)$ in the exponent can hide polynomial factors. This complexity formula,
with $k=1.923$, describes the overall time for both discrete log and
factorization, which are both dominated by sieving and linear algebra in the
precomputation. Evaluating the formula for 768- and 1024-bit $N$ gives us
estimated multiplicative factors by which time and space will increase from
the 768- to the 1024-bit case.

For 1024-bit precomputation, the total time complexity can be expected to
increase by a factor of 1220 using the complexity formula, while space
complexity increases by its square root, approximately 35. These ratios are
relevant for both factorization and discrete log since they have the same
asymptotic behavior. For DH-1024, we get a total cost estimate for the
precomputation of about 6M core-years.

The time complexity for each individual log after the precomputation should
be multiplied by $L_{2^{1024}}(1.206)/L_{2^{768}}(1.206)\approx86$, where
$k=1.206$ follows from~\cite{hidden-snfs-1024-2017}. This last number does
not correspond to what we observed in practice and we attribute that to the
fact that the descent step has been far less studied.

In practice, it is not uncommon for estimates based merely on the complexity
formula to be off by a factor of 10. Estimates of Table~\ref{tab:costs} must
therefore be considered with due caution.

For 1024-bit descent, we experimented with our early-abort implementation to
inform our estimates for descent initialization, which should dominate the
individual discrete log computation. For a random target in Oakley Group~2,
initialization took 22 core-days, and yielded a few primes of at most 130
bits to be descended further. In twice this time, we reached primes of about
110 bits. At this point, we were certain to have bootstrapped the descent and
could continue down to the smoothness bound in a few more core-days if proper
sieving software were available. Thus we estimate that a 1024-bit descent
would take about 30~core-days, once again easily parallelizable.

\paragraph{Costs in hardware}
Although several millions of core-years is a massive computational effort, it
is not necessarily out of reach for a nation-state. At this scale,
significant cost savings could be realized by developing application-specific
hardware given that sieving is a natural target for hardware implementation.
To our knowledge, the best prior description of an ASIC implementation of
1024-bit sieving is the 2007 work of Geiselmann and
Steinwandt~\cite{hardware-scaling-nfs-2007}. Updating their estimates for
modern techniques and adjusting parameters for discrete log allows us to
extrapolate the financial and time costs.

We increase their chip count by a factor of ten to sieve more and save on
linear algebra as above giving an estimate of 3M chips to complete sieving in
one year. Shrinking the dies from the 130~nm technology node used in the
paper to a more modern size reduces costs as transistors are cheaper at newer
technologies. With standard transistor costs and utilization, it would cost
about \$2 per chip to manufacture after fixed design and tape-out costs of
roughly \$2M~\cite{jefferies-report-2012}. This suggests that an \$8M
investment would buy enough ASICs to complete the DH-1024 sieving
precomputation in one year. Since a step of descent uses sieving, the same
hardware could likely be reused to speed calculations of individual
logarithms.

Estimating the financial cost for the linear algebra is more difficult since
there has been little work on designing chips that are suitable for the
larger fields involved in discrete log. To derive a rough estimate, we can
begin with general purpose hardware and the core-year estimate from
Table~\ref{tab:costs}. Using the 300,000 CPU core Titan supercomputer it
would take 4~years to complete the 1024-bit linear algebra stage
(notwithstanding the fact that estimates from Table~\ref{tab:costs} are known
to be extremely coarse, and could be optimistic by a factor of maybe 10).
Titan was constructed in 2012 for \$94M, suggesting a cost of \$0.5B in
supercomputers to finish this step in a year. In the context of
factorization, moving linear algebra from general purpose CPUs to ASICs has
been estimated to reduce costs by a factor of
80~\cite{improving-linear-algebra-nfs-2005}. If we optimistically assume that
a similar reduction can be achieved for discrete log, the hardware cost to
perform the linear algebra for DH-1024 in one year is plausibly on the order
of tens of millions of dollars.

To put this dollar figure in context, the FY\,2012 budget for the U.S.
Consolidated Cryptologic Program (which includes the NSA) was \$10.5
billion\footnote{\small The National Science Foundation's budget was \$7
billion.}~\cite{black-budget-leak-2013}. The 2013 budget request, which
prioritized investment in ``groundbreaking cryptanalytic capabilities to
defeat adversarial cryptography and exploit internet traffic'' included
notable \$100M+ increases in two programs under Cryptanalysis \& Exploitation
Services: ``Cryptanalytic IT Systems'' (to \$247M), and the cryptically named
``PEO Program C'' (to \$360M)~\cite{black-budget-leak-2013}.

\subsection{Is NSA Breaking 1024-bit DH?}

Our calculations suggest that it is plausibly within NSA's resources to have
performed number field sieve precomputations for a small number of 1024-bit
Diffie-Hellman groups. This would allow them to break any key exchanges made
with those groups in close to real time. If true, this would answer one of
the major cryptographic questions raised by the Edward Snowden leaks: How is
NSA defeating the encryption for widely used VPN protocols?

Virtual private networks (VPNs) are widely used for tunneling business or
personal traffic across potentially hostile networks. We focus on the
Internet Protocol Security (IPsec) VPN protocol using the Internet Key
Exchange (IKE) protocol for key establishment and parameter negotiation and
the Encapsulating Security Payload (ESP) protocol for protecting packet
contents.

\paragraph{IKE}
There are two versions, IKEv1 and IKEv2, which differ in message structure
but are conceptually similar. For the sake of brevity, we will use IKEv1
terminology~\cite{rfc7296}.

\newcommand{\skeyid}{\textsf{\small SKEYID}}
\newcommand{\keymat}{\textsf{\small KEYMAT}}
\newcommand{\psk}{\textsf{\small PSK}}

Each IKE session begins with a Phase~1 handshake in which the client and
server select a Diffie-Hellman group from a small set of standardized
parameters and perform a key exchange to establish a shared secret. The
shared secret is combined with other cleartext values transmitted by each
side, such as nonces and cookies, to derive a value called \skeyid\@. When
authenticated with a pre-shared key (PSK) in IKEv1, the PSK value is
incorporated into the derivation of \skeyid.

The resulting \skeyid\ is used to encrypt and authenticate a Phase~2
handshake. Phase~2 establishes the parameters and key material, \keymat, for
protecting the subsequently tunneled traffic. Ultimately, \keymat{} is
derived from \skeyid, additional nonces, and the result of an optional
Phase~2 Diffie-Hellman exchange.

\paragraph{NSA's VPN exploitation process}
Documents published by Der Spiegel describe NSA's ability to decrypt VPN
traffic using passive eavesdropping and without message injection or
man-in-the-middle attacks on IPsec or IKE\@. Figure~\ref{fig:scarynsafigure}
illustrates the flow of information required to decrypt the tunneled traffic.

\begin{figure}
    \noindent\includegraphics[width=\linewidth]{\LogjamFigures/NSA_combined.png}
    \caption{\textbf{NSA's VPN decryption infrastructure}\,---\,%
    This classified illustration published by Der
    Spiegel~\cite{media-35526} shows captured IKE handshake messages
    being passed to a high-performance computing system, which returns the
    symmetric keys for ESP session traffic. The details of this attack are
    consistent with an efficient break for 1024-bit Diffie-Hellman.}
    \label{fig:scarynsafigure}
\end{figure}

\begin{landscape}
\begin{table*}[t]
\small
\centering
\begin{tabularx}{\linewidth}{lrrrr}
\toprule
              & \multicolumn{4}{c}{\emph{Vulnerable servers, if the attacker can precompute for \ldots}} \\
\cmidrule{2-5}
%\multicolumn{1}{r}{\emph{If the attacker can precompute \ldots.}}
              & all 512-bit groups & all 768-bit groups & one 1024-bit group & ten 1024-bit groups\\  
\midrule
HTTPS Top~1M w/ active downgrade\qquad\strut & 45,100 (8.4\%) & 45,100 (8.4\%) & 205,000 (37.1\%) & 309,000 (56.1\%) \\
HTTPS Top~1M & 118 (0.0\%) & 407 (0.1\%) & 98,500 (17.9\%) & 132,000 (24.0\%) \\
HTTPS Trusted w/ active downgrade & 489,000 (3.4\%) & 556,000 (3.9\%) & 1,840,000 (12.8\%) & 3,410,000 (23.8\%) \\ 
HTTPS Trusted &  1,000 (0.0\%) & 46,700 (0.3\%) & 939,000 (6.56\%) & 1,430,000 (10.0\%)\medskip\\
IKEv1 IPv4  & -- & 64,700 (2.6\%) & 1,690,000 (66.1\%)& 1,690,000 (66.1\%) \\
IKEv2 IPv4  & -- & 66,000 (5.8\%) & 726,000 (63.9\%) & 726,000 (63.9\%)\medskip \\
SSH   IPv4  & -- & -- & 3,600,000 (25.7\%) & 3,600,000 (25.7\%) \\
\bottomrule

\end{tabularx}
\caption{\textbf{Estimated impact of Diffie-Hellman attacks in early 2015}\,---\,%
We used Internet-wide scanning to estimate the number of real-world servers for which 
typical connections could be compromised by attackers with various levels of computational
resources. For HTTPS, we provide figures with and without downgrade attacks on the chosen ciphersuite. All others are passive attacks.
}
\end{table*}
\end{landscape}


When the IKE/ESP messages of a VPN of interest are collected, the IKE
messages and a small amount of ESP traffic are sent to the Cryptanalysis and
Exploitation Services (CES)~\cite{media-35526,media-35529,media-35515}.
Within the CES enclave, a specialized ``attack orchestrator'' attempts to
recover the ESP decryption key with assistance from high-performance
computing resources as well as a database of known PSKs
(``CORALREEF'')~\cite{media-35529,media-35526,media-35515}. If the recovery
was successful, the decryption key is returned from CES and used to decrypt
the buffered ESP traffic such that the encapsulated content can be
processed~\cite{media-35529,media-35522}.

\paragraph{Evidence for a discrete log attack}
The ability to decrypt VPN traffic does not necessarily indicate a defeat of
Diffie-Hellman. There are, however, several features of the described
exploitation process that support this hypothesis.

The IKE protocol has been extensively
analyzed~\cite{ike-security-2002,ike-nrl-1999} and is not believed to be
exploitable in standard configurations under passive eavesdropping attacks.
Absent a vulnerability in the key derivation function or transport
encryption, the attacker must recover the decryption keys. This requires the
attacker to calculate \skeyid{} generated from the Phase~1 Diffie-Hellman
shared secret after passively observing an IKE handshake.

While IKE is designed to support a range of Diffie-Hellman groups, our
Internet-wide scans (\S\ref{sec:1024effects}) show that the vast majority of
IKE endpoints select one particular 1024-bit DH group even when offered
stronger groups. Conducting an expensive, but feasible, precomputation for
this single 1024-bit group (Oakley Group 2) would allow the efficient
recovery of a large number of Diffie-Hellman shared secrets used to derive
\skeyid{} and the subsequent \keymat.

Given an efficient oracle for solving the discrete logarithm problem, attacks
on IKE are possible provided that the attacker can obtain the following:
$(1)$ a complete two-sided IKE transcript, and $(2)$ any PSK used for
deriving {\skeyid} in IKEv1. The available documents describe both of these
as explicit prerequisites for the VPN exploitation process outlined above and
provide the reader with internal resources available to meet these
prerequisites~\cite{media-35515}.

Of course, this explanation is not dispositive and the possibility remains
that NSA could defeat VPN encryption using alternative means. A published NSA
document refers to the use of a router ``implant'' to allow decryption of
IPsec traffic indicating that the use of targeted malware is possible. This
implant ``allows passive exploitation with just ESP''~\cite{media-35515}
without the prerequisite of collecting the IKE handshake messages. This
indicates that it is an alternative mechanism to the attack described above.

The most compelling argument for a pure cryptographic attack is the
generality of the NSA's VPN exploitation process. This process appears to be
applicable across a broad swath of VPNs without regard to endpoint's identity
or the ability to compromise individual endpoints.


\subsection{Effects of a 1024-bit Break}
\label{sec:1024effects}

In this section, we use Internet-wide scanning to assess the impact of a
hypothetical DH-1024 break on IKE, SSH, and HTTPS\@. Our measurements,
performed in early 2015, indicate that these protocols would be subject to
widespread compromise by a nation-state attacker who had the resources to
invest in precomputation for a small number of 1024-bit groups.

\paragraph{IKE}
We measured how IPsec VPNs use Diffie-Hellman in practice by scanning a 1\%
random sample of the public IPv4 address space for IKEv1 and IKEv2 (the
protocols used to initiate an IPsec VPN connection) in May~2015. We used the
ZMap UDP probe module to measure support for Oakley Groups~1 and~2 (two
popular 768- and 1024-bit, built-in groups) and which group servers prefer.
Of the 80K hosts that responded with a valid IKE packet, 44.2\% were willing
to negotiate a connection using one of the two groups. We found that 31.8\%
of IKEv1 and 19.7\% of IKEv2 servers support Oakley Group~1 (768-bit) while
86.1\% and 91.0\% respectively supported Oakley Group~2 (1024-bit). In our
sample of IKEv1 servers, 2.6\% of profiled servers preferred the 768-bit
Oakley Group~1 and 66.1\% preferred the 1024-bit Oakley Group~2. For IKEv2,
5.8\% of profiled servers chose Oakley Group~1, and 63.9\% chose Oakley
Group~2.

\paragraph{SSH}
All SSH handshakes complete either a finite field or elliptic curve
Diffie-Hellman exchange. The protocol explicitly defines support for Oakley
Group~2 (1024-bit) and Oakley Group~14 (2048-bit) but also allows a
server-defined group to be negotiated. We scanned 1\% random samples of the
public IPv4 address space in April 2015. We find that 98.9\% of SSH servers
support the 1024-bit Oakley Group~2, 77.6\% support the 2048-bit Oakley
Group~14, and 68.7\% support a server-defined group\@.

During the SSH handshake, the server selects the client's highest priority
mutually supported key exchange algorithm. To estimate what servers will
prefer in practice, we performed a scan in which we mimicked the algorithms
offered by OpenSSH 6.6.1p1, the latest version of OpenSSH\@. In this scan,
21.8\% of servers preferred the 1024-bit Oakley Group~2, and 37.4\% preferred
a server-defined group. 10\% of the server-defined groups were 1024-bit, but,
of those, nearly all provided Oakley Group~2 rather than a custom group.

Combining these equivalent choices, we find that a nation-state adversary who
performed NFS precomputations for the 1024-bit Oakley Group~2 could passively
eavesdrop on connections to 3.6M (25.7\%) publicly accessible SSH servers.

\paragraph{HTTPS}
\dhe is commonly deployed on web servers. 68.3\% of Alexa Top~1M sites
support \dhe, as do 23.9\% of sites with browser-trusted certificates. Of the
Top~1M sites that support \dhe, 84\% use a 1024-bit or smaller group, with
94\% of these using one of five groups.

Despite widespread support for \dhe, a passive eavesdropper can only decrypt
connections that organically agree to use Diffie-Hellman. We estimate the
number of sites for which this will occur by offering the same sets of
ciphersuites as Chrome, Firefox, and Safari. Approximately 24.0\% of browser
connections with HTTPS-enabled Top~1M sites (and 10\% with browser-trusted
sites) will negotiate \dhe with one of the ten most popular 1024-bit primes;
17.9\% of connections with Top~1M sites could be passively eavesdropped given
the precomputation for a single 1024-bit prime.


\section{Recommendations}
\label{sec:lessons}

In this section, we present concrete recommendations to recover the expected
security of Diffie-Hellman.

\paragraph{Transition to elliptic curves.}
Transitioning to elliptic curve Diffie-Hellman (ECDH) key exchange avoids all
known feasible cryptanalytic attacks. Current elliptic curve discrete log
algorithms do not gain as much of an advantage from precomputation. In
addition, ECDH keys are shorter and computations are faster. We recommend
transitioning to elliptic curves; this is the most effective solution to the
vulnerabilities in this paper. We note that in August 2015, the NSA announced
that it was planning to transition away from elliptic curve cryptography for
its Suite B cryptographic algorithms and would replace them with algorithms
resistant to quantum computers~\cite{nsa-suiteb}. However, since no fully
vetted and standardized quantum-resistant algorithms exist currently,
elliptic curves remain the most secure choice for public key operations.

\paragraph{Increase minimum key strengths.}
To protect against the Logjam attack, server operators should disable \dheexp
and configure \dhe ciphersuites to use primes of 2048 bits or larger.
Browsers and clients should raise the minimum accepted size for
Diffie-Hellman groups to at least 1024 bits in order to avoid downgrade
attacks.

\paragraph{Don't deliberately weaken crypto.}
The Logjam attack illustrates the fragility of cryptographic ``front doors''.
Although the key sizes originally used in \dheexp were intended to be
tractable only to NSA, two decades of algorithmic and computational
improvements have significantly lowered the bar to attacks on such key sizes.
Despite the eventual relaxation of crypto export restrictions and subsequent
attempts to remove support for \dheexp{}, the technical debt induced by the
additional complexity has left implementations vulnerable for decades. Like
FREAK~\cite{freak-attack-2015}, our attacks warn of the long-term
debilitating effects of deliberately weakening cryptography.


\section{Conclusion}
\label{sec:conclusion}

We find that Diffie-Hellman key exchange, as used in practice, is often less
secure than widely believed. The problems stem from the fact that the number
field sieve for discrete log allows an attacker to perform a single
precomputation that depends only on the group, after which computing
individual logarithms in that group has a far lower cost. Although this is
well known to cryptographers, it apparently has not been widely understood by
system builders. Likewise, many cryptographers did not appreciate that a
large fraction of Internet communication depends on a few small, widely
shared groups.
%\looseness=-1

A key lesson is that cryptographers and creators of practical systems need to
work together more effectively. System builders should take responsibility
for being aware of applicable cryptanalytic attacks. Cryptographers should
involve themselves in how crypto is actually being applied, such as through
engagement with standards efforts and software review. Bridging the perilous
gap that separates these communities will be essential for keeping future
systems secure.

%\section*{Acknowledgments}
%
%The authors thank Michael Bailey, Daniel Bernstein, Ron Dreslinski,
%Tanja Lange, Adam Langley, Kenny Paterson, Andrei Popov, Ivan Ristic,
%Edward Snowden, Brian Smith, Martin Thomson, and Eric Rescorla.  This
%work was supported by the U.S. National Science Foundation, the Office
%of Naval Research, the European Research Council, and the French
%National Research Agency, with additional support from the Mozilla
%Foundation, Supermicro, Google, Cisco, the Morris Wellman
%Professorship, and the Alfred P. Sloan Foundation.  Some experiments
%used the Grid'5000 testbed, supported by INRIA, CNRS, RENATER, and
%others.


\chapter{Measuring Export-Grade Symmetric Cryptography}
\label{chapter:drown}
\documentclass[letterpaper,twocolumn,10pt]{article}
\usepackage{usenix,epsfig,endnotes}

\pagestyle{plain}

\usepackage[colorlinks=true, citecolor=black, urlcolor=black, linkcolor=black]{hyperref}
\usepackage{marginnote}
\newcommand{\todo}[1]{\textcolor{red}{\marginnote{TODO} To Do: #1}}
\newcommand{\etal}{{\em et~al.}}
\urlstyle{rm}
\usepackage{amsfonts}
\usepackage{amsmath}
\usepackage{textcomp}
\usepackage{xspace}
\usepackage{amsthm}
\usepackage{array}
\usepackage{enumitem}
\usepackage{tabularx}
\usepackage{cite}
\newcolumntype{P}[1]{>{\centering\arraybackslash}p{#1}}

\usepackage[T1]{fontenc}
\usepackage[utf8]{inputenc}

\usepackage{authblk}
\usepackage{slashbox}

\usepackage{booktabs}
\usepackage{titlesec}

\usepackage[protrusion=true,expansion=true,kerning]{microtype} % removed tracking, which seems to screw up the small caps in the references

\usepackage{balance}

% dirty spacing tricks
\titlespacing*{\paragraph}
{0pt}{4pt}{6pt}
\titlespacing*{\section}
{0pt}{6pt}{3pt}
\titlespacing*{\subsection}
{0pt}{6pt}{2pt}
\titlespacing*{\subsubsection}
{0pt}{6pt}{2pt}
\titleformat*{\subsection}{\normalfont\fontsize{11}{12.5}\bfseries} % AH: This is a common improvement to the Usenix style
\setlist[itemize]{itemsep=0.5pt, topsep=3pt}
\setlist[enumerate]{itemsep=0.5pt, topsep=3pt}

% Better URL hyphenation (AH 12/08)
\def\UrlBreaks{\do-\do\.\do\@\do\\\do\!\do\_\do\|\do\;\do\>\do\]%
\do\)\do\,\do\?\do\'\do+\do\=\do\#}
\def\UrlBigBreaks{\do\:\do\/}%

\theoremstyle{plain}
\newtheorem{proposition}{Proposition}[section]

% correct bad hyphenation here
\hyphenation{op-tical net-works semi-conduc-tor}

% include command definitions
% General definitions and terms
\newcommand{\PKCS}{PKCS\#1 v1.5\xspace}
\newcommand{\PKCSconform}{\PKCS\ conformant\xspace}
\newcommand{\sslconform}{SSLv2 conformant\xspace}
\newcommand{\tlsconform}{TLS conformant\xspace}
\newcommand{\Enc}{\mathsf{Enc}}
\newcommand{\Dec}{\mathsf{Dec}}
\newcommand{\OBleichenbacher}{\mathcal{O}_\textsf{BB}}
\newcommand{\Oracle}{\mathcal{O}}
\newcommand{\pms}{premaster secret\xspace}
\newcommand{\ssltwo}{SSLv2\xspace}
\newcommand{\sslthree}{SSLv3\xspace}
\newcommand{\hashcomputation}{hash computation\xspace} % todo rename???

% hex helpers
\newcommand{\hexspace}{\hspace{0.06cm}}
\newcommand{\hexhelp}[2]{#1\hexspace#2}
\newcommand{\hexhelpb}[4]{\hexhelp{#1}{#2}\hexspace\hexhelp{#3}{#4}}
\newcommand{\hex}[1]{{\tt 0x#1}}
\newcommand{\hexb}[2]{{\tt 0x\hexhelp{#1}{#2}}}
\newcommand{\hexc}[3]{{\tt 0x\hexhelp{#1}{#2}\hexspace#3}}
\newcommand{\hexd}[4]{{\tt 0x\hexhelp{#1}{#2}\hexspace\hexhelp{#3}{#4}}}
\newcommand{\hexh}[8]{{\tt 0x\hexhelpb{#1}{#2}{#3}{#4}\hexspace \hexhelpb{#5}{#6}{#7}{#8}}}

% feel free to rename oracles 
\newcommand{\OracleSSL}{\mathcal{O}_\textsf{SSLv2}}
\newcommand{\OracleSSLexp}{\mathcal{O}_\textsf{SSLv2-export}}
\newcommand{\OracleSSLclear}{\mathcal{O}_\textsf{SSLv2-extra-clear}}
\newcommand{\OracleSSLleaky}{\mathcal{O}_\textsf{SSLv2-export-leaky}}

\newcommand{\tOracleSSLexp}{SSLv2 export oracle\xspace}
\newcommand{\tOracleSSLclear}{extra clear oracle\xspace}
\newcommand{\tOracleSSLleaky}{Leaky export oracle\xspace}

\newcommand{\pos}[1]{{[#1]}}

% Keep extra detail for the extended version
\newif\ifext\extfalse
\newif\ifdraft\draftfalse
\newif\ifblind\blindtrue

\draftfalse
\blindfalse
%\exttrue

\usepackage{watermark}
\thiswatermark{\parbox{\textwidth}{\vskip-6pt\centering%\fontsize{10.75}{13}\normalfont
    \emph{Proceedings of the 25th USENIX Security Symposium}, August 2016\hfill
    \bfseries {\url{https://drownattack.com}\normalfont}
    \vskip3pt
    \rule[\baselineskip]{\textwidth}{.75pt}
}}

\usepackage[top=1in,left=1in,right=1in,bottom=1in]{geometry}

\makeatletter
\g@addto@macro\normalsize{%
  \addtolength\abovedisplayskip{-0.2\baselineskip}
  \addtolength\belowdisplayskip{-0.2\baselineskip}
  \addtolength\abovedisplayshortskip{-0.2\baselineskip}
  \addtolength\belowdisplayshortskip{-0.2\baselineskip}
}
\makeatother

\begin{document}
\thispagestyle{empty}

\title{DROWN: Breaking TLS using SSLv2}
\ifblind
\author{}
\else
\author[1]{Nimrod Aviram}
\author[2]{Sebastian Schinzel}
\author[3]{Juraj Somorovsky}
\author[4]{Nadia Heninger}
\author[2]{Maik Dankel}
\author[5]{Jens~Steube}
\author[4]{Luke Valenta}
\author[6]{David Adrian}
\author[6]{J. Alex Halderman}
\author[7]{Viktor Dukhovni}
\author[8]{Emilia~K\"asper}
\author[4]{Shaanan Cohney}
\author[3]{Susanne Engels}
\author[3]{Christof Paar}
\author[1]{Yuval Shavitt}
\affil[1]{Department of Electrical Engineering, Tel Aviv University}
\affil[2]{M\"unster University of Applied Sciences}
\affil[3]{Horst G\"ortz Institute for IT Security, Ruhr University Bochum}
\affil[4]{University of Pennsylvania}
\affil[5]{Hashcat Project}
\affil[6]{University of Michigan}
\affil[7]{Two Sigma/OpenSSL}
\affil[8]{Google/OpenSSL}

\renewcommand\Affilfont{\fontsize{10}{12}\selectfont}

\renewcommand\Authands{ and }
\fi

%\pagestyle{plain}

\maketitle

\newif\ifsubmit\submittrue % Cuts for usenix submission

\newcommand{\twolinecell}[2][r]{%
\begin{tabular}[#1]{@{}c@{}}#2\end{tabular}}

% \ns (``number space''): Insert space equivalent to a number, to avoid needing leading zeros to align numbers in tables; first argument specifies a multiplier (AH 4/2016)
%     Example: 100%\\ \ns 50% \\ \ns[2] 1%
\newlength{\nswidth}\newcommand{\ns}[1][1]{\settowidth{\nswidth}{0}\hspace{#1\nswidth}}

\newcommand{\tabDrownAll}{
  \begin{table*}[t]
  \centering\small
  \begin{tabularx}{\textwidth}{Xrrrrrrr} 
  \toprule
  & & \multicolumn{3}{c}{\it Any certificate} & \multicolumn{3}{c}{\it Trusted certificates} \\
  \cmidrule(lr){3-5} \cmidrule(lr){6-8}
  \textbf{Protocol} & \textbf{Port} & \textbf{SSL/TLS}  & \twolinecell{\bf \ssltwo \\\bf support}        & \twolinecell{\bf Vulnerable\\\bf key} & \textbf{SSL/TLS} & \twolinecell{\bf \ssltwo\\\bf support} &  \twolinecell{\bf Vulnerable \\\bf key} \\
  \midrule
  SMTP     & 25  & 3,357\,K &   936\,K (28\%)  & 1,666\,K (50\%)  &  1,083\,K  &   190\,K (18\%) & 686\,K (63\%)   \\
  POP3     & 110 & 4,193\,K &   404\,K (10\%)  & 1,764\,K (42\%)  &  1,787\,K  &   230\,K (13\%) & 1,031\,K (58\%) \\
  IMAP     & 143 & 4,202\,K &   473\,K (11\%)  & 1,759\,K (42\%)  &  1,781\,K  &   223\,K (13\%) & 1,022\,K (57\%) \\
  HTTPS    & 443 & 34,727\,K& 5,975\,K (17\%)  & 11,444\,K (33\%) &  17,490\,K & 1,749\,K (10\%) & 3,931\,K (22\%) \\
  SMTPS    & 465 & 3,596\,K &   291\,K \ns (8\%)   & 1,439\,K (40\%)  &  1,641\,K  &     40\,K \ns (2\%) & 949\,K (58\%)   \\
  SMTP     & 587 & 3,507\,K &   423\,K (12\%)  & 1,464\,K (42\%)  &  1,657\,K  &    133\,K \ns (8\%) & 986\,K (59\%)   \\
  IMAPS    & 993 & 4,315\,K &   853\,K (20\%)  & 1,835\,K (43\%)  &  1,909\,K  &   260\,K (14\%) & 1,119\,K (59\%) \\
  POP3S    & 995 & 4,322\,K &   884\,K (20\%)  & 1,919\,K (44\%)  &  1,974\,K  &   304\,K (15\%) & 1,191\,K (60\%) \\
  \midrule
  (Alexa~Top~1M) & 443 & 611\,K   & 82\,K (13\%)     & 152\,K (25\%)  & 456\,K     & 38\,K \ns (8\%)     & 109\,K (24\%)   \\
%% \ifext
%% \else
%%   \midrule
%%   \bf Special DROWN:\smallskip\\
%%   HTTPS & 443 & 34,727\,K  & 4,029\,K (12\%) & 9,089\,K (26\%) & 17,490\,K & 2,523\,K (14\%) & 3,793\,K (22\%) \\
%%     (Alexa~1M) & 443 & 611\,K & 22\,K (4\%)  & 52\,K (9\%)     & 456\,K & 33\,K (7\%)     & 85\,K (19\%)    \\
%% \fi
  \bottomrule
   \end{tabularx}
   \caption{\textbf{Hosts vulnerable to general DROWN}\,---\,%
   We performed Internet-wide scans to measure the number of hosts supporting \ssltwo on several different protocols.  A host is vulnerable to DROWN if its public key is exposed anywhere via \ssltwo.  Overall vulnerability to DROWN is much larger than support for \ssltwo due to widespread reuse of keys.
   }
   \label{table:general}
  \end{table*}
}

\newcommand{\tabSpecialAll}{
\begin{table*}[t]
  \centering\small
  \begin{tabularx}{\textwidth}{Xrrrrrrr}
    \toprule
    & & \multicolumn{3}{c}{\it Any certificate} & \multicolumn{3}{c}{\it Trusted certificates} \\
    \cmidrule(lr){3-5} \cmidrule(lr){6-8}
    \textbf{Protocol} & \textbf{Port} & \multicolumn{1}{c}{\bf SSL/TLS} & \twolinecell{\bf Special DROWN\\\bf oracles} & \twolinecell{\bf Vulnerable\\\bf key} & \multicolumn{1}{c}{\bf SSL/TLS} & \twolinecell{\bf Vulnerable\\\bf key} & \twolinecell{\bf Vulnerable\\\bf name} \\
    \midrule
    SMTP  & 25  & 3,357\,K   & 855\,K (25\%)   & 896\,K (27\%)   & 1,083\,K  & 305\,K (28\%)   & 398\,K (37\%)   \\
    POP3  & 110 & 4,193\,K   & 397\,K \ns (9\%)    & 946\,K (23\%)   & 1,787\,K  & 485\,K (27\%)   & 674\,K (38\%)   \\
    IMAP  & 143 & 4,202\,K   & 457\,K (11\%)   & 969\,K (23\%)   & 1,781\,K  & 498\,K (30\%)   & 690\,K (39\%)   \\
    HTTPS & 443 & 34,727\,K  & 4,029\,K (12\%) & 9,089\,K (26\%) & 17,490\,K & 2,523\,K (14\%) & 3,793\,K (22\%) \\
    SMTPS & 465 & 3,596\,K   & 334\,K \ns (9\%)    & 765\,K (21\%)   & 1,641\,K  & 430\,K (26\%)   & 630\,K (38\%)   \\
    SMTP  & 587 & 3,507\,K   & 345\,K (10\%)   & 792\,K (23\%)   & 1,657\,K  & 482\,K (29\%)   & 667\,K (40\%)   \\
    IMAPS & 993 & 4,315\,K   & 892\,K (21\%)   & 1,073\,K (25\%) & 1,909\,K  & 602\,K (32\%)   & 792\,K (42\%)   \\
    POP3S & 995 & 4,322\,K   & 897\,K (21\%)   & 1,108\,K (26\%) & 1,974\,K  & 641\,K (32\%)   & 835\,K (42\%)   \\
    \midrule
    (Alexa~Top~1M) & 443 & 611\,K & 22\,K \ns (4\%)  & 52\,K \ns (9\%)     & 456\,K    & 33\,K \ns (7\%)     & 85\,K (19\%)    \\
    \bottomrule
  \end{tabularx}
   \caption{\textbf{Hosts vulnerable to special DROWN}\,---\,%
   A server is vulnerable to special DROWN if its key is exposed by a host with the CVE-2016-0703 bug. Since the attack is fast enough to enable man-in-the-middle attacks, a server is also vulnerable (to impersonation) if any name in its certificate is found in any trusted certificate with an exposed key.}
   \label{table:special}
\end{table*}
}



\section*{Abstract}
\abstract{
Abstract data types are good.    
}

%\footnote{Submissions are due on Thursday, February 18, 2016, 9:00 pm EST. Paper submissions should be at most 13 typeset pages, excluding bibliography and well-marked appendices.}

\section{Introduction}
% !TEX root ../proposal.tex

Large-scale empirisicm, enabled by Internet-wide scanning, provides missing
insight into the security of cryptography used on the Internet. 

% What insight is "missing"? What are the other methods?

We gain additional insight into cryptography from large-scale empiricism.

The Internet is a large, distributed system with a diverse set of clients and
servers~\cite{something}. Compatibility with legacy clients remains a top priority 
for many operators~\cite{something} and protocol designers~\cite{something}.

% High-level questions that empiricism helps with. (TLS 1.3 compatibility woes?)
%
% List of insights that help with this.

We can use that insight to improve the overall security of the Internet, today.

We can use that insight to better inform the design of the Internet in the future.

The latest version of TLS 1.3~\cite{rfc8446} was drafted over the course of \TK years, with countermeasures
in place for both known cryptographic vulnerabilites in earlier version of TLS, as well as with updated negotation and parameter selection processes, designed to prevent deployment failures that had traditionally been considered outside of the scope of standards.

We can use it alongside analysis of individual components. (With our power combined!)

Using large-scale empiriscism requires solving engineering challenges.

\section{Large-scale Empiricism}

% Some people study cryptograpy usages in individual programs~\cite{most-dangerous-code-2012}.
% 
% API design helps.
% 
% But ultimately, what is happening? Are we impacting the security on the web as a whole?
% 
% But what about ecosystems?
% 
% Study usage across ecosystems.
% 
% Usage is configuration.
% 
% Design experiments to glean information into configuration.
% 
% Talk about configuration of the ecosystem beyond just configuration of hosts (cross-protocol correlation).

Security research often involves searching for new classes of
vulnerabilites~\cite{something}, or identifying vulnerabilites within
existing systems~\cite{something}. Other research has concerned identifying
misuse of security-critical APIs~\cite{gutmann-lessons}, including systematic
misuse among large swaths of library code
users~\cite{most-dangerous-code-2012}. While improving API design and
secure-by-default programming models certainly improve the security behavior
surrounding network communication for many programs, it does not provide
insight into the security behavior of Internet ecosystems themselves.

Large-scale empiricism enables us to understand the security behavior
ecosystems, rather than \TK. To understand the behavior of users of cryptographic applications, 

\subsection{Cryptography in the TLS Ecosystem}

Configuration of web servers.

Logjam.

Diffie-Hellman. Subgroup.

RFC 5114

\subsection{Unexpected Interactions in Cryptography at Scale}

Beyond examining individual protocols at the global perspective, new issues can
be identified by looking across multiple similar protocols. Although it is
tempting to consider the security of any given system in isolation, complex
interactions between systems necessarily impacts security.

At the most-basic level, cross-protocol key-reuse links the security of any two
protocols together in key-compromise scenarios. When the same server private
key is used for both a mail server and a web server, compromising the often
less-secured mail server effectively compromises the web
server~\cite{mail-2015}.

Looking past simply key reuse, name reuse across protocols utilizing TLS opens
additional attack vectors. Any service with a certificate chaining to a
publicly trusted root that shares a Common Name or Subject Alternative Name
with a web server, or that covers the web address under a wildcard name, can be
used to impersonate the web server if the key is compromised.

Furthermore, specific vulnerabilites in the TLS protocol and implementations
can be utilized in a cross-protocol context to attack users of a web service
using an unpatched ``forgotten-about'' server, such as a mail server, without
even explicitly compromising the private key of the service. The best example
of this is the DROWN vulnerability~\cite{drown-2016}, in which the mere
existence of an SSLv2 host that shared a key with a TLS host enabled decryption
of otherwise secure TLS connections using modern cryptography.

\paragraph{Cross-protocol interactions between TLS and SSLv2}

SSLv2

DROWN

Generalize DROWN to cross-protocol key reusage

\paragraph{Export Cryptography}
Don't weaken cryptography.

Takeaways from Black Hat talk.

\section{Engineering Challenges}

To do science, we have to collect data.

To collect the data, we have to build things.

Determining what to build and how to build is also difficult.

Engineering is very closely related to methodology, which is where the science is.

\subsection{Speed}

Assume we're using scanning.

TLS-Attacker is slow.

Performance matters, especially once you start cross-correlating.

Discovery and collection separate. (ZMap vs ZGrab)

\subsection{Data Processing}

Beyond collecting data, you actually have to process it.

Straightforward for one-off studies.

What if you want longitudinal data?

Take-away This all requires work.


\section{Background}
% !TEX root = subgroup.tex

\section{Background}

\subsection{Groups, orders, and generators}
\label{sec:group-background}

The two types of groups used for Diffie-Hellman key exchange in practice are
multiplicative groups over finite fields (``mod $p$'') and elliptic curve
groups. We focus on the ``mod $p$'' case, so a group is typically specified by
a prime $p$ and a generator $g$, which generates a multiplicative subgroup
modulo $p$.  Optionally, the group order $q$ can be specified; this is the
smallest positive integer $q$ satisfying $g^q \equiv 1 \bmod p$.  Equivalently,
it is the number of distinct elements of the subgroup $\{g, g^2, g^3, \dots
\bmod p\}$.

By Lagrange's theorem, the order $q$ of the subgroup generated by $g$ modulo
$p$ must be a divisor of $p-1$. Since $p$ is prime, $p-1$ will be even, and
there will always be a subgroup of order 2 generated by the element $-1$. For
the other factors $q_i$ of $p-1$, there are subgroups of order $q_i \bmod p$.
One can find a generator $g_i$ of a subgroup of order $q_i$ using a randomized
algorithm: try random integers $h$ until $h^{(p-1)/q_i} \ne 1 \bmod p$; $g_i =
h^{(p-1)/q_i} \bmod p$ is a generator of the subgroup.  A random $h$ will
satisfy this property with probability $1 - 1/q_i$.

In theory, neither $p$ nor $q$ is required to be prime. Diffie-Hellman key
exchange is possible with a composite modulus and with a composite group order.
In such cases, the order of the full multiplicative group modulo $p$ is
$\phi(p)$ where $\phi$ is Euler's totient function, and the order of the
subgroup generated by $g$ must divide $\phi(p)$. Outside of implementation
mistakes, Diffie-Hellman in practice is done modulo prime $p$.
%\looseness=-1

\subsection{Diffie-Hellman Key Exchange}

Diffie-Hellman key exchange allows two parties to agree on a shared secret in
the presence of an eavesdropper~\cite{diffie1976new}. Alice and Bob begin by
agreeing on shared parameters (prime $p$, generator $g$, and optionally group
order $q$) for an algebraic group.  Depending on the protocol, the group may be
requested by the initiator (as in IKE), unilaterally chosen by the responder
(as in TLS), or fixed by the protocol itself (SSH originally built in support
for a single group).

Having agreed on a group, Alice chooses a secret $x_a < q$ and sends Bob $y_a =
g^{x_a}\bmod p$.  Likewise, Bob chooses a secret $x_b < q$ and sends Alice $y_b =
g^{x_b}\bmod p$. Each participant then computes the shared secret key
$g^{x_a x_b}\bmod p$.%  \looseness=-1 

Depending on the implementation, the public values $y_a$ and $y_b$ might be
\emph{ephemeral}---freshly generated for each connection---or \emph{static} and
reused for many connections.

\subsection{Discrete log algorithms}

The best known attack against Diffie-Hellman is for the eavesdropper to compute
the the private exponent $x$ by calculating the discrete log of one of Alice or
Bob's public value $y$. With knowledge of the exponent, the attacker can
trivially compute the shared secret.  It is not known in general whether the
hardness of computing the shared secret from the public values is equivalent to
the hardness of discrete log.

The \emph{computational Diffie-Hellman assumption} states that computing the
shared secret $g^{x_ax_b}$ from $g^{x_a}$ and $g^{x_b}$ is hard for some choice
of groups.  A stronger assumption, the \emph{decisional Diffie-Hellman
problem}, states that given $g^{x_a}$ and $g^{x_b}$, the shared secret
$g^{x_ax_b}$ is computationally indistinguishable from random for some groups.
This assumption is often not true for groups used in practice; even with safe
primes as defined below, many implementations use a generator that generates
the full group of order $p-1$, rather than the subgroup of order $(p-1)/2$.  This means
that a passive attacker can always learn the value of the secret exponent modulo 2.
To avoid leaking this bit of information about the exponent, both sides could agree to 
compute the shared secret as $y^{2x} \bmod p$.  We have not seen implementations
with this behavior.
%\looseness=-1

There are several families of discrete log algorithms, each of which apply to
special types of groups and parameter choices. Implementations must take care
to avoid choices vulnerable to any particular algorithm. These include:

\paragraph{Small-order groups}
The Pollard rho~\cite{pollard1975monte} and Shanks' baby step-giant step
algorithms~\cite{shanks1971class} each can be used to compute discrete logs in
groups of order $q$ in time $O(\sqrt{q})$.  To avoid being vulnerable,
implementations must choose a group order with bit length at least twice the
desired bit security of the key exchange. In practice, this means that group
orders $q$ should be at least 160 bits for an 80-bit security level.

\paragraph{Composite-order groups}
If the group order $q$ is a composite with prime factorization $q = \prod_i
q_i^{e_i}$, then the attacker can use the Pohlig-Hellman
algorithm~\cite{pohlig1978improved} to compute a discrete log in time $O(\sum_i
e_i \sqrt{q_i})$.  The Pohlig-Hellman algorithm computes the discrete log in
each subgroup of order $q_i^{e_i}$ and then uses the Chinese remainder theorem
to reconstruct the log modulo $q$.  Adrian et al.~\cite{weakdh-ccs15} found
several thousand TLS hosts using primes with composite-order groups, and were
able to compute discrete logs for several hundred Diffie-Hellman key exchanges
using this algorithm.  To avoid being vulnerable, implementations should choose
$g$ so that it generates a subgroup of large prime order modulo $p$.
%\looseness=-1

\paragraph{Short exponents}
If the secret exponent $x_a$ is relatively small or lies within a known range of
values of a relatively small size, $m$, then the Pollard lambda ``kangaroo''
algorithm~\cite{Pollard2000} can be used to find $x_a$ in time $O(\sqrt{m})$.  To
avoid this attack, implementations should choose secret exponents to have bit
length at least twice the desired security level.  For example, using a 256-bit
exponent for for a 128-bit security level.

\paragraph{Small prime moduli} When the subgroup order is not small or
composite, and the prime modulus $p$ is relatively large, the fastest known
algorithm is the number field sieve~\cite{gordon1993discrete}, which runs in
subexponential time in the bit length of $p$, $\exp\left((1.923+o(1))(\log
p)^{1/3} (\log \log p)^{2/3}\right)$. Adrian et al.~recently applied the number field sieve
to attack 512-bit primes in about 90,000 core-hours~\cite{weakdh-ccs15}, and
they argue that attacking 1024-bit primes---which are widely used in
practice---is within the resources of large governments. To avoid this attack,
current recommendations call for $p$ to be at least 2048
bits~\cite{nist-sp-800-57-rev4}. When selecting parameters, implementers should
ensure all attacks take at least as long as the number field sieve for their parameter set.

\subsection{Diffie-Hellman group characteristics}

\paragraph{``Safe'' primes}
In order to maximize the size of the subgroup used for Diffie-Hellman, one can
choose a $p$ such that $p = 2q + 1$ for some prime $q$. Such a $p$ is called a
``safe'' prime, and $q$ is a Sophie Germain prime.  For sufficiently large safe
primes, the best attack will be solving the discrete log using the number field
sieve.
Many standards explicitly specify the use of safe primes for Diffie-Hellman in
practice.  The Oakley protocol~\cite{rfc2412} specified five ``well-known''
groups for Diffie-Hellman in 1998. These included three safe primes of size
768, 1024, and 1536 bits, and was later expanded to include six more groups in
2003~\cite{rfc3526}. The Oakley groups have been built into numerous other
standards, including IKE~\cite{rfc2409} and SSH~\cite{rfc4253}.

%Given a finite field, with a cyclic group modulo some prime, $\mathbb{Z}/p$, and group operations $+$ and $\cdot$. 
%
%Both participants choose an element, $a$ and $b$ respectively, randomly from the integers up to the order of the subgroup. The public keys take the form of $g^a$ and $g^b$ where $g$ is a generator of the subgroup and exponentiation is repeated application of the multiplication operation within the group. Both parties then exponentiate their partners public key by their secret to yield values of $g^{ba}$ and $g^{ab}$ respectively. By the abelian nature of the group these are the same group element, thus yielding a shared secret. 
%
%For the case where the group is sufficiently large, not of smooth order, and $a$ and $b$ are of sufficient length, the Diffie-Hellman assumption posits that that the most efficient way to find the value of $g^{ab}$ given only $g^{a}$ and $g^{b}$ is equivalent to finding a solution to an instance of the discrete log problem, which is assumed hard. However, the above relies on three assumptions that can all be targeted in real implementations of the protocol.
%
%The solution to the DLP can be found if the group has a small number of elements or the exponent is small enough such that there is no 'wrap-around' when the generator is exponentiated, reducing the problem to solving the logarithm in the integers.
%
%A group can be chosen such that it contains a subgroup in which all the prime factors of the generator are smaller than some value $B$, which for integers is called a '$B-smooth$' integer. Then an attacker can solve the discrete log problem in the orders of the factors, and using the Chinese remainder theorem can reconstruct the unique solution in the original group.
%

\paragraph{DSA groups}
The DSA signature algorithm~\cite{dsa} is also based on the hardness of
discrete log.  DSA parameters have a subgroup order $q$ of much smaller size
than $p$.  In this case $p-1 = q r$ where $q$ is prime and $r$ is a large
composite, and $g$ generates a group of order $q$.  FIPS 186-4~\cite{dsa}
specifies 160-bit $q$ for 1024-bit $p$ and 224- or 256-bit $q$ for 2048-bit
$p$.  The small size of the subgroup allows the signature to be much shorter
than the size of $p$.

\subsection{DSA Group Standardization}

DSA-style parameters have also been recommended for use for Diffie-Hellman key
exchange.  NIST Special Publication 800-56A, ``Recommendation for Pair-Wise Key
Establishment Schemes Using Discrete Logarithm
Cryptography''~\cite{barker2007sp}, first published in 2007, specifies that
finite field Diffie-Hellman should be done over a prime-order subgroup $q$ of
size 160 bits for a 1024-bit prime $p$, and a 224- or 256-bit subgroup for a
2048-bit prime.  While the order of the multiplicative subgroups is in line
with the hardness of computing discrete logs in these subgroups, no explanation
is given for recommending a subgroup of precisely this size rather than setting
a minimum subgroup size or using a safe prime.  Using a shorter exponent will
make modular exponentiation more efficient, but the order of the subgroup $q$
does not increase efficiency---on the contrary, the additional modular exponentiation
required to validate that a received key exchange message is contained in the correct
subgroup will render key exchange with DSA primes less efficient than using a ``safe'' 
prime for the same exponent length.  Choosing a small subgroup order is not known to have
much impact on other cryptanalytic attacks, although the number field sieve is
somewhat (not asymptotically) easier as the linear algebra step is performed
modulo the subgroup order $q$.~\cite{weakdh-ccs15}

RFC 5114, ``Additional Diffie-Hellman Groups for Use with IETF
Standards''~\cite{rfc5114}, specifies three DSA groups with the above orders
``for use in IKE, TLS, SSH, etc.''  These groups were taken from test data
published by NIST~\cite{nistffcsamples}. They have been widely implemented in
IPsec and TLS, as we will show below. We refer to these groups as Group 22
(1024-bit group with 160-bit subgroup), Group 23 (2048-bit group with 224-bit
subgroup), and Group 24 (2048-bit group with 256-bit subgroup) throughout the
remainder of the paper to be consistent with the group numbers assigned for
IKE.

RFC 6989, ``Additional Diffie-Hellman Tests for the Internet Key Exchange
Protocol Version 2 (IKEv2)''~\cite{rfc6989}, notes that ``mod $p$'' groups with
small subgroups can be vulnerable to small subgroup attacks, and mandates that
IKE implementations should validate that the received value is in the correct
subgroup or never repeat exponents.

\subsection{Small subgroup attacks}
\label{subsec:small-subgroup-attack}

Since the security of Diffie-Hellman relies crucially on the group parameters,
implementations can be vulnerable to an attacker who provides maliciously
generated parameters that change the properties of the group. 
With the right parameters and implementation decisions, an attaker may be able
to efficiently determine the Diffie-Hellman shared secret. In some cases, a
passive attacker may be able to break a transcript offline.

\paragraph{Small subgroup confinement attacks}
In a small subgroup confinement attack, an attacker (either a man-in-the-middle
or a malicious client or server) provides a key-exchange value $y$ that lies in a
subgroup of small order.  This forces the other party's view of the shared
secret, $y^x$, to lie in the subgroup generated by the attacker.  This
type of attack was described by van Oorschot and Wiener~\cite{van1996diffie}
and ascribed to Vanstone and Anderson and Vaudenay~\cite{Anderson1996}.  Small
subgroup confinement attacks are possible even when the server does not repeat
exponents---the only requirement is that an implementation does not validate
that received Diffie-Hellman key exchange values are in the correct subgroup.

%Consider a simplified Diffie-Hellman key exchange protocol where Alice and Bob negotiate a shared secret $k = g^{ab} \bmod p$ using Diffie-Hellman key exchange, and then Alice symmetrically encrypts a message using the secret $k$ and transmits it to Bob.  Alice might specify a safe prime $p$ and a generator $g$ of order $q = (p-1)/2$, and use a static Diffie-Hellman public value $y_a = g^a \bmod p$.  

When working $\bmod\,p$, there is always a subgroup of order 2, since $p-1$ is
even. A malicious client Mallory could initiate a Diffie-Hellman key exchange
value with Alice and send her the value $y_M = p-1 \equiv -1 \bmod p$, which is
is a generator of the group of order $2 \bmod p$.  When Alice attempts to
compute her view of the shared secret as $k_a = y_M^a \bmod p$, there are only
two possible values, $1$ and $-1 \bmod p$.

The same type of attack works if $p-1$ has other small factors $q_i$.  Mallory
can send a generator $g_i$ of a group of order $q_i$ as her Diffie-Hellman key
exchange value. Alice's view of the shared secret will be an element of the
subgroup of order $q_i$. Mallory then has a $1/q_i$ chance of blindly guessing Alice's
shared secret in this invalid group. Given a message from Alice encrypted using
Alice's view of the shared secret, Mallory can brute force Alice's shared secret in $q_i$ guesses.

 More recently, Bhargavan and
Delignat-Lavaud~\cite{conf/ndss/BhargavanDP15} describe ``key synchronization''
attacks against IKEv2 where a man-in-the-middle connects to both the initiator and
responder in different connections, uses a small subgroup confinement attack
against both, and observes that there is a $1/q_i$ probability of the shared
secrets being the same in both connections.  Bhargavan and Leurent~\cite{sloth}
describe several attacks that use subgroup confinement attacks to obtain a
transcript collision and break protocol authentication.

To protect against subgroup confinement attacks, implementations should use
prime-order subgroups with known subgroup order. Both parties must validate
that the key exchange values they receive are in the proper subgroup. That is,
for a known subgroup order $q$, a received Diffie-Hellman key exchange value
$y$ should satisfy $y^q \equiv 1 \bmod p$.  For a safe prime, it suffices to
check that $y$ is strictly between $1$ and $p-1$. 

\paragraph{Small subgroup key recovery attacks}
Lim and Lee~\cite{Lim1997} discovered a further attack that arises when an
implementation fails to validate subgroup order and resues a static
secret exponent for multiple key exchanges. A malicious party may be able
to perform multiple subgroup confinement attacks for different prime factors
$q_i$ of $p-1$ and then use the Chinese remainder theorem to reconstruct the
static secret exponent.

The attack works as follows.  Let $p-1$ have many small factors $p-1 = q_1 q_2
\dots q_n$.  Mallory, a malicious client, uses the procedure described in
Section~\ref{sec:group-background} to find a generator of the subgroup $g_i$ of
order $q_i \bmod p$.  Then Mallory transmits $g_i$ as her Diffie-Hellman key
exchange value, and receives a message encrypted with Alice's view of the
shared secret $g_i^{x_a}$, which Mallory can brute force to learn the value of
$x_a \bmod q_i$.  Once Mallory has repeated this process several times, she can
use the Chinese remainder theorem to reconstruct $x_a \bmod \prod_i q_i$.  The
running time of this attack is $\sum_i q_i$, assuming that Mallory performs an
offline brute-force search for each subgroup.

A randomly chosen prime $p$ is likely to have subgroups of large enough order
that this attack is infeasible to carry out for all subgroups.  However, if in
addition Alice's secret exponent $x_a$ is small, then Mallory only needs to carry out this
attack for a subset of subgroups of orders $q_1, \dots, q_k$ satisfying
$\prod_{i=0}^k q_i > x_a$, since the Chinese remainder theorem ensures that $x_a$
will be uniquely defined.
Mallory can also improve on the running time of the attack by taking advantage
of the Pollard lambda algorithm.  That is, she could use a small subgroup
attack to learn the value of $x_a \bmod \prod_{i=1}^k q_i$ for a subset of
subgroups $\prod_{i=1}^k q_i < x_a$, and then use the Pollard lambda algorithm to
reconstruct the full value of $a$, as it has now been confined to a smaller
interval.

In summary, an implementation is vulnerable to small subgroup key recovery
attacks if it does not verify that received Diffie-Hellman key exchange values
are in the correct subgroup; uses a prime $p$ such that $p-1$ has small
factors; and reuses Diffie-Hellman secret exponent values.  The attack is made
even more practical if the implementation uses small exponents.

A related attack exists for elliptic curve groups: an invalid curve attack.
Similarly to the case we describe above, the attacker generates a series of
elliptic curve points of small order and sends these
points as key exchange messages to the victim.  If the victim does not validate that the received
point is on the intended curve, they return a response that reveals information
about the secret key modulo different group orders.  After enough queries, the
attacker can learn the victim's entire secret.  Jager, Schwenk, and
Somorovsky~\cite{Jager2015} examined eight elliptic curve implementations and
discovered two that failed to validate the received curve point. For elliptic
curve groups, this attack can be much more devastating because the attacker has
much more freedom in generating different curves, and can thus find many
different small prime order subgroups.  For the finite field Diffie-Hellman
attack, the attacker is limited only to those subgroups whose orders are factors
of $p-1$.



\section{Breaking TLS with SSLv2}
In this section, we describe our cross-protocol DROWN attack that uses an \ssltwo server as an oracle to efficiently decrypt TLS connections.  We first describe our techniques using a generic \ssltwo oracle.  In Section~\ref{vulnerability}, we show how a protocol flaw in \ssltwo can be used to construct such an oracle, and describe our general DROWN attack.  In Section~\ref{sec:special}, we show how an implementation flaw in common versions of OpenSSL leads to a very powerful oracle, and describe our efficient special DROWN attack.

%\subsection{An efficient Bleichenbacher attack}

\subsection{Attack scenario}
\label{sec:attack-scenario}

We consider a server that accepts TLS connections from clients. The connections are established using a secure, state-of-the-art TLS version (1.0--1.2) and a \texttt{TLS\_RSA} cipher suite where the private key is not known to the attacker.

\paragraph{Server RSA key exposed via \ssltwo.}
The same RSA public key as the TLS connections is also used for \ssltwo. 
\if0
This can arise in a number of situations that are common in practice.  The simplest case is a server (mis)configured to also support \ssltwo on the port accepting the TLS connections.  But the \ssltwo support may be on an entirely different port, or an entirely different server.
This situation might arise when an organization uses a wildcard certificate for both its website on port 443 and its SMTP server on port 25.  Both of these situations are common in practice; see Section~\ref{sec:scans} for more detail.  
\fi
For simplicity, our presentation will refer to the servers accepting TLS and \ssltwo connections as the same entity.

\paragraph{The attacker's position in the network.}
Our attacker is able to passively eavesdrop on traffic between the client and server and record RSA-based TLS traffic, but does not perform any active man-in-the-middle interference.
\ifext
The goal of our attacker is to decrypt a \texttt{TLS\_RSA} connection established between the server and one of the clients that connects to it.
Our attacker could be as sophisticated as a nation state with eavesdropping capabilities at Internet exchange points, or as simple as a neighbor on an open WiFi network.
\fi

The attacker can expect to decrypt one out of 1,000 intercepted TLS connections in our attack for typical parameters.  This is a devastating threat in many scenarios.  For example, a decrypted TLS connection might reveal a client's HTTP cookie or plaintext password, and an attacker would only need to successfully decrypt a single ciphertext to compromise the client's account.

In order to collect 1,000 TLS connections, the attacker might simply wait patiently until sufficiently many connections are recorded.  If the attacker's intended victim is the \emph{server}, rather than a specific client, observing this many connections from many clients might take only a short time for an attacker who is located at a company firewall or who could perform a DNS spoofing or BGP hijacking attack to redirect traffic transparently through themselves.  If the attacker's intended victim is a \emph{particular client}, this is still feasible in many cases.  As an example, the Mozilla Thunderbird email client will check for new email messages every ten minutes by default.  A targeted user will make 1,000 connections after leaving the application running for a week.  A less patient attacker could embed or inject malicious JavaScript on an otherwise innocuous web site to cause the client to connect repeatedly to the victim server in a short time frame, as in the BEAST attack~\cite{BEAST}.  Normally such connections would use TLS session resumption instead of completing a fresh handshake on each time, but if an attacker can trigger an error, the next connection will be negotiated with a fresh handshake.


\subsection{A generic \ssltwo oracle}

Our attacks make use of a padding oracle that can be queried on a ciphertext and leaks information about decrypted plaintext; this abstractly models the information gained from an \ssltwo server's behavior.  Our \ssltwo oracles reveal many bytes of plaintext, resulting in an efficient attack.

Our cryptographic oracle $\Oracle$ has the following functionality: 
$\Oracle$ decrypts an RSA ciphertext $c$ and responds with ciphertext validity based on the structure of the decrypted message $m$.  
The ciphertext is valid only if $m$ starts with \hexb{00}{02} followed by non-null padding bytes, a delimiter byte \hex{00}, and a \texttt{master\_key} $mk_{secret}$ of correct byte length $k$.
In the following, we denote such a ciphertext to be \textit{\sslconform}.

All of the \ssltwo padding oracles we instantiate give the attacker similar information about a \PKCSconform \ssltwo ciphertext:
\begin{equation*} 
\Oracle(c) =  
\begin{cases} 
mk_{secret} & \text{ if } c^d \bmod N = 00 || 02 || PS || 00 || mk_{secret}  \\ 
0 & \text{ otherwise.} 
\end{cases} 
\end{equation*}
That is, the oracle $\Oracle(c)$ will return the decrypted message $mk_{secret}$ if it is queried on a \PKCSconform \ssltwo ciphertext $c$ corresponding to a correctly \PKCS padded encryption of $mk_{secret}$.  The attacker then learns $k + 3$ bytes of information about $m = c^d \bmod N$: the first two bytes are $00 || 02$, and the last $k+1$ bytes are $00 || mk_{secret}$.  The length $k$ of $mk_{secret}$ varies based on the cipher suite used in the instantiation of the oracle.  For export-grade cipher suites such as \texttt{SSL\_RSA\_EXPORT\_WITH\_RC2\_CBC\_40\_MD5},
%or \texttt{SSL\_RSA\_EXPORT\_WITH\_RC4\_40\_MD5}\@,
$k$ will be 5 bytes, so the attacker learns 8 bytes of information about $m$. For 
%\texttt{SSL\_RSA\_WITH\_DES\_CBC\_SHA}\@, $k$ is 8 bytes; for 
\texttt{SSL\_DES\_192\_EDE3\_CBC\_WITH\_MD5}\@, $k$ is 24 bytes and the attacker learns 27 bytes of plaintext.

\subsection{DROWN attack template}
\label{sec:adapted-bb-compact}
Our attacker will use an \ssltwo oracle $\Oracle$ to decrypt a TLS \texttt{ClientKeyExchange}.  
The behavior of $\Oracle$ poses two problems for the attacker. First, a TLS ciphertext transmitted in a TLS key exchange decrypts to a 48-byte \pms. But since no \ssltwo cipher suites have 48-byte key strengths, this means that a valid TLS ciphertext is invalid to our oracle $\Oracle$. 
In order to apply Bleichenbacher's attack, the attacker needs to transform the TLS ciphertext into a valid \ssltwo key exchange message. Second, $\Oracle$ is very restrictive, since it strictly checks the length of the unpadded message. 
According to Bardou et al.~\cite{bardou2012efficient}, using such an oracle for Bleichenbacher's attack would require 12 million oracle queries.\footnote{See Table~1 in~\cite{bardou2012efficient}. The oracle is denoted with the term \texttt{FFF}.} 
%As each query would require an exhaustive search over $2^{40}$ values, this would make the attack significantly more costly.

Our attacker overcomes these problems by following this generic attack flow:
\begin{enumerate}
 \setcounter{enumi}{-1}
	\item The attacker collects many encrypted TLS RSA key exchange messages.
	\item He then attempts to convert the intercepted TLS ciphertexts containing a 48-byte \pms to valid RSA \PKCS encoded ciphertexts containing messages of length appropriate to the \ssltwo oracle $\Oracle$. We accomplish this by taking advantage of RSA ciphertext malleability and a technique of Bardou et al.~\cite{bardou2012efficient}.
	\item Once the attacker has obtained a valid \ssltwo RSA ciphertext, he can continue with a modified version of Bleichenbacher's attack, and decrypt the message after many more oracle queries.
	\item The attacker can then transform the decrypted plaintext back into the original plaintext, which is one of the collected TLS handshakes.
\end{enumerate}

We describe the algorithmic improvements we use to make each of these steps efficient below.

\subsubsection{Finding an \sslconform ciphertext}
\label{sec:trimmers}
The first step for the attacker is to transform the original TLS \texttt{ClientKeyExchange} message $c_0$ from a \tlsconform ciphertext into an \sslconform ciphertext. 
A trivial approach would be to generate multipliers $s_i \in \{s_1,s_2,\ldots\}$, and compute ciphertexts $c_i = (c_0 {s_i}^e) \bmod N$, until one gets accepted by $\Oracle$.
However, the number of generated ciphertexts would be high, because $\Oracle$ is very restrictive; for 2048-bit RSA keys and an oracle returning a 5-byte $k$ the probability that a random ciphertext becomes \sslconform is $P_{rnd} \approx (1/256)^3 * (255/256)^{249} \approx 2^{-25}$.

Instead, we rely on the concept of \emph{trimmers}, which were introduced by Bardou et al.~\cite{bardou2012efficient}. 
Assume that the message $m_{0} = {c_0}^d \bmod N$ is divisible by a small number~$t$. In that case,  $m_{0} \cdot t^{-1} \bmod{N}$ simply equals the natural number $m_{0} / t$. 
If we choose $u \approx t$, and multiply the original message with a fraction $u/t$, the resulting number will lie near the original message: $m_0 \approx m_0 / t \cdot u$.   We shall refer to such fractions as ``small'' fractions.

This method allows us to generate new \sslconform messages with a much higher probability. 
Let $c_0$ be an intercepted \tlsconform RSA ciphertext, and let $m_0 = c_0^d \bmod N$ be its corresponding plaintext.  We select a multiplier $s = u/t \bmod N = u t^{-1} \bmod N$ where $u$ and $t$ are coprime, compute the value $c_1 = c_0 s^e \bmod N$, and query $\Oracle(c_1)$.  We will receive a response if $m_1 = m_0 \cdot u/t$ is \sslconform.  

As an example, let us assume a 2048-bit RSA ciphertext with $k = 5$, and consider the fraction $u = 7, t = 8$.  The probability that a random ciphertext $c_0$ will be \sslconform is 1/7,774, so we expect to make 7,774 oracle queries before discovering a ciphertext $c_0$ for which $c_0 u/t$ is \sslconform, much better than a randomly selected multiplier. Appendix~\ref{sec:fraction-probability} gives more details on computing these probabilities.

\subsubsection{Shifting known plaintext bytes}
\label{sec:rotations}
Once we have obtained an \sslconform ciphertext $c_1$, we have also learned from our oracle information about the $k+1$ least significant bytes ($mk_{secret}$ together with the delimiter byte \hex{00}) and two most significant \hexb{00}{02} bytes of the \sslconform message $m_1$.  We would like to \emph{rotate} these known bits around to the right, so that we have a large block of contiguous known most significant bytes of plaintext.
In this section, we show that this can be accomplished by multiplying by some shift $2^{-r} \bmod N$.  In other words, given an \sslconform ciphertext $c_1 = m_1^e \bmod N$, we can efficiently generate an \sslconform ciphertext $c_2 = m_2^e \bmod N$ where $m_2 = s \cdot m_1 \cdot 2^{-r} \bmod N$ and we know several most significant bytes of $m_2$. 

Let $R = 2^{8(k+1)}$ and $B = 2^{8(\ell-2)}$. Abusing notation slightly, let the integer $m_1 = 2 \cdot B + PS \cdot R + mk_{secret}$ be the plaintext satisfying $m_1^e = c_1 \bmod N$.  At this stage, the $k$-byte integer $mk_{secret}$ is known and the $\ell-k-3$-byte integer $PS$ is not.

Let $\tilde{m_1} = 2 \cdot B + mk_{secret}$ be the known components of $m_1$, so $m_1 = \tilde{m_1} + PS \cdot R$. We can use this to compute a new plaintext for which we know many most significant bytes.  Consider the value 
\[
m_1 \cdot R^{-1} \bmod N = \tilde{m_1} \cdot R^{-1} + PS \bmod N.
\]
The value of $PS$ is unknown, but we know that it consists of $\ell-k-3$ bytes.  This means that the known value $\tilde{m_1} \cdot R^{-1}$ shares most of its $k+3$ most significant bytes with $m_1 \cdot R^{-1}$.

Furthermore, we can iterate this process by finding a new multiplier $s$ such that $m_2 = s \cdot m_1 \cdot R^{-1} \bmod N$ is also \sslconform.  A randomly chosen $s < 2^{30}$ will work with probability $2^{-25.4}$.  We can take advantage of the bytes we have already learned about $m_1$ to efficiently compute such an $s$ with only 678 oracle queries in expectation for a 2048-bit RSA modulus.   Appendix~\ref{sec:rotation-details} gives more details.

\subsubsection{Adapted Bleichenbacher iteration}
\label{sec:bb-iteration}
It is feasible for all of our oracles to use the previous technique to entirely recover a plaintext message.  However, for our \ssltwo protocol oracle it is cheaper to continue using Bleichenbacher's original attack, once we have used the above techniques to obtain a \sslconform message $m_3$ and an integer $s_3$ such that $m_3 \cdot s_3$ is \sslconform.  At this point, we can apply the original algorithm proposed by Bleichenbacher as described in Section~\ref{sec:bleichenbacher}, with minimal modifications.

Each step obtains a message that starts with the required \hexb{00}{02} bytes after two queries in expectation.
Since we know the value of the $k+1$ least significant bytes after multiplying by any integer, we can query the oracle only on multipliers that cause the $(k+1)$st least significant byte to be zero.  However, we cannot ensure that the padding string is entirely nonzero; for a 2048-bit modulus this will hold with probability 0.37.

For a 2048-bit modulus, the total expected number of queries when using this technique to fully decrypt the plaintext is $2048 * 2 / 0.37 \approx 11,000$.


\section{General DROWN} 
\label{vulnerability}

In this section, we describe how any correct \ssltwo implementation that accepts export-grade cipher suites can be used as a padding oracle.  We then show how to adapt the techniques described in Section~\ref{sec:adapted-bb-compact} to decrypt TLS RSA ciphertexts.

\subsection{The SSLv2 export padding oracle} 
\label{vulnerability}
\ssltwo is vulnerable to a direct message side channel vulnerability exposing a Bleichenbacher oracle to the attacker.
The vulnerability follows from three properties of \ssltwo.  First, the server immediately responds with a \texttt{ServerVerify} message after receiving the \texttt{ClientMasterKey} message, which includes the RSA ciphertext, without waiting for the \texttt{ClientFinished} message that proves the client knows the RSA plaintext.  Second, when choosing 40-bit export RC2 or RC4 as the symmetric cipher, only 5 bytes of the \texttt{master\_key} ($mk_{secret}$) are sent encrypted using RSA, and the remaining 11 bytes are sent in cleartext.  Third, a server
implementation that correctly implements the anti-Bleichenbacher countermeasure 
and receives an RSA key exchange message with invalid
padding will generate a random premaster secret and carry out the
rest of the TLS handshake using this randomly generated key material.

This allows an attacker to deduce the validity of RSA ciphertexts in the following manner:

\begin{enumerate}
	\item The attacker sends a \texttt{ClientMasterKey} message, which contains an RSA ciphertext $c_0$ and any sequence of 11 bytes as the clear portion of the \texttt{master\_key}, $mk_{clear}$. The server responds with a \texttt{ServerVerify} message, which contains the \texttt{challenge} encrypted using the \texttt{server\_write\_key}.
	\item The attacker performs an \textit{exhaustive search} over the possible values of the 5 bytes of the \texttt{master\_key} $mk_{secret}$. He then computes the corresponding \texttt{server\_write\_key} and checks whether the \texttt{ServerVerify} message decrypts to the \texttt{challenge}. One value should pass this check; let this value be termed $mk_0$. Recall that if the RSA plaintext was valid, $mk_0$ is the unpadded data in the RSA plaintext. Otherwise, $mk_0$ is a randomly generated sequence of 5 bytes.
	\item The attacker re-connects to the server with the same RSA ciphertext $c_0$. The server responds with another \texttt{ServerVerify} message that contains the current \texttt{challenge} encrypted using the current \texttt{server\_write\_key}. If the decrypted RSA ciphertext was valid, the attacker can directly decrypt a correct \texttt{challenge} value from the \texttt{ServerVerify} message by using the \texttt{master\_key} $mk_0$. Otherwise, if the \texttt{ServerVerify} message does not correctly decrypt to the \texttt{challenge}, the RSA ciphertext was invalid, and the attacker knows the $mk_0$ value was generated at random.
\end{enumerate}

Thus we can instantiate an oracle $\OracleSSLexp$ using the procedure above; each oracle query requires two server connections and $2^{40}$ decryption attempts in the simplest case.  For each oracle call $\OracleSSLexp(c)$, the attacker learns whether $c$ is valid, and if so, learns the two most significant bytes \hexb{00}{02}, the sixth least significant \hex{00} delimiter byte, and the value of the 5 least significant bytes of the plaintext $m$.

If the server does not support 40-bit export ciphers, the attack can also be mounted in feasible computation time by choosing DES as the symmetric cipher.  Choosing DES means the exhaustive search is now done over a key space of 56 bits, thus increasing the cost of the attack by a factor of \begin{math} 2^{16} \end{math}, but does not fundamentally change anything except the increased cost.

\ifsubmit\relax\else
\subsection{OpenSSL special DROWN oracle}

We discovered a vulnerability present in OpenSSL versions prior to March 4, 2015 that allows a client to improperly provide cleartext key bytes for non-export ciphers.  Affected servers will substitute these bytes for bytes from the encrypted key.  This allows a client to successively learn a byte at a time of an encrypted key by brute forcing only 256 possibilities for each query. For a non-export 128-bit cipher suite such as \texttt{SSL\_RC4\_WITH\_MD5}, the attacker learns 19 bytes of the decrypted message.  We describe this vulnerability in more detail in Appendix~\ref{sec:clear-key-vuln}.  A client can then construct a Bleichenbacher oracle from this behavior by validating the \texttt{ServerVerify} message against the candidate key provided in the \texttt{clear\_key\_data}, resulting in no brute-force computation.
\fi

\begin{figure}[t]
	%\centering 
	\includegraphics[width=\linewidth]{img/ssl-tls} 
	\caption{\textbf{Our \ssltwo-based Bleichenbacher attack on TLS\@.} An attacker passively collects RSA ciphertexts from a TLS 1.2 handshake, and then performs oracle queries against a server that supports \ssltwo with the same public key to decrypt the TLS ciphertext.}
	\label{fig:ssl-tls}
\end{figure}

\subsection{TLS decryption attack}

In this section, we describe how the oracle described in Section~\ref{vulnerability} can be used to carry out a feasible attack to decrypt passively collected TLS ciphertexts.

\subsubsection{Attack scenario}
As described in Section~\ref{sec:attack-scenario}, we consider a server that accepts TLS connections from clients using an RSA public key that is exposed via \ssltwo, and an attacker who is able to passively observe these connections.

\paragraph{Server supports export cipher suites for \ssltwo.}
We also assume the server supports export cipher suites for \ssltwo.
This can happen for two reasons.
First, the same servers that fail to follow best practices in disabling \ssltwo~\cite{ProhibitingSSLv2} may also fail to follow best practices by supporting export cipher suites.
Alternatively, the servers might be running a version of OpenSSL prior to January 2016, in which case they are vulnerable to the OpenSSL cipher suite selection bug described in Section~\ref{sec:openssl-selection}, and an attacker may negotiate a cipher suite of his choice independent of the server configuration.

\paragraph{Correct Bleichenbacher countermeasure.}
We assume the server implements the recommended countermeasure against Bleichenbacher's attack in all protocol versions, including \ssltwo. If the decrypted RSA ciphertext has invalid padding, the server generates a random \pms or \texttt{master\_key} and continues the handshake with this random string. We assume this countermeasure is implemented correctly and the server is neither vulnerable to timing nor flush-and-reload side-channel attacks~\cite{Meyer14,Zhang:2014:CSA:2660267.2660356}.

\paragraph{Computing power.}
The attacker needs access to computing power sufficient to perform a $2^{50}$ time attack, mostly brute forcing symmetric key encryption.  After our optimizations, this can be done with a one-time investment of a few thousand dollars of GPUs, or in a few hours for a few hundred dollars in the cloud.  Our cost estimates are described in~Section~\ref{sec:ec2_results}.

\if0
In our simplest attack scenario, an attacker is passively observing many connections between modern clients and server who negotiate a secure TLS version (1.0--1.2) with an RSA cipher suite and a well-generated, secure RSA public key.   The server is also configured to support \ssltwo with the same certificate as in TLS if a client requests it, although the modern victim clients will never negotiate \ssltwo.  The server implements the recommended countermeasures against Bleichenbacher attacks described in Section~\ref{sec:bleichenbacher}.



Our attacker will use our SSLv2 oracle $\OracleSSL$ to decrypt a TLS \texttt{ClientKeyExchange}.  We specialize the discussion below to our protocol-level oracle described in Section~\ref{vulnerability} and refer to this attack as the \emph{general} DROWN attack.  The adaptations to the OpenSSL clear-key oracle, which produces our faster \emph{special} DROWN attack are similar and are described in Section~\ref{sec:special}.
\fi

\subsubsection{Constructing the attack}

The attacker can exploit the \ssltwo vulnerability as illustrated in Figure~\ref{fig:ssl-tls}, following the generic attack outline described in Section~\ref{sec:adapted-bb-compact} and has several distinct phases:
\begin{enumerate}
 \setcounter{enumi}{-1}
	\item He passively collects 1,000 TLS handshakes from connections using RSA key exchange.
	\item The attacker then attempts to convert the intercepted TLS ciphertexts containing a 48-byte \pms to valid RSA \PKCS encoded ciphertexts containing five-byte messages using the fractional trimmers described in Section~\ref{sec:trimmers}, and querying $\OracleSSLexp$. The attacker sends the modified ciphertexts to the server using fresh \ssltwo connections with weak symmetric ciphers and uses the \texttt{ServerVerify} messages to deduce ciphertext validity as described in the previous section. For each queried RSA ciphertext, the attacker must perform a brute force attack on the weak symmetric cipher. The attacker expects to obtain a valid \ssltwo ciphertext after roughly 10,000 oracle queries, or 20,000 connections to the server.
	\item Once the attacker has obtained a valid \ssltwo RSA ciphertext $m_1$, he uses the shifting technique explained in Section~\ref{sec:rotations} to find an integer $s_1$ such that $m_2 = m_1 \cdot 2^{-40} \cdot s_1$ is also \sslconform.  Appendix~\ref{sec:general-rotations} contains more details on this step.
	\item The attacker then applies the shifting technique again to find another integer $s_2$ such that
		$m_3 = m_2 \cdot 2^{-40} \cdot s_2$ is also \sslconform.
	\item He then searches for yet another integer $s_3$ such that $m_3 \cdot s_3$ is also \sslconform.
	\item Finally, the attacker can continue with our adapted Bleichenbacher iteration technique described in Section~\ref{sec:bb-iteration}, and decrypts the message after an expected 10,000 additional oracle queries, or 20,000 connections to the server.
	\item The attacker can then transform the decrypted plaintext back into the original plaintext, which is one of the 1,000 intercepted TLS handshakes.
\end{enumerate}

\paragraph{The rationale behind the different phases.}
Bleichenbacher's original algorithm requires a conformant message $m_0$, and a multiplier $s_1$ such that $m_1 = m_0 \cdot s_1$ is also conformant.
Na\"{\i}vely, it would appear we can apply the same algorithm here, after completing Phase 1.
However, the original algorithm expects $s_1$ to be of size about $2^{24}$. This is not the case when we use fractions for $s_1$, as the integer $s_1 = u t^{-1} \bmod N$ will be the same size as $N$.

Therefore, our approach is to find a conformant message for which we know the 5 most significant bytes; this will happen after multiple rotations and
this message will be $m_3$.
After finding such a message, finding $s_3$ such that $m_4 = m_3 \cdot s_3$ is also conformant becomes trivial.
From there, we can finally apply the adapted Bleichenbacher iteration technique as described in Appendix~\ref{sec:general-bleichenbacher}.

\begin{table}[t]
  \centering
	\begin{tabular}{rrrrrr}
	\toprule
	\textbf{Optimizing} & \textbf{Cipher-} & \textbf{$|F|$}     & \textbf{\ssltwo}  & \textbf{Offline} \\
        \textbf{for}        & \textbf{texts}   &           & \textbf{connections} & \textbf{work} \\
	\midrule
	offline work        &           12,743 &          1 &            50,421  & $2^{49.64}$ \\
        offline work        &            1,055 &         10 &              46,042  & $2^{50.63}$ \\
	% compromise is achieved by using fractions {8/7, 8/9}; numbers are now final
       	compromise          &            4,036 &          2 &              41,081  & $2^{49.98}$ \\
	online work         &            2,321 &          3 &              38,866  & $2^{51.99}$ \\
	online work         &              906 &          8 &              39,437  & $2^{52.25}$ \\
	\bottomrule
	\end{tabular}
        \caption{\textbf{2048-bit Bleichenbacher attack complexity.} The cost to decrypt one ciphertext can be adjusted by choosing the set of fractions $F$ the attacker applies to each of the passively collected ciphertexts in the first step of the attack. This choice affects several parameters: the number of these collected ciphertexts, the number of connections the attacker makes to the \ssltwo server, and the number of offline decryption operations.}% The number of connections for the remaining phases after phase 1 is roughly 24,936, independently of phase 1.}
        \label{tab:reasonable_parameters}
\end{table}

\begin{table}[t]
\begin{tabular*}{\linewidth}{@{\extracolsep{\fill}\hskip\tabcolsep}rrrrr}
\toprule
\textbf{Key size}    & \textbf{Phase 1} & \textbf{Phases 2--5} & \textbf{Total}   & \textbf{Offline} \\
                     &                 &                 & \textbf{queries} & \textbf{work}    \\
\midrule
% This is all when using 4/5, which has a maximal "native probability" of 0.1.
% Offline work is always step1 * 2**38
%              1024 &  (0.1 * 0.62 * 1/256)**(-1)  & + 256 / 0.62 * 2 + 2/0.62 + 1024 * 2 / 0.62
               1024 &  4,129   &  4,132  &  8,261 & $2^{50.01}$ \\

%              2048 & (0.1 * 0.37 * 1/256)**(-1)   & + (2.72 * 2**8) * 2 + 2.72*2 + 2048 * 2 / 0.37 (number for phase 5 also matches runs)
               2048 &  6,919   &  12,468 & 19,387 & $2^{50.76}$ \\

%              4096 & (0.1 * 0.14 * 1/256)**(-1)   & + 2**8 / 0.14 * 2 + 2/0.14 + 4096 * 2 / 0.14
               4096 &  18,286  &  62,185 & 80,471 & $2^{52.16}$ \\
\bottomrule
\end{tabular*}
\caption{\textbf{Oracle queries required by our attack.} In Phase 1, the attacker queries the oracle until an \ssltwo conformant ciphertext is found.  In Phases 2--5, the attacker decrypts this ciphertext using leaked plaintext.  These numbers minimize total queries.  In our attack, an oracle query represents two server connections.}
\label{tab:optimal_queries}
\end{table}

\newcommand{\GPUTable}{
\begin{table*}[t]
 \centering
 \begin{tabular}{llrrrr}
        \toprule
        \textbf{Platform} & \textbf{Hardware} & \textbf{Cost} & \textbf{Full attack}  & \textbf{Cost to perform attack in 1 day} \\
        \midrule
        Na\"{\i}ve CPU & 4 Intel Xeon E7-4820 & $\$21,400$ & $114$ days  & \$$2,440,000$\\
        Na\"{\i}ve GPU & ZOTAC GeForce GTX TITAN & $\$2,400$  & $189$ days & \$$450,000$ \\
        Na\"{\i}ve FPGA & 64 Spartan-6 LX150 & $\$60,000$ & $51.5$ days  & \$$3,090,000$  \\
        \cmidrule{1-5}
        Optimized Hashcat & NVIDIA GTX / AMD R9 & \$18,040 & $0.75$ days & \$13,500 \\
        Optimized EC2 & NVIDIA & \$440 & $0.33$ days & \$147 \\
        \bottomrule
  \end{tabular}	
\caption{\textbf{Time and cost efficiency of our attack on different hardware platforms.} The brute force attacks against symmetric export keys are the most expensive part of our attack.  We compared the performance of a na\"{\i}ve implementation of our attack on different platforms, and decided that a GPU implementation held the most promise.  We then heavily optimized our GPU implementation, obtaining several orders of magnitude in speedup.}
\label{perf_comparison}
\end{table*}
}

\subsubsection{Attack performance}
\label{sec:bb-performance}
The attacker wishes to minimize three major costs in the attack: the number of recorded ciphertexts from the victim client, the number of connections to the victim server, and the number of symmetric keys to be brute forced.
The requirements for each of these elements are governed by the set of fractions to be multiplied with each RSA ciphertext in the first phase, as described in Section~\ref{sec:trimmers}.

Table~\ref{tab:reasonable_parameters} highlights a few choices for $F$ and the resutling performance metrics for 2048-bit RSA keys.
Appendix \ref{sec:general-performance} provides more details on the derivation of these numbers and other possible optimization choices.
Table \ref{tab:optimal_queries} gives the expected number of Bleichenbacher queries for different RSA key sizes, when minimizing total oracle queries.

%These are fractions of small coprime numbers, similar to the example $u/t = 8/7$, and ideally amenable to the additional optimization described above.





%The special DROWN attack requires similar numbers of ciphertexts and oracle queries, but the amount of computation is negligible.



\subsection{Implementing general DROWN with GPUs} 
%\subsection{Optimizing brute forcing export keys}
%% In the following section, we experimentally evaluate the cost of brute forcing export \texttt{master\_key} values on CPU, GPU, FPGA, and cloud computing platforms. 
%% We then experimentally evaluate our general DROWN attack using the 
%% \texttt{SSL\_RC2\_128\_CBC\_EXPORT40\_WITH\_MD5}
%% cipher suite, which is the most suitable for this attack.

%% \subsubsection{Comparing hardware platforms}
\if0
The most computationally expensive part of our general DROWN attack is breaking the 40-bit symmetric key.  We wanted to find the platform that would have the best tradeoff of cost and speed for the attack, so we performed some preliminary experiments comparing performance of symmetric key breaking on CPUs, GPUs, and FPGAs.  These experiments used a na\"{\i}ve version of the attack using the OpenSSL implementation of MD5 and RC2.

The CPU machine contained four Intel Xeon E7-4820 CPUs with a total of 32 cores (64 concurrent threads). The GPU system was equipped with a ZOTAC GeForce GTX TITAN and an Intel Xeon E5-1620 host CPU\@. The FPGA setup consisted of 64 Spartan-6 LX150 FPGAs.

We benchmarked the performance of the CPU and GPU implementations over a large corpus of randomly generated keys, and then extrapolated to the full attack.
For the FPGAs, we tested the functionality in simulation and estimated the actual runtime by theoretically filling the FPGA up to 90\% with the design, including communication.
Table~\ref{perf_comparison} compares the three platforms.

While the FPGA implementation was the fastest in our test setup, the speed-to-cost ratio of GPUs was the most promising. Therefore, we decided to focus on optimizing the attack on the GPU platform.
\fi

\ifext
\subsubsection{Optimized GPU implementation}
\label{sec:gpu_brief}
We developed a highly optimized GPU implementation of our general DROWN brute force attack.  Our first na\"{\i}ve GPU implementation performed around 26MH/s, where MH measures the calculation of an MD5 hash and the RC2 decryption. The optimizations described below gave a final speed of 515MH/s, a speedup factor of 19.8.
\fi

\ifext
We obtained our improvements through a number of optimizations.  Our original implementation ran into a communication bottleneck in the PCI-E bus in transmitting candidate keys from CPU to GPU, so we removed this bottleneck by generating key candidates on the GPU itself.  We optimized memory management, including storing candidate keys and the RC2 permutation table in constant memory, which is almost as fast as a register, instead of slow global memory.  We optimized the cryptographic checks themselves by rewriting the RC2 implementation to use 32-bit instructions, removing unnecessary RC2 keysize checks, dropping unused ADD instructions during MD5, and manually shifting input bytes into the MD5 input registers to avoid loop branches.
\looseness=1
\fi

\ifext

\label{sec:ec2_results}
%In this section, we discuss attack performance and cost when a rented cloud compute cluster is used for the GPU breaking.
\fi

The most computationally expensive part of our general DROWN attack is breaking the 40-bit symmetric key, so we developed a highly optimized GPU implementation of this brute force attack.  Our first na\"{\i}ve GPU implementation performed around 26MH/s, where MH denotes the time required for testing one million possible values of $mk_{secret}$. Our optimized implementation runs at a final speed of 515MH/s, a speedup factor of 19.8.  
\label{sec:gpu_brief}

We obtained our improvements through a number of optimizations.  For example, our original implementation ran into a communication bottleneck in the PCI-E bus in transmitting candidate keys from CPU to GPU, so we removed this bottleneck by generating key candidates on the GPU itself.  We optimized memory management, including storing candidate keys and the RC2 permutation table in constant memory, which is almost as fast as a register, instead of slow global memory. 
\ifext  We optimized the cryptographic checks themselves by rewriting the RC2 implementation to use 32-bit instructions, removing unnecessary RC2 keysize checks, dropping unused ADD instructions during MD5, and manually shifting input bytes into the MD5 input registers to avoid loop branches.  We describe these optimizations in further detail in Appendix~\ref{sec:gpu}. \fi

We experimentally evaluated our optimized implementation on a local cluster and in the cloud.
We used it to execute a full attack of $2^{49.6}$ tested keys on each platform.
The required number of keys to test during the attack is a random variable, distributed geometrically, with an expectation that ranges between $2^{49.6}$ and $2^{52.5}$ depending on the choice of optimization parameters.
We treat a full attack as requiring $2^{49.6}$ tested keys overall.

\paragraph{Hashcat.}
Hashcat~\cite{hashcat} is an open source optimized password-recovery tool.
The Hashcat developers allowed us to use their GPU servers for our attack evaluation. 
The servers contain a total of 40 GPUs: 32 Nvidia GTX 980 cards, and 8 AMD R9 290X cards.
The value of this equipment is roughly \$18,040.
Our full attack took less than 18 hours to complete on the Hashcat servers, with the longest single instance taking 17h9m.

% Nimrod: Generally, we can't report times which are different from wallclock times without further explanation,
% like "retroactively assuming a perfecly balanced distribution..."
% The instance that took longest took 1029m41.176, which is 17.16 hours,
% so I think any number smaller than 18 hours would need to be discussed.

\paragraph{Amazon EC2.}
\label{sec:ec2_results}
% AH: Suggest trimming this as in the submitted version
\ifext
%\section{Brute Forcing Keys with Amazon EC2} 
\label{sec:ec2-details}

%Amazon Elastic Cloud Compute (EC2)~\cite{ec2} is a service that provides on-demand virtualized compute resources to customers. 
%This is an affordable alternative to provisioning one's own local cluster.

Amazon EC2 billing is based on the \textit{instance-hour}. An \textit{instance} represents a single virtualized machine and its associated cores, memory, and storage. For our experiments we used \texttt{g2} instances, which are equipped with high-performance NVIDIA GPUs, each with 1,536 CUDA cores. The two available models for this instance type are the \texttt{g2.2xlarge} and the \texttt{g2.8xlarge}, containing one and four GPUs, respectively.

It is possible to request instances at a fixed on-demand rate, or bid on instances at the discounted spot instance rate. Spot instances may be terminated depending on demand, but the savings in cost are significant compared to the on-demand rate. 
When we ran our experiments in January 2016, the on-demand rate for the \texttt{g2.2xlarge} model was \$0.65/hr and the rate for the \texttt{g2.8xlarge} model was \$2.65/hr, while the average spot rates we paid were \$0.09/hr and \$0.83/hr respectively.

We used a cluster composed of 200 spot instances: 150 \texttt{g2.2xlarge} which contain one GPU and 50 \texttt{g2.8xlarge}, each containing four GPUs, spread across multiple availability zones within the US-East region.
This distribution was determined by price: we were not able to launch more than 50 \texttt{g2.8xlarge} instances without a sharp spike in spot prices. We used the optimized Hashcat implementation on the same workload of key requests as the experiments run on the Hashcat servers.  We used Slurm~\cite{yoo2003slurm} to distribute jobs across compute nodes.

The GPU breaking experiment completed successfully, with two minor caveats. First, the 150 \texttt{g2.2xlarge} nodes completed their workloads at the 6h26m mark, while the other 50 \texttt{g2.8xlarge} nodes did not finish until the 7h41m mark. More careful job distribution would ensure that all nodes completed at approximately the same time, reducing the overall runtime. Second, in this particular run, $7.2\%$ of the jobs that we expected to complete were terminated early due to overheating GPUs.  The attack was successful despite the failed jobs, so we did not rerun them. In a more carefully engineered implementation, the unfinished jobs could have been reallocated to the unused GPU capacity without increasing the overall runtime.

The total cost of the experiment was \$440, and terminated in under 8 hours including startup and shutdown.


\else
We also ran our optimized GPU code on the Amazon Elastic Compute Cloud (EC2) service.  We used a cluster composed of 200 variable-price ``spot'' instances: 150 \texttt{g2.2xlarge} instances, each containing one high-performance NVIDIA GPU with 1,536 CUDA cores and 50 \texttt{g2.8xlarge} instances, each containing four of these GPUs.  
When we ran our experiments in January 2016, the average spot rates we paid were \$0.09/hr and \$0.83/hr respectively.  
%The 150 \texttt{g2.2xlarge} nodes finished after 6h26m, while the \texttt{g2.8xlarge} finished after 7h41m.  $7.2\%$ of the jobs that we expected to complete failed due to overheating GPUs.  The attack was successful despite the failed jobs, so we did not rerun them.  
Our full attack finished in under 8 hours including startup and shutdown for a cost of \$440.  
\ifext See Appendix~\ref{sec:ec2-details} for more details. \fi
\fi


\subsection{OpenSSL SSLv2 cipher suite selection bug}

General DROWN is a protocol flaw, but the population of vulnerable hosts is
increased due to a bug in OpenSSL that causes many servers to erroneously
support \ssltwo and export ciphers even when configured not to. The OpenSSL
team intended to disable \ssltwo by default in 2010, with a change that removed
all \ssltwo cipher suites from the default list of ciphers offered by the
server~\cite{openssl-changelog}.  However, the code for the protocol itself was
not removed in standard builds and \ssltwo itself remained enabled. We
discovered a bug in OpenSSL's \ssltwo cipher suite negotiation logic that
allows clients to select \ssltwo cipher suites even when they are not
explicitly offered by the server. We notified the OpenSSL team of this
vulnerability, which was assigned CVE-2015-3197.  The problem was fixed in
OpenSSL releases 1.0.2f and 1.0.1r~\cite{openssl-changelog}.
%\looseness=1


\section{Special DROWN}
\label{sec:special}

We discovered a vulnerability in recent
(but not current) versions of the OpenSSL SSLv2 handshake code that
creates a powerful Bleichenbacher oracle, and drastically reduces the amount
of computation required to implement our attack.  
The vulnerability, which has been designated CVE-2016-0703, was
present in the OpenSSL codebase from at least the start of the repository,
in 1998, until it was unknowingly fixed on March 4, 2015 by a
patch~\cite{openssl-clear-patch} designed to correct an unrelated
problem~\cite{CVE-2015-0293}.
By adapting DROWN to
exploit this special case, we can cut the number of connections
required by more than 50\% and reduce the computational work to a negligible amount.

%To distinguish the two, we call the attack developed so far
%\emph{general} DROWN and refer to the variant that
%exploits the OpenSSL bug as \emph{special} DROWN\@.  General DROWN is
%a protocol-level attack that makes few assumptions about the SSLv2
%server, other than that it allows export cipher handshakes.  Special
%DROWN exploits a specific implementation bug, but it is highly practical.
%It may be the first widespread Bleichenbacher vulnerability within reach of ``script-kiddies'' and other low-resource
%attackers.

%% This dramatic discovery came too late to fully incorporate into the
%% body of our submission, so for now we confine the bulk of the
%% discussion to this section.

% !TEX root = ../../../proposal.tex
\subsection{The OpenSSL ``extra clear'' oracle}

%\subsubsection{A key recovery attack on SSLv2 handshakes}

\label{sec:clear-key-vuln}

Prior to the fix, OpenSSL servers improperly allowed the \texttt{ClientMasterKey} message to contain
\texttt{clear\_key\_data} bytes for \emph{non-export} ciphers.  When such bytes are present,
the server substitutes them for bytes from the
encrypted key. For example, consider the case that the client chooses a 128-bit cipher and sends a 16-byte
encrypted key $k\pos{1}, k\pos{2}, \ldots, k\pos{16}$ but, contrary to the protocol specification, includes 4
null bytes of \texttt{clear\_key\_data}. Vulnerable OpenSSL versions will
construct the following \texttt{master\_key}:

\small $[00\ 00\ 00\ 00\ k\pos{1}\ k\pos{2}\ k\pos{3}\ k\pos{4}\ \dots\ k\pos{9}\ k\pos{10}\ k\pos{11}\ k\pos{12}]$\normalsize

This enables a straightforward key recovery attack against such versions.
An attacker that has intercepted an \ssltwo connection takes the RSA
ciphertext of the encrypted key and replays it in non-export handshakes to
the server with varying lengths of \texttt{clear\_key\_data}. For a 16-byte
encrypted key, the attacker starts with 15 bytes of clear key, causing the server to use the \texttt{master\_key}:

\small$[00\ 00\ 00\ 00\ 00\ 00\ 00\ 00\ 00\ 00\ 00\ 00\ 00\ 00\ 00\ k\pos{1}]$\normalsize

The attacker can brute force the first byte of the encrypted key by
finding the matching \texttt{ServerVerify} message among 256
possibilities. Knowing $k\pos{1}$, the attacker makes another
connection with the same RSA ciphertext but 14 bytes of clear key,
resulting in the \texttt{master\_key}:

\small $[00\ 00\ 00\ 00\ 00\ 00\ 00\ 00\ 00\ 00\ 00\ 00\ 00\ 00\ k\pos{1}\ k\pos{2}]$\normalsize

% Note: The number below is 15, since you the original intercepted connection
% servers as the case with 0 bytes of clear_key_data.
The attacker can now easily brute force $k\pos{2}$. With only 15 probe
connections and an expected $15 \cdot 128 = 1,920$
trial encryptions, the attacker learns the entire \texttt{master\_key} for the
recorded session.

%\subsubsection{An improved oracle for DROWN}

As this oracle is obtained by improperly sending unexpected clear-key bytes,
we call it the Extra Clear oracle.

This session key-recovery attack can be directly converted to a Bleichenbacher oracle. Given a candidate ciphertext and symmetric key length $\ell_k$, the attacker sends the ciphertext with $\ell_k$ known bytes of \texttt{clear\_key\_data}. The oracle decision is simple:
\begin{itemize}
\item If the ciphertext is valid, the \texttt{ServerVerify} message will reflect a \texttt{master\_key} consisting of those $\ell_k$ known bytes.
\item If the ciphertext is invalid, the \texttt{master\_key} will be replaced with $\ell_k$ random bytes (by following the countermeasure against the Bleichenbacher attack), resulting in a different \texttt{ServerVerify} message.
\end{itemize}

This oracle decision requires one connection to the server and one \texttt{ServerVerify} computation. After the attacker has found a valid ciphertext corresponding to a $\ell_k$-byte encrypted key, they recover the $\ell_k$ plaintext bytes by repeating the key recovery attack from above.  Thus our oracle $\OracleSSLclear(c)$ requires one connection to determine whether $c$ is valid.  After $\ell_k$ connections, the attacker additionally learns the $\ell_k$ least significant bytes of $m$.  We model this as a single oracle call, but the number of server connections will vary depending on the response.


\tabDrownAll

\subsection{TLS decryption with special DROWN}
\label{sec:clear_analysis}

Using our oracle $\OracleSSLclear$, we can construct an extremely efficient version of our TLS decryption attack.  The OpenSSL \tOracleSSLclear provides three significant advantages over our export oracle $\OracleSSLexp$: (1) It no longer requires an export cipher suite, and, in fact, we gain efficiency by exploiting regular SSLv2 ciphers; (2) It requires only one handshake per oracle query; and (3) Computation is reduced to one \texttt{ServerVerify} decryption per oracle query, versus $2^{40}$.

\subsubsection{Attack scenario}

As before, we consider a server that accepts TLS connections, and a client that negotiates a secure, state-of-the-art TLS version with a \texttt{TLS\_RSA} cipher suite.  The same RSA key pair used for TLS is also used on a server that is running a vulnerable version of OpenSSL.

\subsubsection{Constructing the attack}

The attacker can exploit the OpenSSL extra clear vulnerability to efficiently decrypt a TLS ciphertext as follows.  We will use the cipher suite \texttt{SSL\_DES\_192\_EDE3\_CBC\_WITH\_MD5} as the cipher suite, allowing the attacker to recover 24 bytes of key at a time from the oracle.
We first present a straightforward adaptation of the general DROWN attack to the \tOracleSSLclear,
before later applying a few additional optimizations made possible by this new oracle.

\begin{enumerate}
 \setcounter{enumi}{-1}
 \item The attacker intercepts several hundred TLS handshakes using RSA key exchange.
 \item The attacker uses the fractional trimmers as described in Section~\ref{sec:trimmers} to convert the TLS ciphertexts into an \sslconform ciphertext $c_0$.
 \item Once the attacker has obtained a valid \ssltwo ciphertext $c_1$, he repeatedly uses the shifting technique described in Section~\ref{sec:rotations} to rotate the message by 25 bytes each iteration, learning 27 bytes with each shift.  After several iterations, he has learned the entire plaintext.
 \item The attacker then transforms the decrypted \ssltwo plaintext into the decrypted TLS plaintext. 
 \end{enumerate}

\paragraph{Attack costs}
Using 40 fractional trimmers, this more efficient oracle attack allows
the attacker to recover one in 260 TLS session keys using only about
17,000 connections to the server.  The computation cost is so low that
we can complete the full attack on a single workstation in under one
minute. Appendix~\ref{sec:special-performance} gives more details.

Mounting the attack using the optimized version of Special DROWN
described in Appendix~\ref{sec:special-performance} allows the
attacker to target one of 100 connections, at the expense of
increasing the number of queries to 27,000.

\subsection{MITM attack against TLS}

Special DROWN is fast enough that it can decrypt a TLS premaster
secret \emph{online}, during a connection handshake.  A
man-in-the-middle attacker can use it to compromise connections
between modern browsers and TLS servers---even those configured to
prefer non-RSA cipher suites.

\paragraph{Attack scenario.}
The MITM attacker impersonates the server and sends a
\texttt{ServerHello} message that selects a cipher suite with RSA as
the key-exchange method.  Then, the attacker uses special DROWN to
decrypt the \pms.  The main difficulty is completing the decryption and producing a valid
\texttt{ServerFinished} message before the client's connection times
out.  Most browsers will allow the handshake to last up to one minute~\cite{LogJam}.

Using the fully optimized version of special DROWN, the attack still requires intercepting 
an average of 100 ciphertexts, only one of
which will be decrypted, probabilistically.  The simplest
way for the attacker to facilitate this is to use JavaScript to cause
the client to connect repeatedly to the victim server, as described in
Section~\ref{sec:attack-scenario}.  Each connection is tested
against the oracle with only small number of fractions, and the attacker can discern
immediately when he receives a positive response from the oracle.

Once the attacker has obtained a positive response, he
can proceed to the final phase of the special DROWN attack described above, 
which employs 200-bit rotation 10 times to fully decrypt the
plaintext.   Our current implementation requires under
30 seconds for this phase on a single PC.

% AH: The 34 microsecond number for 2048-bit RSA doesn't seem right.
% Here's a trial with openssl on a fairly modern server:
%
%% $ openssl speed rsa
%% Doing 512 bit private rsa's for 10s: 176064 512 bit private RSA's in 10.00s
%% Doing 512 bit public rsa's for 10s: 2092137 512 bit public RSA's in 10.00s
%% Doing 1024 bit private rsa's for 10s: 53091 1024 bit private RSA's in 10.00s
%% Doing 1024 bit public rsa's for 10s: 779785 1024 bit public RSA's in 10.00s
%% Doing 2048 bit private rsa's for 10s: 7199 2048 bit private RSA's in 10.00s
%% Doing 2048 bit public rsa's for 10s: 234671 2048 bit public RSA's in 10.00s
%% Doing 4096 bit private rsa's for 10s: 1005 4096 bit private RSA's in 10.01s
%% Doing 4096 bit public rsa's for 10s: 63173 4096 bit public RSA's in 10.01s

% Nimrod: Let's go with your number then.
% Just for the record, I got the number from the QUIC Crypto document:
% https://docs.google.com/document/d/1g5nIXAIkN_Y-7XJW5K45IblHd_L2f5LTaDUDwvZ5L6g/edit#heading=h.bzxklo2i5w6k

The ability of the victim server to perform 17,000 handshakes in less than a
minute is not an impediment for modern hardware.  An RSA
private key operation with a 2048-bit modulus requires on the order of
1~ms using OpenSSL on a recent-generation CPU, so the cryptographic
portion of the attacker's queries induces additional server load of
roughly 14~core-seconds.  In tests with a nearby server running Apache
2.4, we could easily complete 10,000 HTTPS requests in under 10
seconds.

% AH: Here's further data in support of this fact, tested against a local
% server.  10k TLS/RSA connections took about 7 seconds to complete.

%% $ ab -n 10000 -c 100 -Z AES128-SHA https://www.eecs.umich.edu/x
%% This is ApacheBench, Version 2.3 <$Revision: 1528965 $>
%% Copyright 1996 Adam Twiss, Zeus Technology Ltd, http://www.zeustech.net/
%% Licensed to The Apache Software Foundation, http://www.apache.org/

%% Benchmarking www.eecs.umich.edu (be patient)

%% Server Software:        Apache/2.2.15
%% Server Hostname:        www.eecs.umich.edu
%% Server Port:            443
%% SSL/TLS Protocol:       TLSv1.2,AES128-SHA,2048,128

%% Document Path:          /x
%% Document Length:        285 bytes

%% Concurrency Level:      100
%% Time taken for tests:   7.130 seconds
%% Complete requests:      10000
%% Failed requests:        0
%% Non-2xx responses:      10000
%% Total transferred:      4660000 bytes
%% HTML transferred:       2850000 bytes
%% Requests per second:    1402.45 [#/sec] (mean)
%% Time per request:       71.304 [ms] (mean)
%% Time per request:       0.713 [ms] (mean, across all concurrent requests)
%% Transfer rate:          638.22 [Kbytes/sec] received

%% Connection Times (ms)
%%               min  mean[+/-sd] median   max
%% Connect:       28   57  47.7     52    1069
%% Processing:     7   13   7.2     13     234
%% Waiting:        7   12   7.2     12     230
%% Total:         47   70  48.1     65    1082






%In addition to the \tOracleSSLclear $\OracleSSLclear$ described in Section~\ref{sec:special},
%the same set of OpenSSL versions allow a different oracle,
%which we term \tOracleSSLleaky $\OracleSSLleaky$.
%This additional oracle is powerful in a different way than
%\tOracleSSLclear --- it still requires roughly $2^{40}$ offline computations
%per query, but it is more permissive when checking for conformant messages.

\subsection{The OpenSSL ``leaky export'' oracle}
In addition to the extra clear implementation bug, the same set of OpenSSL versions
also contain a separate bug, where they do not follow the correct algorithm
for their implementation of the Bleichenbacher countermeasure.
We now describe this faulty implementation:
\begin{itemize}
	\item The \ssltwo \texttt{ClientKeyExchange} message contains the
	$mk_{clear}$ bytes immediately before the ciphertext $c$. Let $p$
	be the buffer starting at the first $mk_{clear}$ byte.

	\item Decrypt $c$ in place. If the decryption operation succeeds,
	and $c$ decrypted to a plaintext of a correct padded length,
	$p$ now contains the 11 $mk_{clear}$ bytes followed by the 5
	$mk_{secret}$ bytes.

	\item If $c$ decrypted to an unpadded plaintext $k$ of incorrect length,
	the decryption operation overwrites the first $j = min(|k|, 5)$ bytes
	of $c$ with the first $j$ bytes of $k$.

	\item If $c$ is not \sslconform and the decryption operation failed,
	randomize the first five bytes of $p$, which are the first
	five bytes of $mk_{clear}$.
\end{itemize}

This behavior allows the attacker to distinguish between these three cases.
Suppose the attacker sends 11 null bytes as $mk_{clear}$.
Then these are the possible cases:

\begin{enumerate}
\item $c$ decrypts to a correctly padded plaintext $k$ of the expected length, 5
	bytes. Then the following \texttt{master\_key} will be constructed:\smallskip\small\\
	$[00\ 00\ 00\ 00\ 00\ 00\ 00\ 00\ 00\ 00\ 00\ k\pos{1}\ k\pos{2}\ k\pos{3}\ k\pos{4}\ k\pos{5}]$
	\normalsize
\item $c$ decrypts to a correctly padded plaintext $k$ of a wrong length.
	Let $r$ be the five random bytes the server generated.
	The yielded \texttt{master\_key} will be:\smallskip\small\\
	\hspace*{-12pt}$[r\pos{1}\ r\pos{2}\ r\pos{3}\ r\pos{4}\ r\pos{5}\ 00\ 00\ 00\ 00\ 00\ 00\ k\pos{1}\ k\pos{2}\ k\pos{3}\ k\pos{4}\ k\pos{5}]$\medskip\\
	\normalsize
    when $|k| \ge 5$. If $|k| < 5$, the server substitutes the
	first $|k|$ bytes of $c$
	with the first $|k|$ bytes of $k$.
	Using $|k| = 3$ as an example, 
	the \texttt{master\_key} will be:\smallskip\small\\
	\hspace*{-12pt}$[r\pos{1}\ r\pos{2}\ r\pos{3}\ r\pos{4}\ r\pos{5}\ 00\ 00\ 00\ 00\ 00\ 00\ k\pos{1}\ k\pos{2}\ k\pos{3}\ c\pos{4}\ c\pos{5}]$\normalsize\vspace{-11pt}
%\ifext	where $c\pos{1}, \ldots, c\pos{5}$ denote the first five bytes of the RSA ciphertext $c$. \fi
\item $c$ is not \sslconform, and hence the decryption operation failed.
	The resulting \texttt{master\_key} will be:\medskip\small\\
	\hspace*{-12pt}$[r\pos{1}\ r\pos{2}\ r\pos{3}\ r\pos{4}\ r\pos{5}\ 00\ 00\ 00\ 00\ 00\ 00\ c\pos{1}\ c\pos{2}\ c\pos{3}\ c\pos{4}\ c\pos{5}]$
\end{enumerate}
The attacker detects case (3) by performing an exhaustive search over the
$2^{40}$ possibilities for $r$, and checking whether any of the resulting
values for the \texttt{master\_key} correctly decrypts the observed
\texttt{ServerVerify} message. If no $r$ value satisfies this property, then
$c^d$ starts with bytes \hexb{00}{02}. The attacker then distinguishes between
cases (1) and (2) by performing an exhaustive search over the five bytes of $k$,
and checking whether any of the resulting values for $mk$ correctly 
decrypts the observed \texttt{ServerVerify} message.

As this oracle leaks information when using export ciphers,
we have named it the Leaky Export oracle.

In conclusion, $\OracleSSLleaky$ allows an attacker to obtain a valid oracle response
for all ciphertexts which decrypt to a correctly-padded plaintext of \textit{any} length. This is in contrary to the previous oracles $\OracleSSLclear$ and $\OracleSSLexp$, which required the plaintext to be of a specific length.
Each oracle query to $\OracleSSLleaky$ requires one connection to the server
and $2^{41}$ offline work.


\paragraph{Combining the two oracles.}
\label{sec:special_drown_summary}

The attacker can use the Extra Clear and Leaky Export oracles
together in order to reduce the number of queries required for the TLS decryption attack.
They first test a \tlsconform ciphertext for divisors using the Leaky Export oracle, then use fractions dividing the plaintext with both oracles.
Once the attacker has obtained a valid \ssltwo ciphertext $c_1$, they repeatedly
use the shifting technique described in Section~\ref{sec:rotations} to rotate
the message by 25 bytes each iteration while choosing 3DES as the
symmetric cipher, learning 27 bytes with each shift.  After several iterations,
they have learned the entire plaintext, using 6,300 queries (again for a 2048-bit
modulus).
This brings the overall number of queries for this variant of the attack to
$ 900 + 16 * 4 + 6,300 = 7,264 $.
These parameter choices are not necessarily optimal.  We give more details in Appendix~\ref{sec:special-both}.


%\section{MITM attacks and QUIC}

%An attacker can also use a Bleichenbacher-type attack to compute valid RSA signatures on arbitrary messages.  Mathematically, RSA signing and decryption are identical.  Such an attack could theoretically be used to forge a signed Server Key Exchange message for Diffie-Hellman cipher suites, thus allowing an attacker to perform a man-in-the-middle attack against all TLS versions up to TLSv1.3.~\cite{tls-quic-pkcs-2015}  Since the server key exchange message includes the client and server randoms, the attacker must forge the signature online before the handshake times out. We are not able to use all of our optimizations for signature forgery, so such an attack does not seem feasible without additional improvements, even for special DROWN.

\section{Extending the attack to QUIC}
\label{sec:quic}

DROWN can also be extended to a feasible-time man-in-the-middle attack
against QUIC~\cite{tls-quic-pkcs-2015}. QUIC~\cite{quic,
quic-langley-2014} is a recent cryptographic protocol designed and implemented
by Google that is intended to reduce the setup time to establish a secure
connection while providing security guarantees analogous to TLS\@. QUIC's
security relies on a static ``server config'' message signed by the server's
public key. Jager et al.\@~\cite{tls-quic-pkcs-2015} observe that
an attacker who can forge a signature on a malicious QUIC server config once
would be able to impersonate the server indefinitely. In this section, we
show an attacker with significant resources would be able to mount such an
attack against a server whose RSA public keys is exposed via \ssltwo.

A QUIC client receives a ``server config'' message, signed by the server's public key, which enumerates connection parameters: a static elliptic curve Diffie-Hellman public value, and a validity period.  In order to mount a man-in-the-middle attack against any client, the attacker wishes to generate a valid server config message containing their own Diffie-Hellman value, and an expiration date far in the future.

%\paragraph{Unauthenticated QUIC discovery.}
The attacker needs to present a forged QUIC config to the client in order to carry out a successful attack.  This is straightforward, since QUIC discovery may happen over non-encrypted HTTP~\cite{QUICDiscovery}.  The server does not even need to support QUIC at all: an attacker could impersonate the attacked server over an unencrypted HTTP connection and falsely indicate that the server supports QUIC\@. The next time the client connects to the server, it will attempt to connect using QUIC, allowing the attacker to present the forged ``server config'' message and execute the attack~\cite{tls-quic-pkcs-2015}.

\begin{table}[t]
  \centering
% The resizing seems to be minimal, but necessary. This is allowed, right?
\resizebox{\columnwidth}{!}{
	\begin{tabular}{rrrrrrr}
	\toprule
	\textbf{Pro-}   & \textbf{Attack}  & \textbf{Oracle}   & \textbf{\ssltwo} & \textbf{Offline} & \textbf{See}                     \\
        \textbf{tocol}  & \textbf{type}    &                   & \textbf{connec-} & \textbf{work}    & \textbf{\S}                    \\
                        &                  &                   & \textbf{tions}   &                  &                     \\
	\midrule
	TLS             & Decrypt          & \ssltwo           &         $41,081$ & $2^{50}$      & \ref{sec:bb-performance}         \\
        TLS             & Decrypt          & Special           &          $7,264$ & $2^{51}$         & \ref{sec:special_drown_summary}  \\
        TLS             & MITM             & Special           &         $27,000$ & $2^{15}$         & \ref{sec:special_mitm_tls}       \\
	QUIC            & MITM             & \ssltwo           &         $2^{25}$ & $2^{65}$         & \ref{sec:quic_general_drown}     \\
	QUIC            & MITM             & Special           &         $2^{25}$ & $2^{25}$         & \ref{sec:quic_special_drown}     \\
	QUIC            & MITM             & Special           &         $2^{17}$ & $2^{58}$         & \ref{sec:quic_special_drown}     \\
	\bottomrule
	\end{tabular}
}
        \caption{\textbf{Summary of attacks.}
		``Oracle'' denotes the oracle required to mount each attack,
		which also implies the vulnerable set of \ssltwo implementations.
		\ssltwo denotes any \ssltwo implementation,
		while ``Special'' denotes an OpenSSL version vulnerable to
		special DROWN.\@
        }
        \label{tab:attacks_summary}
\end{table}

\subsection{QUIC signature forgery attack based on general DROWN}
\label{sec:quic_general_drown}

The attack proceeds much as in Section~\ref{sec:adapted-bb-compact}, except that some of the optimizations are no longer applicable, making the attack more expensive.

The first step is to discover a valid, PKCS conformant \ssltwo ciphertext.  In the case of TLS decryption, the input ciphertext was PKCS conformant to begin with; this is not the case for the QUIC message $c_0$.  
Thus for the first phase, the attacker iterates through possible multiplier values $s$ until they randomly encounter a valid \ssltwo message in $c_0 \cdot s^d$. 
For 2048-bit RSA keys, the probability of this random event is $P_{rnd} \approx 2^{-25}$; see Section~\ref{sec:adapted-bb-compact}.

Once the first \sslconform message is found, the attacker proceeds with the signature forgery as they would in Step 2 of the  TLS decryption attack\@. The required number of oracle queries for this step is roughly 12,468 for 2048-bit RSA keys.

\paragraph{Attack cost.}
The overall oracle query cost is dominated by the $2^{25} \approx 34$ million expected queries in the first phase, above.  At a rate of 388 queries/second, the attacker would finish in one day; at a rate of 12 queries/second they would finish in one month.

For the \ssltwo export padding oracle, the offline computation to break a 40-bit symmetric key for each query requires iterating over $2^{65}$ keys.
At our optimized GPU implementation rate of 515 million keys per second, this would require 829,142 GPU days.
Our experimental GPU hardware retails for \$400.  An investment of \$10 million to purchase 25,000 GPUs would reduce the wall clock time for the attack to 33 days.

Our implementation run on Amazon EC2 processed about 174 billion keys per \texttt{g2.2xlarge} instance-hour, so at a cost of \$0.09/instance-hour the full attack would cost \$9.5 million and could be parallelized to Amazon's capacity.

\tabDrownAll

\subsection{Optimized QUIC signature forgery based on special DROWN}
\label{sec:quic_special_drown}
For targeted servers that are vulnerable to special DROWN, we are unaware of
a way to combine the two special DROWN oracles; the attacker would have to
choose a single oracle which minimizes his subjective cost.
For the \tOracleSSLclear, there is only negligible computation per oracle query, so the computational cost for the first phase is $2^{25}$.
For the \tOracleSSLleaky, as explained below, the required offline work is $2^{58}$,
and the required number of server connections is roughly $145,573$.
Both oracles appear to bring this attack well within the means of a
moderately provisioned adversary.
\looseness=1

\paragraph{Mounting the attack using Leaky Export.}
For a 2048-bit RSA modulus, the probability of a random message being conformant
when querying $\OracleSSLleaky$ is
$P_{rnd} \approx (1/256)^2 * (255/256)^{8} * (1 - (255/256)^{246}) \approx 2^{-17}$.
Therefore, to compute $c^d$ when $c$ is not \sslconform,
the attacker randomly generates values for $s$ and tests
$c \cdot s^{e}$ against the \tOracleSSLleaky.
After roughly $2^{17} \approx 131,000$ queries, they obtain a positive response,
and learn that $c^d \cdot s$ starts with bytes \hexb{00}{02}.

Na\"{\i}vely, it would seem the attacker can then apply one of the techniques
presented in this work, but $\OracleSSLleaky$ does not provide knowledge of
any least significant plaintext bytes when the plaintext length is not 
at most the correct one.
Instead, the attacker proceeds directly according to the algorithm presented
in~\cite{efficient-padding-oracle-2012}.
Referring to Table 1 in~\cite{efficient-padding-oracle-2012},
$\OracleSSLleaky$ is denoted with the term \texttt{FFT},
as it returns a positive response for a correctly padded plaintext of any length,
and the median number of required queries for this oracle is 14,501.
This number of queries is dominated by the 131,000 queries the attacker has already executed.
As each query requires testing roughly $2^{41}$ keys, the required offline work is
approximately $2^{58}$.

\ifext
\paragraph{Attack detectability.}
\todo{A. Let's make it clear that Google and Wikipedia are not vulnerable.
B. Not all queries require re-handshakes, so there will be likely an increase in server load that is proportionally higher than just increase in traffic.
We'll address both points after the deadline.}

A victim server might be expected to notice the large number of queries required to execute the attack.  However, our query complexity is dwarfed by the amount of traffic that large sites such as Google and Wikipedia receive daily.  
Google is said to process 3.5 billion search queries a day.
$2^{26}$ server connections performed over four days corresponds to about 1\% of this amount.
Similarly, Wikipedia received 16 billion page views during January 2016~\cite{WikipediaStats}.
An attacker who made $2^{26}$ connections over a period of twelve days would result in a 1\% increase in traffic.
\fi

\paragraph{Future changes to QUIC\@.}
In addition to disabling QUIC support for non-whitelisted servers, Google have informed us that they plan to change the QUIC standard, so that the ``server config'' message will include a client nonce to prove freshness. They also plan to limit QUIC discovery to HTTPS.

%Table~\ref{tab:attacks_summary} summarizes the attacks presented in this work.

\ifext
\subsection{SSLv2 servers with CA certificates} 
Some web servers support \ssltwo while presenting a CA certificate,
which can be used to issue further leaf certificates. In that case, an attacker
could create his own certificate and use the vulnerable server to forge a CA
signature over his certificate by executing an attack similar to the above.
The number of queries is identical to the number of queries required for the
attack against QUIC. This attack would allow the attacker to impersonate any
website against any client trusting the CA certificate.

% Nimrod: I wouldn't mention Zyxel, they might get angry and sue us
We did not observe any trusted CA certificates used on vulnerable servers.
We did, however, observe a number of routers that supported \ssltwo
while presenting CA certificates that are untrusted by modern browsers.
\fi

\if0
\subsection{Attacking QUIC using the \tOracleSSLleaky}
The primary obstacle in this attack is the task of "blinding",
i.e. converting a non-\sslconform message $m$ into an \sslconform message $m'$.
This task is made significantly less costly using the following approach,
which the \tOracleSSLleaky enables.

First, the attacker wishes to convert a ciphertext $c$, for which he assumes no knowledge,
into a ciphertext $c'$ that decrypts to a correctly padded plaintext of any length.
Indeed, a ciphertext will decrypt to such a plaintext if:
\begin{equation*} 
	\begin{split} 
		m_1||m_2 \text{ } = &\text{ } \hex{00} || \hex{02}\\
		\hex{00} \text{ } \not \in &\text{ } \{m_3, \ldots,m_{10}\}\\ 
		\hex{00} \text{ } \in &\text{ } \{m_{11}, \ldots,m_{\ell}\}\\ 
	\end{split}
\end{equation*}

For a 2048-bit RSA modulo, the probability of these properties holding for a random message is
$P_{rnd} \approx (1/256)^2 * (255/256)^{8} * (1 - (255/256)^{246}) \approx 2^{-17}$.

Therefore, in order to compute $c^d$, when $c$ is not \sslconform,
the attacker randomly generates values for $s$ and tests
$c \cdot s^{e}$ against the \tOracleSSLleaky.
After roughly $2^{17} \approx 131,000$ queries, he obtains a positive response,
and can deduce that $c^d \cdot s$ starts with bytes \hex{00}{02}.

Na\"{\i}vely, it would seem the attacker can then apply one of the techniques
presented in this work, but $\OracleSSLleaky$ does not provide knowledge of
any least significant plaintext bytes when the plaintext is not of
a length which is at most the correct one.
Instead, he can then proceed directly according to the algorithm presented by
Bardou \etal~\cite{efficient-padding-oracle-2012}.
Referring to Table 1 in~\cite{efficient-padding-oracle-2012},
this oracle is denoted with the term \texttt{FFT},
as it returns a positive response for a correctly padded plaintext of any length,
and the median number of required queries for this oracle is 14,501.
This number of queries is dominated by the 131,000 queries the attacker has already executed.

As for the cost of the attack, executing roughly 131,000 queries would require
only three hours at a rate of 12 queries/second.
The more costly requirement appears to be the offline work, which requires
$ 2^{17} * 2^{40} = 2^{57}$ offline decryption operations.
At our optimized GPU implementation rate of 515 million keys per second,
this would require 3238 GPU days.
Using 40 GPUs, as we did for the implementation run described in Section~\ref{sec:ec2_results}, would reduce the wall clock time for the attack to 81 days.
This version of the attack against QUIC appears to bring the cost of the attack
to within the means of even attackers with a modest budget.
\fi



\section{Measurements}
\label{sec:scans}

We performed Internet-wide scans to analyze the number of systems vulnerable to
DROWN\@. A host is directly vulnerable to general DROWN if it supports \ssltwo.
Similarly, a host is directly vulnerable to special DROWN if it supports
\ssltwo and has the extra clear bug (which also implies the leaky export bug).
These directly vulnerable hosts can be
used as oracles to attack any other host with the same key. Hosts that do not
support \ssltwo are still vulnerable to general or special DROWN if their RSA
key pair is exposed by any general or special DROWN oracle, respectively. The
oracles may be on an entirely different host or port.  Additionally, any host
serving a browser-trusted certificate is vulnerable to a special DROWN
man-in-the-middle if any name on the certificate appears on any other
certificate containing a key that is exposed by a special DROWN oracle.

We used ZMap~\cite{zmap-2013} to perform full IPv4 scans on eight different ports
during late January and February 2016.  We examined port 443 (HTTPS), and
common email ports 25 (SMTP with STARTTLS), 110 (POP3 with STARTTLS), 143 (IMAP
with STARTTLS), 465 (SMTPS), 587 (SMTP with STARTTLS), 993 (IMAPS), and 995
(POP3S).  For each open port, we attempted three complete handshakes: one
normal handshake with the highest available SSL/TLS version; one \ssltwo
handshake requesting an export RC2 cipher suite; and one \ssltwo handshake with
a non-export cipher and sixteen bytes of plaintext key material sent during key
exchange, which we used to detect if a host has the extra clear bug.

We summarize our general DROWN results in Table~\ref{table:general}. The
fraction of SSL/TLS hosts that directly supported \ssltwo varied substantially
across ports. 28\% of SMTP servers on port 25 supported \ssltwo, likely due to
the opportunistic encryption model for email transit. Since SMTP fails-open to
plaintext, many servers are configured with support for the largest possible
set of protocol versions and cipher suites, under the assumption that even bad
or obsolete encryption is better than plaintext~\cite{better-crypto}. The other
email ports ranged from 8\% for SMTPS to 20\% for POP3S and IMAPS. We found
17\% of all HTTPS servers, and 10\% of those with a browser-trusted
certificate, are directly vulnerable to general DROWN\@.

\tabSpecialAll


\subsection{OpenSSL SSLv2 cipher suite selection bug}
\label{sec:openssl-selection}
The \ssltwo protocol is supported in OpenSSL by default in all versions under 1.1.0.
OpenSSL removed \ssltwo cipher suites from the default cipher string in 2010 between versions 0.9.8n and 1.0.0; the changelog discusses this as being equivalent to disabling support for \ssltwo by default~\cite{opensslchangelog}.   Unfortunately, during our experiments we discovered that OpenSSL servers do not respect the cipher suites advertised in the \texttt{ServerHello} message.
That is, the client can select an \textit{arbitrary} cipher suite in the \texttt{ClientMasterKey} message and force the use of export cipher suites even if they are explicitly disabled in the server configuration.
The \ssltwo protocol itself was still enabled by default in the OpenSSL standalone server for the most recent OpenSSL versions prior to our disclosure. %(this was not necessarily the case when OpenSSL was used as a plugin in Apache or other webservers).
%In addition to verifying this vulnerability in our lab, we have encountered several SSLv2 servers on the Internet which have apparently disabled export cipher suites (as judged by their \texttt{ServerHello} message), where we could indeed force the use of these cipher suites on those servers.

We notified the OpenSSL team of this vulnerability, which was assigned CVE ID CVE-2015-3197. We have cooperated to develop a fix, which was included in OpenSSL releases 1.0.2f and 1.0.1r~\cite{opensslchangelog}. 
%In addition, these versions by default disabled \ssltwo support.

%\todo{Mention POP3 is likely vulnerable without any active attacks involving the client, since we expect to have a handshake every few minutes anyway}


\paragraph{Widespread public key reuse.}
Reuse of RSA key material across hosts and certificates is
widespread~\cite{mail-tls-holz-2016,weak-keys-2012}. Often this is benign:
organizations may issue multiple TLS certificates for distinct domains with
the same public key in order to simplify use of TLS acceleration hardware and
load balancing. However, there is also evidence that system administrators
may not entirely understand the role of the public key in certificates. For
example, in the wake of the Heartbleed vulnerability, a substantial fraction
of compromised certificates were reissued with the same public
key~\cite{heartbleed-2014}.

There are many reasons why the same public key or certificate would be reused
across different ports and services within an organization. For example a
mail server that serves SMTP, POP3, and IMAP from the same daemon would
likely share the same TLS configuration. Additionally, an organization might
choose to purchase a single wildcard TLS certificate, and use it on both web
servers and mail servers. Public keys have also been observed to be widely
shared across independent organizations due to default certificates and
public keys that are shipped with networked devices and software, improperly
configured virtual machine images, and random number generation flaws.

The number of hosts vulnerable to DROWN rises significantly when we take RSA
key reuse into account. For HTTPS, 17\% of hosts are vulnerable to general
DROWN because they support both TLS and \ssltwo on the HTTPS port, but 33\%
are vulnerable when considering RSA keys used by another service.

\paragraph{Special DROWN\@.}
As shown in Table~\ref{table:special},
9.1\,M HTTPS servers (26\%) are
vulnerable to special DROWN, as are 2.5\,M HTTPS servers with browser-trusted
certificates~(14\%). 66\% as many HTTPS hosts are vulnerable to special DROWN
as to general DROWN\@ (70\% for browser-trusted servers). While 2.7\,M public
keys are vulnerable to general DROWN, only 1.1\,M are vulnerable to special DROWN
(41\% as many). Vulnerability among Alexa Top Million domains is also lower, with
only 9\% of domains vulnerable (7\% for browser-trusted domains).

Since special DROWN enables active man-in-the-middle attacks, any host serving
a browser-trusted certificate with at least one name that appears on any
certificate with an RSA key exposed by a special DROWN oracle is vulnerable to an
impersonation attack. Extending our search to account for certificates with
shared names, we find that 3.8\,M~(22\%) hosts with browser-trusted certificates
are vulnerable to man-in-the-middle attacks, as well as 19\% of the
browser-trusted domains in the Alexa Top Million.


\section{Related work}
TLS has had a long history of implementation flaws and protocol attacks~\cite{POODLE,CRIME,RC4biases,Lucky13,BEAST,SLOTH,Durumeric:2014:MH:2663716.2663755}. We discuss relevant Bleichenbacher and cross-protocol attacks below.

\paragraph{Bleichenbacher's attack.}
Bleichenbacher's adaptive chosen ciphertext attack against SSL was first published in 1998~\cite{Bleichenbacher}. Several works have adapted his attack to different scenarios~\cite{klima2003attacking,bardou2012efficient,Jager2012}.
The TLS standard explicitly introduces countermeasures against the attack~\cite{rfc5246}, but several modern implementations have been discovered to be vulnerable to timing-attack variants in recent years~\cite{Meyer14,Zhang:2014:CSA:2660267.2660356}. These side-channel attacks are implementation failures and only apply when the attacker is co-located with the victim.

\ifext
Klima \etal~\cite{klima2003attacking} extended the attack to take advantage of leaked protocol version numbers present in the decrypted plaintext, rather than the validity of the padding format.  Bardou \etal~\cite{bardou2012efficient} applied the attack to several cryptographic hardware implementations, and developed the concept of ``trimmers" to aid the mathematical algorithm behind the attack, which we also use in this work.
\fi

\if0
Meyer \etal~\cite{Meyer14} inspected various software and hardware implementations and discovered timing side-channels that enabled the attack. Zhang  \etal~applied Bleichenbacher's attack to develop a cache flush-and-reload timing attack against OpenSSL in cross-tenant environments~\cite{Zhang:2014:CSA:2660267.2660356}. These side-channel attacks, however, are applicable only in scenarios where the attacker is physically close to or co-located with the victim and are based on implementation failures.

Jager et al.\@ described a similar Bleichenbacher oracle, as we use in our paper, to attack XML Encryption in Web Services~\cite{Jager2012}. To this end, they exploited the fact that RSA~PKCS\#1~v1.5 was used in combination with symmetric algorithms in CBC mode of operation.
\fi

%Very recently, it was practically shown that it is still possible to construct \PKCS oracles based on different side-channels in well-used TLS libraries. At USENIX Security 2014, Meyer et al. showed that tiny timing differences can be used to decrypt TLS connections~\cite{Meyer14}. At CCS 2014, Zhang et al. evaluated application of flush-and-reload attacks to decrypt RSA \PKCS ciphertexts~\cite{Zhang:2014:CSA:2660267.2660356}. However, these two techniques are only possible if the analyzed TLS library \textit{implements the countermeasure incorrectly}, and if the attacker can execute the attacks \textit{from a near server distance}: either from a LAN~\cite{Meyer14} or even from the same physical machine~\cite{Zhang:2014:CSA:2660267.2660356}. In addition, these two side-channels can lead to wrong oracle responses, which could break the attack execution~\cite{Meyer14}.

\paragraph{Cross-protocol attacks.}
Jager et al.\@ \cite{Jager:2015:STQ:2810103.2813657} showed that a cross-protocol
Bleichenbacher RSA padding oracle attack is possible against the proposed TLS
1.3 standard, in spite of the fact that TLS 1.3 does not include RSA key
exchange, if server implementations use the same certificate
for previous versions of TLS and TLS 1.3.
Wagner and Schneier~\cite{WagnerSchneier:SSLAnalysis:96} developed a cross-cipher suite attack for
SSLv3, in which an attacker could reuse a signed server
key exchange message in a later exchange with a different
cipher suite.
Mavrogiannopoulos et al.\@ \cite{CCS:MVVP12}
developed a cross-cipher suite attack allowing an attacker to use
elliptic curve Diffie-Hellman as prime field Diffie-Hellman.

\paragraph{Attacks on export-grade cryptography.}
Recently, the FREAK~\cite{SMACKTLS} and Logjam~\cite{LogJam} attacks allowed an
active attacker to downgrade a connection to export-grade RSA and
Diffie-Hellman, respectively.  DROWN exploits export-grade symmetric ciphers,
completing the export-grade cryptography attack trifecta.

\if0
\paragraph{Further attacks on SSL/TLS\@.}
Other attacks on SSL and TLS include:
POODLE~\cite{POODLE}, which exploits SSLv3's lack of a requirement for the contents of padding bytes, and its MAC-then-encrypt construction;
CRIME~\cite{CRIME}, which exploits support for compression and observes ciphertexts' lengths in order to decrypt traffic;
The RC4 Biases attack~\cite{RC4biases}, which utilizes biases in the RC4 keystream;
Lucky13~\cite{Lucky13}, which exploits small timing differences and MAC-then-encrypt;
and BEAST~\cite{BEAST}, which exploits predictable IVs in TLS\@.
 Bhargavan and Leurent presented SLOTH attacks and broke TLS and other protocols using MD5 for computing transcript hashes~\cite{SLOTH}.
\fi



\section{Discussion}
% !TEX root = subgroup.tex

\section{Discussion}

The small subgroup attacks require a number of special conditions to go wrong
in order to be feasible. For the case of small subgroup confinement attacks, a
server must both use a non-safe group and fail to validate subgroup order; the
widespread failure of implementations to implement or enable group order validation
means that large numbers of hosts using non-``safe'' primes are vulnerable to this type of attack.

For a full key recovery attack to be possible the server must additionally
reuse a small static exponent.  In one sense, it is surprising that any
implementations might satisfy all of the requirements for a full key recovery
attack at once.  However, when considering all of the choices that
cryptographic libraries leave to application developers when using
Diffie-Hellman, it is surprising that any protocol implementations manage to
use Diffie-Hellman securely at all. 

We now use our results to draw lessons for the security and cryptographic
communities, provide recommendations for future cryptographic protocols, and
suggest further research.

\paragraph{RFC 5114 Design Rationale} Neither NIST SP 800-56A nor RFC~5114 give
a technical justification for fixing a much smaller subgroup order than the
prime size. Using a shorter private exponent comes with performance benefits.
However, there are no known attacks that would render a short exponent used with
a safe prime less secure than an equivalently-sized exponent used with in a
subgroup with order matched to the exponent length. The cryptanalyses of both short 
exponents and small subgroups are decades old.

If anything, the need to
perform an additional modular exponentiation to validate subgroup order makes
Diffie-Hellman over DSA groups \textit{more} expensive than the safe prime
case, for identical exponent lengths. As a more minor effect, a number field
sieve-based cryptanalytic attack against a DSA prime is computationally
slightly easier than against a safe prime.  The design rationale may have its
roots in preferring to implicitly use the assumption that Diffie-Hellman is
secure for a small prime-order subgroup without conditions on exponent length,
rather than assuming Diffie-Hellman with short exponents is secure inside a
group of much larger order.  Alternatively, this insistence may stem from the
fact that the security of DSA digital signatures requires the secret exponent
to be uniformly random, although no such analogous attacks are known for
Diffie-Hellman key exchange.~\cite{nguyen-2002}
Unfortunately, our empirical results show that the necessity to specify and validate subgroup order for Diffie-Hellman key exchange makes implementations more fragile in practice.

\paragraph{Cryptographic API design}
Most cryptographic libraries are designed with a large number of potential
options and knobs to be tuned, leaving too many security-critical choices to
the developers, who may struggle to remain current with the diverse and
ever-increasing array of cryptographic attacks. These exposed knobs are likely
due to a prioritization of performance over security. These confusing options in
cryptographic implementations are not confined to primitive design: Georgiev et al.~\cite{most-dangerous-code-2012} discovered that SSL certificate validation was broken in a large number of non-browser TLS applications due to developers misunderstanding and misusing library calls.  In the case of the small
subgroup attacks, activating most of the conditions required for the attack
will provide slight performance gains for an application: using a small
exponent decreases the work required for exponentiation, reusing Diffie-Hellman
exponents saves time in key generation, and failing to validate subgroup order
saves another exponentiation. It is not reasonable to assume that applications
developers have enough understanding of algebraic groups to be able to make the
appropriate choices to optimize performance while still providing sufficient
security for their implementation.

\paragraph{Cryptographic standards}
Cryptographic recommendations from standards committees are often too weak or
vague, and, if strayed from, provide little recourse. The purpose of
standardized groups and standardized validation procedures is to help remove
the onus from application developers to know and understand the details of the
cryptographic attacks. A developer should not have to understand the inner
workings of Pollard lambda and the number field sieve in order to size an
exponent; this should be clearly and unambiguously defined in a standard.
However, the tangle of RFCs and standards attempting to define current best practices
in key generation and parameter sizing do not paint a clear picture, and instead 
describe complex combinations of approaches and
parameters, exposing the fragility of the cryptographic ecosystem. As a result,
developers often forget or ignore edge cases, leaving many implementations of
Diffie-Hellman too close to vulnerable for comfort. Rather than provide the
bare minimums for security, the cryptographic recommendations from standards
bodies should be designed for defense-in-depth such that a single mistake on the
part of a developer does not have disastrous consequences for security.  The
principle of defense-in-depth has been a staple of the systems security
community; cryptographic standards should similarly be designed to avoid
fragility.

\paragraph{Protocol design}
The interactions between cryptographic primitives and the needs of protocol designs
can be complex.  The after-the-fact introduction of RFC 5114 primes illustrates some of the
unexpected difficulties: both IKE and SSH specified group validation only for safe primes, and
a further RFC specifying extra group validation checks needed to be defined for IKE.
Designing protocols to encompass many
unnecessary functions, options, and extensions leaves room for implementation
errors and makes security analysis burdensome. 
IKE is a notorious example of a difficult-to-implement protocol with many edge cases. 
Just Fast Keying (JFK), a protocol created as a successor to IKEv1, was designed to be an
exceedingly simple key exchange protocol without the unnecessarily complicated
negotiations present in IKE~\cite{aiello2004just}. However, the IETF instead
standardized IKEv2, which is nearly as complicated as IKEv1. Protocols and
cryptosystems should be designed with the developer in mind\textemdash easy
to implement and verify, with limited edge cases. The worst possible
outcome is a system that appears to work, but provides less security than
expected. 

To construct such cryptosystems, secure-by-default primitives are key. As we
show in this paper, finite-field based Diffie-Hellman has many edge cases that
make its correct use difficult, and which occasionally arise as bugs at the
protocol level. For example, SSH and TLS allow the server to generate arbitrary
group parameters and send them to the client, but provide no mechanism for the
server to specify the group order so that the client can validate the
parameters.  Diffie-Hellman key exchange over groups with different properties
cannot be treated as a black-box primitive at the protocol level.

\paragraph{Recommendations}
As a concrete recommendation, modern Diffie-Hellman implementations should
prefer elliptic curve groups over safe curves with proper point
validation~\cite{safecurves}. These groups are much more efficient and have
shorter key sizes than finite-field Diffie-Hellman at equivalent security
levels. The TLS 1.3 draft includes a list of named curves designed to modern
security standards~\cite{rfc8446}.  If elliptic curve Diffie-Hellman is
not an option, then implementations should follow the guidelines outlined in
RFC~7919 for selecting finite field Diffie-Hellman primes~\cite{rfc7919}.
Specifically, implementations should prefer ``safe'' primes of documented
provenance of at least 2048 bits, validate that key exchange values are
strictly between $1$ and $p-1$, use ephemeral key exchange values for every
connection, and use exponents of at least 224 bits.

%Finite field Diffie-Hellman can be implemented securely, but implementers must
%follow the recommendations of NIST SP 800-56A to the letter.

%\todo{should we recommend ECC? Should we make a stronger argument for phasing out finite-field DH?}
%\todo{we should add a paragraph that has concrete recommendations on how to implement diffie-hellman correctly in terms of sizing exponents}
%Finite-field based Diffie-Hellman has many flaws, and should be phased out of future standards.

%We recommend only using safe primes and that library implementers ensure that exponent lengths are always matched to
%the length of the prime. It is also vital that libraries ensure that fresh exponents are generated with each request for a new key.
%This adds minimal effort for the library implementers but reduces the overal fragility of the cryptosystem and reduces the burden on application developers. Furthermore, if developers introduce errors into their code, the defense-in-depth strategy will ensure that the overall security of the system ins't compromised.
%
%Library implementations at the time of writing, required a careful sequence of options and calls to ensure keys were generated in a safe manner. Implementors should minimize the number of options available to the user, as such options introduce room for error and increase the number of edge cases. Documentation of existing libraries is often dense and users must navigate multiple levels of redirection in order to determineappropriate practices for their case. The documentation provides recommendations that often include instructions on how to trade between security and resource consumption. Provided recommendations instead need to err on the side of safety as users can not be trusted to successfully navigate the details, given the fragility of the systems.  

%- Whit was totally not right.


\ifblind
\else
\section*{Acknowledgements}

The authors thank team Hashcat for making their GPUs available for the execution of the attack,
Ralph Holz for providing early scan data, Adam Langley for insights about QUIC, Graham Steel for insights about TLS False Start, the OpenSSL team for their help with disclosure, Ivan Ristic for comments on session resumption in a BEAST-styled attack, and Tibor Jager and Christian Mainka for further helpful comments. %The authors also would like to thank Sarah Madden for DROWN logo and web site design and Christian Dresen for additional website development.
We thank the exceptional sysadmins at the University of Michigan for their help
and support throughout this project, including Chris Brenner, Kevin Cheek,
Laura Fink, Dan Maletta, Jeff Richardson, Donald Welch, Don Winsor, and others
from ITS, CAEN, and DCO.

This material is based upon work supported by the U.S. National Science Foundation under Grants No.\@ CNS-1345254, CNS-1408734, CNS-1409505, CNS-1505799, CNS-1513671, and CNS-1518888, an AWS Research Education grant, a scholarship from the Israeli Ministry of Science, Technology and Space, a grant from the Blavatnik Interdisciplinary Cyber Research Center (ICRC) at Tel Aviv University, a gift from Cisco, and an Alfred P. Sloan Foundation research fellowship.
%Any opinions, findings, and conclusions or recommendations expressed in this material are those of the authors and do not necessarily reflect the views of the U.S. National Science Foundation. % Removed, since not required after peer review
\fi

{\footnotesize\bibliographystyle{acm}\raggedright
\bibliography{drown,rfc}}


\appendix

\ifext
\section{Public key reuse}
\label{sec:pub_key_reuse}

Reuse of RSA keys among different services was identified as a huge amplification to the number of services vulnerable to DROWN\@. Table~\ref{amount_shared_keys} describes the number of reused RSA keys among different protocols. The two clusters 110-143 and 993-995 stick out as they share the majority of public keys. This is expected, as most of these ports are served by the same IMAP/POP3 daemon. 
The rest of the ports also share a substantial fraction of public keys, usually between 21\% and 87\%. The numbers for HTTPS (port 443) differ as there are four times as many public keys in HTTPS as in the second largest protocol.
\begin{table*}[th]
\centering\footnotesize
 \begin{tabular}{rrrrrrrrr} 
\toprule

\textbf{Port} & \textbf{25 (SMTP)} & \textbf{110 (POP3)} & \textbf{143 (IMAP)} & \textbf{443 (HTTPS)} & \textbf{465 (SMTPS)} & \textbf{587 (SMTP)} & \textbf{993 (IMAPS)} & \textbf{995 (POP3S)}\smallskip\\
\textbf{ 25} &  1,115 (100\%) &           331 (32\%) &       318 (32\%) &       196 (4\%) &        403 (47\%) &       307 (48\%) &       369 (33\%) &       321 (32\%) \\
\textbf{110} &    331 (30\%) &         1,044 (100\%) &      795 (79\%) &       152 (3\%) &        337 (39\%) &       222 (35\%) &       819 (72\%) &       877 (87\%) \\
\textbf{143} &    318 (29\%) &           795 (76\%) &     1,003 (100\%) &      149 (3\%) &        321 (38\%) &       220 (35\%) &       878 (78\%) &       755 (75\%) \\
\textbf{443} &    196 (18\%) &           152 (15\%) &       149 (15\%) &     4,579 (100\%) &      129 (15\%) &        94 (15\%) &       175 (16\%) &       151 (15\%) \\
\textbf{465} &    403 (36\%) &           337 (32\%) &       321 (32\%) &       129 (3\%) &        857 (100\%) &      463 (73\%) &       396 (35\%) &       364 (36\%) \\
\textbf{587} &    307 (28\%) &           222 (21\%) &       220 (22\%) &        94 (2\%) &        463 (54\%) &       637 (100\%) &      259 (23\%) &       229 (23\%) \\
\textbf{993} &    369 (33\%) &           819 (78\%) &       878 (88\%) &       175 (4\%) &        396 (46\%) &       259 (41\%) &     1,131 (100\%) &      859 (85\%) \\
\textbf{995} &    321 (29\%) &           877 (84\%) &       755 (75\%) &       151 (3\%) &        364 (42\%) &       229 (36\%) &       859 (76\%) &     1,010 (100\%) \\

\bottomrule
 \end{tabular}
 \caption{\textbf{Impact of key reuse across ports.} Number of shared public keys among two ports, in thousands.
          Each column states what number and percentage of keys from the port in the header row are used on other ports.
          For example, 18\% of keys used on port 25 are also used on port 443, but only 4\% of keys used on port 443 are also used on port 25. }
 \label{amount_shared_keys}
\end{table*}

% \begin{figure*}[th]
% \small
%  \begin{tabular}{r|rr|rr} 
% \toprule
% \textbf{Port} & \textbf{DROWN}      & \textbf{\dots with Trusted} & \textbf{CVE-2015-3197} & \textbf{\dots with Trusted} \\
%               & \textbf{Handshakes} & \textbf{Certificate}        &                        & \textbf{Certificate}\\
% \midrule
%    25 &    910,585  &   178,726 (20\%) &   256,436 (28\%) &  69,131 (8\%)\\ 
%   110 &    399,105  &   229,727 (58\%) &   308,724 (77\%) & 190,631 (48\%)\\
%   143 &    469,029  &   222,431 (47\%) &   400,646 (85\%) & 199,877 (43\%)\\
%   443 &  5,652,105  & 1,726,373 (31\%) & 1,356,030 (24\%) & 364,134 (6\%)\\
% Alexa &  tba (xx\%) & tba (xx\%)      &   tba     (xx\%) & tba (xx\%)\\
%   465 &    287,431  &    38,831 (14\%) &   176,419 (61\%) &  22,117 (8\%)\\
%   587 &    407,591  &   122,628 (30\%) &   179,048 (44\%) &  59,703 (15\%)\\
%   993 &    846,005  &   258,192 (31\%) &   652,485 (77\%) & 235,895 (28\%)\\
%   995 &    878,553  &   302,775 (35\%) &   664,364 (76\%) & 258,710 (30\%)\\
% \bottomrule
%  \end{tabular}
%  \label{amount_cve_2015_3197}
%  \caption{\textbf{Hosts vulnerable to OpenSSL's cipher suite selection bug (CVE 2015-3197)}}
% \end{figure*}
\fi

\section{Adaptations to Bleichenbacher's attack}
\label{sec:adapted-bb}

\subsection{Calculating the success probability of a fraction}
\label{sec:fraction-probability}

For a given fraction $u/t$, we can compute the probability of success with a randomly chosen \tlsconform ciphertext.  Let $m_1 = m_0 \cdot u/t = m_1[1]||...||m_1[\ell]$ - i.e. $m_1[i]$ is the $i$th byte of $m_1$.  Let $k$ be the fixed byte length of the oracle response.  For $s = u/t \bmod N$ where $u$ and $t$ are coprime, $m_1$ will be \sslconform if the following conditions all hold:
\begin{enumerate}
	\item $m_0$ is divisible by $t$. For randomly generated $m_0$, this condition holds with probability $1/t$.  
		\item $m_1[1] = 0$ and $m_1[2] = 2$, or the integer $m \cdot u/t \in [2B, 3B-1)$.
		For a randomly generated $m_0$ divisible by $t$ and for a given fraction $u/t$, this condition holds with probability
\begin{equation*}
P = 
\begin{cases}
3 - 2 \cdot t/u & \text{for }   2/3 < u/t < 1 \\
3 \cdot t/u - 2 & \text{for }   1 < u/t < 3/2 \\
0 & \text{otherwise}
\end{cases}
\label{eq:oracle}
\end{equation*} 
\item $\forall i \in [3, \ell-(k+1)], m_1[i] \neq 0$, or all bytes between the first two bytes and the $(k+1)$ least significant bytes are non-zero.  This condition holds with probability $(1 - 1/256)^{\ell-(k+3)}$.
\item $m_1[\ell-k] = 0$, or the $(k+1)$st least significant byte is 0.  This condition holds with probability $1/256$.
\end{enumerate}
As an example, let us assume a 2048-bit RSA ciphertext with $k = 5$, and consider the fraction $u = 7, t = 8$.  We have
\begin{align*}
P(t|m_0)= 1/t &= 1/8 \\
P( m_1[1,2] = 00||02 \, \big\vert \, t|m_0) &= 0.71\\
P(\forall i \in [3, \ell-6] \, m_1[i] \neq 0) = (1 - 1/256)^{248} &= 0.37\\
P(m_1[\ell-5] = 0) &= 1/256
\end{align*}
The overall probability of success is $P = 1/8 \cdot 0.71 \cdot 0.37 \cdot 1/256 = 1 / 7,774$; thus we expect to find an \sslconform ciphertext after testing 7,774 randomly chosen \tlsconform ciphertexts.  We can decrease the number of \tlsconform ciphertexts needed by multiplying each candidate ciphertext by several fractions.

\subsection{Optimizing the chosen set of fractions}
\label{sec:fraction-optimization}
%Section~\ref{sec:adapted-bb-compact} already introduced the optimization that allows us to narrow the key space for a single query by observing that, for example, using the fraction $u/t=8/7$ results in the new candidate message $m_1 = m_0 / t \cdot u$ is divisible by $u=8$, and the last three bits of $m_1$ (and thus \texttt{master\_key}) are zero.

In order to deduce the validity of a single ciphertext, the attacker would have to perform a non-trivial brute-force search over all 5 byte \texttt{master\_key} values. This translates into $2^{40}$ encryption operations.

The search space can be reduced by an additional optimization, which relies on the fractional multipliers used in the first step.
Suppose the attacker uses a fraction $u/t=8/7$ to compute a new \sslconform candidate, and suppose that $m_0$ is indeed divisible by $t=7$. 
This implies that the new candidate message $m_1 = m_0 / t \cdot u$ is divisible by $u=8$, and the last three bits of $m_1$ (and thus \texttt{$mk_{secret}$}) are zero. 
This allows the attacker to reduce the searched \texttt{master\_key} space by selecting specific fractions.


More generally, for an integer $u$, the largest power of 2 by which $u$ is divisible, is denoted by $v_2(u)$, and multiplying by a fraction $u/t$ saves us a factor of $v_2(u)$ in the required encryption attempts.
With this observation, the trade-off between the 3 metrics: the required number of intercepted ciphertexts, the required number of queries, and the required number of encryption attempts, becomes non-trivial to analyze.

Therefore, we have resorted to using simulations when evaluating the performance metrics for sets of fractions.
The probability that multiplying a ciphertext by any fraction out of a given set of fractions results in an \sslconform message is difficult to compute, since the events are in fact inter-dependent: If $m \cdot 16/15$ is conforming, then $m$ is divisible by $5$, greatly increasing the probability that $m \cdot 4/5$ is also conforming.
However, it is easy to perform a Monte Carlo simulation, where we randomly generate ciphertexts, and measure the probability that any fraction out of a given set produces a conforming message.
The expected required number of intercepted ciphertexts is the inverse of that probability.

Formally, if we denote the set of fractions as $F$, and the event that a message $m$ is conforming as $C(m)$, we perform a Monte Carlo estimation of the probability
$ P_F = P(\exists f \in F: C(m \cdot f)) $, and the expected number of required intercepted ciphertexts equals $1/{P_F}$.

The required number of oracle queries is simply $ 1/P_F \cdot |F| $: For each ciphertext, we need to query the oracle with each fraction.
Accordingly, the required number connections to the server is $ 2 \cdot 1/P_F \cdot |F| $, since as explained earlier each logical query consists of two connections to the server.

And as for the required number of encryption attempts, if we denote this number when querying with a given fraction $f = u/t$ as $E_f$, then
$E_f = E_{u/t} = 2^{40-v_2(u)}$.
If we further define the required encryption attempts when testing a single ciphertext with each fraction from a given set of fraction $F$ as
$E_F = \sum_{f \in F} E_f$
then the required number of encryption attempts throughout the attack for a given set of fractions is $(1/{P_F}) \cdot E_F$.

Using this approach, we can now give precise figures for the expected number of required intercepted ciphertexts, connections to the targeted server, and encryption attempts.
The results presented in Table~\ref{tab:reasonable_parameters} were obtained by using the monte-carlo estimation technique described above, with one billion random ciphertexts per tested fraction set $F$.

\subsection{Efficiently computing rotations and multipliers}
\label{sec:rotation-details}

For a randomly chosen $s$, the probability that the two most significant bytes are $\hexb{00}{02}$ is $2^{-16}$; for a 2028-bit modulus $N$ the probability that the next $\ell - k - 3$ bytes of $m_2$ are all nonzero is about 0.37 as in the previous section, and the probability that the $k+1$ least significant delimiter byte is $\hex{00}$ is 1/256.  Thus a randomly chosen $s$ will work with probability $2^{-25.4}$ and we expect to need to try $2^{25.4}$ values of $s$ before succeeding.

However, since we have already learned $k+3$ most significant bytes of $m_1 \cdot R^{-1} \bmod N$,
for $k \ge 4$ and $s < 2^{30}$ we do not need to query the oracle to learn if the two most significant bytes are \sslconform; we can compute this ourselves from our knowledge of $\tilde{m_1} \cdot R^{-1}$.  We could simply iterate through values of $s$, test that the top two bytes of $\tilde{m_1} \cdot R^{-1} \bmod N$ are \sslconform, and only query the oracle $\Oracle$ for values of $s$ that satisfy this test; this means that for our 2048-bit modulus we expect to test $2^{16}$ values offline per oracle query.  The probability that our query is conformant is then $P = (1/256) * (255/256)^{249} \approx 1/678$ so we expect to perform 678 oracle queries before finding a fully \sslconform ciphertext $c_2 = (s \cdot R^{-1})^e c_1 \bmod N$.

We can speed up the brute force testing of $2^{16}$ values of $s$ using algebraic lattices.  We are searching for values of $s$ satisfying $\tilde{m_1} R^{-1} s < 3 B \bmod N$, or given an offset $s_0$ we would like to find solutions $x$ and $z$ to the equation $\tilde{m_1} R^{-1} (s_0 + x) = 2 B + z \bmod N$ where $|x| < 2^{16}$ and $|z| < B$.  Let $X =  2^{15}$.  We can construct the lattice basis
\[
L = 
\begin{bmatrix}
-B & X\tilde{m_1} R^{-1} & \tilde{m_1} R^{-1} s_0 + B \\
0 & XN & 0 \\
0 & 0 & N
\end{bmatrix}
\]
We then run the LLL algorithm~\cite{lll} on $L$ to obtain a reduced lattice basis $V$ containing vectors $v_1, v_2, v_3$.  We then construct the linear equations $f_1(x,z) = v_{1,1}/B \cdot z +v_{1,2}/X \cdot x + v_{1,3} = 0$ and $f_2(x,z) = v_{2,1}/B \cdot z +v_{2,2}/X \cdot x + v_{2,3} = 0$ and solve the system of equations to find a candidate integer solution $x = \tilde{s}$.  We then test $s = \tilde{s} + s_0$ as our candidate solution in this range.

$\det L = XZN^2$ and $\dim L = 3$, thus we expect the vectors $v_i$ in $V$ to have length approximately $|v_i| \approx (XZN^2)^{1/3}$.  We will succeed if $|v_i| < N$, or in other words $XZ < N$.  $N \approx 2^{8\ell}$, so we expect to find short enough vectors. This approach works well in practice and is significantly faster than iterating through $2^{16}$ possible values of $\tilde{s}$ for each query.

In summary, given an \sslconform ciphertext $c_1 = m_1^e \bmod N$, we can efficiently generate an \sslconform ciphertext $c_2 = m_2^e \bmod N$ where $m_2 = s \cdot m_1 \cdot R^{-1} \bmod N$ and we know several most significant bytes of $m_2$, using only a few hundred oracle queries in expectation.  We can iterate this process as many times as we like to continue generating \sslconform ciphertexts $c_i$ for which we know increasing numbers of most significant bytes, and which have a known multiplicative relationship to our original message $c_0$. 

\subsection{Rotations in the general DROWN attack}
\label{sec:general-rotations}
After the first phase, we have learned an \sslconform ciphertext $c_1$, and we wish to shift known plaintext bytes from least to most significant bits.
Since we learn the least significant 6 bytes of plaintext of $m_1$ from a successful oracle $\OracleSSLexp$ query, we could use a shift of $2^{-48}$ to transfer 48 bits of known plaintext to the most significant bits of a new ciphertext.  However, we perform a slight optimization here, to reduce the number of encryption attempts.  We instead use a shift of $2^{-40}$, so that the least significant byte of $m_1 \cdot 2^{-40}$ and $\tilde{m_1} \cdot 2^{-40}$ will be known.  This means that we can compute the least significant byte of $m_1 \cdot 2^{-40} \cdot s \bmod N$, so oracle queries now only require $2^{32}$ encryption attempts each. This brings the total expected number of encryption attempts for this phase to
$2^{32} * 678 \approx 2^{41}$.

We perform two such plaintext shifts in order to obtain an \sslconform message, $m_3$ that resides in a narrow interval of length at most $2^{8\ell-66}$. Then we can then obtain a multiplier $s_3$ such that $m_3 \cdot s_3$ is also \sslconform.
Since $m_3$ lies in an interval of length is at most $2^{8\ell-66}$, with high probability for any $s_3 < 2^{30}$, $m_3 \cdot s_3$ lies in an interval whose length is at most $2^{8\ell-36} < B$, so we know the two most significant bytes of $m_3 \cdot s_3$.
Furthermore, we know the exact value of the 6 least significant bytes even after multiplication.
So we test possible values of $s_3$, and for values such that $m_3 \cdot s_3$ starts with the required 00 02 bytes, and the 6th least significant byte is zero, we query the oracle as to the validity of $c_3 \cdot s_3^e \bmod N$.
The only condition for PKCS conformance which we haven't verified before querying the oracle is
\[
\forall i \in [3, \ell-6], (m_3 \cdot s_3)[i] \neq 0
\]
which holds with probability 0.37.
So after roughly $1 / 0.37 = 2.72$ queries, we expect to get a positive answer from the oracle.

Since we know the value of the 6 least significant bytes after multiplication, there's no component of breaking a symmetric cipher here - if the message is \sslconform after multiplication, we know the symmetric key, and can test whether it fits the received \texttt{ServerVerify} message.

\subsection{General DROWN Bleichenbacher iterations}
\label{sec:general-bleichenbacher}
After we have bootstrapped the attack using rotations,, the original algorithm proposed by Bleichenbacher can be applied with minimal modifications.

The original step obtains a message that starts with the required 00 02 bytes once in roughly every two queries on average, and requires the number of queries to be roughly double the number of bits in the RSA modulus.
Since we know the value of the 6 least significant bytes after multiplying by any integer, we can only query the oracle for multipliers that cause the 6th least significant byte to be zero, and we don't need to break a symmetric key since we know the value of the 5 least significant bytes.
However, we cannot ensure that the padding is non-zero when querying---we simply hope that is the case, which as usual happens with probability 0.37.

Therefore, for a 2048-bit modulus, the overall expected number of queries for this phase is roughly $2048 * 2 / 0.37 = 11,000$. This is indeed the average number of queries we require in practice when running our implementation of the attack.

\subsection{General DROWN attack performance}
\label{sec:general-performance}

For a given set of fractions, $F$,
the required number of recorded client connections $A$ is a random variable distributed geometrically with a success probability $P = P_F$.
For typical fraction sets, $1/13,000 < P_F < 1/600$.
The required number of Bleichenbacher queries against the target server during the first step of the attack is a random variable, $B$, such that $B = |F| \cdot A$.
As each query consists of two separate connections to the target server, the required number of connections is always  twice the number of queries.
And last, the required keys to be tested overall is another random variable $C = k_F \cdot B; k_F \approx 2^{40}$.

Summing the figures from the different phases for a 2048-bit RSA modulus, the attack requires in expectation $13,838 + 1,393 + 1,393 + 6 + 22,140 = 38,770$ connections to the target server, when optimizing for the number of queries in phase 1.  Each oracle query requires two connections to the server.

Re-calculating the numbers for a 1024 bit modulus, the primary element that needs to change is $P_1 = P(\forall i \in [3, \ell-6]: m_i \neq 0) = (1 - 1/256)^{120} = 0.62$, which appears in phases 1, 2, 3 and 5. For phase 5, the number of queries is now in expectation $1024 * 2 / 0.62 = 3,303$. The total expected number of server connections is therefore $8,258 + 826 + 826 + 6 + 6,606 = 16,522$, again when optimizing for the number of queries in phase 1.

Similarly, re-calculating the numbers for a 4096 bit modulus, $P_1 = (1 - 1/256)^{504} = 0.14$, and the number of queries in phase 5 is now roughly $4096 * 2 / 0.14 = 58,514$. The algorithm for phase 5 can be further optimized if that is the case of interest; we omit these optimizations for space reasons. Again, summing up yields $36,571 + 3,657 + 3,657 + 29 + 117,028 = 160,942$ required connections to the server.

\balance

\subsection{Special DROWN attack performance}
\label{sec:special-performance}

In the first step, we can use the same fraction analysis as before. The probability that the three padding bytes are correct remains unchanged. The probability that all the intermediate padding bytes are non-zero is now slightly higher, $P_1 = (1 - 1/256)^{229} = 0.41$, yielding an overall maximal success probability $P = 0.1 \cdot 0.41 \cdot \frac{1}{256} = 1/6,244$ per oracle query. Since we now only need to connect to the server once per oracle query, the expected number of connections in this step is the same, $6,243$. Phase~1 now yields a message with 3 known padding bytes and 24 known plaintext bytes.

For the remaining rotation steps, each rotation requires an expected 630 oracle queries.  The attacker at this point could directly complete the original Bleichenbacher attack by performing 11,000 sequential queries in the final phase.  However, with this more powerful oracle it is more efficient for the attacker to apply a rotation 10 more times to recover the remaining bits of the plaintext. The number of queries required in this phase is now $10\cdot 256/0.41\approx 6,300$, and the queries for each of the 10 steps can be executed in parallel.

\paragraph{Using multiple queries per fraction.}
For the $\OracleSSLclear$ oracle, the attacker can increase his chances of success by querying the server multiple times per
ciphertext and fraction, using different cipher suites with different key lengths. He can query DES and hope
the 9th least significant byte is zero, then negotiate 128-bit RC4
and hope the 17th least significant byte is zero, then negotiate
3DES and hope the 25th least significant is zero. All three queries also require
the intermediate padding bytes to be non-zero. This technique
triples the success probability for a given pair of (ciphertext, fraction),
at a cost of triple the queries. Its primary benefit is that fractions with smaller
denominators (and thus higher probabilities of success) are now even more likely
to succeed.

For a random ciphertext, when choosing 70 fractions, the probability of the
first zero delimiter byte being in one of these three positions is 0.01.
% Nimrod: literally 0.01004702047755441
Hence, the attacker can use only 100 recorded ciphertexts, and expect to use
$100 * 70 * 3 = 21,000$ oracle queries. For the \tOracleSSLclear, each
query requires one SSLv2 connection to the server.
After obtaining the first positive response from the oracle, the attacker
proceeds to phase 2 using 3DES.


%\balance

\ifext
\section{Highly optimized GPU implementation}
\label{sec:gpu}

\GPUTable

The most computationally expensive part of our general DROWN attack is breaking the 40-bit symmetric key.  We wanted to find the platform that would have the best tradeoff of cost and speed for the attack, so we performed some preliminary experiments comparing performance of symmetric key breaking on CPUs, GPUs, and FPGAs.  These experiments used a na\"{\i}ve version of the attack using the OpenSSL implementation of MD5 and RC2.

The CPU machine contained four Intel Xeon E7-4820 CPUs with a total of 32 cores (64 concurrent threads). The GPU system was equipped with a ZOTAC GeForce GTX TITAN and an Intel Xeon E5-1620 host CPU\@. The FPGA setup consisted of 64 Spartan-6 LX150 FPGAs.

We benchmarked the performance of the CPU and GPU implementations over a large corpus of randomly generated keys, and then extrapolated to the full attack.
For the FPGAs, we tested the functionality in simulation and estimated the actual runtime by theoretically filling the FPGA up to 90\% with the design, including communication.
Table~\ref{perf_comparison} compares the three platforms.

While the FPGA implementation was the fastest in our test setup, the speed-to-cost ratio of GPUs was the most promising. Therefore, we decided to focus on optimizing the attack on the GPU platform.
We developed several optimizations:

\paragraph{Generating key candidates on GPUs.} Our na\"{\i}ve implementation generated key candidates on the CPUs. For each \hashcomputation, a key candidate was transmitted to the GPU, and the GPU responded with the key validity. The bottleneck in this approach was the PCI-E Bus. Even newer boards with PCI-E 3.0 or even PCI-E 4.0 are too slow to handle the large amount of data required to keep the GPUs busy. We solved this problem by generating the key candidates directly on the GPUs.
	
\paragraph{Generating memory blocks of keys.}
Our hash computation kernel had to access different candidate keys from the GPU memory. Accessing global memory is typically a slow operation and we needed to keep memory access as minimal as possible. Ideally we would be able to access the candidate keys on a register level or from a constant memory block, which is almost as fast as a register. However, there are not enough registers or constant memory available to store all the key values. 

We decided to divide each key value into two parts $k_H$ and $k_L$, where $|k_H|=1$ byte and $|k_L|=4$ bytes. We stored all possible $2^8$ $k_H$ values in the constant read-only memory, and all possible $2^{32}$  $k_L$ values in the global memory. 
Next we used an in-kernel loop. We loaded the latter 4 bytes from the slow global memory and stored it in registers. Inside the inner loop we iterated through our first byte $k_H$ by accessing the fast constant memory. The resulting key candidate was computed as $k=k_H||k_L$. 
	
\paragraph{Using 32-bit data types.}
Although modern GPUs support several data types ranging in size from 8 to 64 bits, many instructions are designed for 32-bit data types. This fits the design of MD5 perfectly, because it uses 32-bit data types. RC2, however, uses both 8-bit and 16-bit data types, which are not suitable for 32-bit instruction sets. This forced us to rewrite the original RC2 algorithm to use 32-bit instructions.
% to prevent casting from different datatypes which would typically involve a additional bitfield extraction call.
	
\paragraph{Avoiding loop branches.} Our kernel has to concatenate several inputs to generate the \texttt{server\_write\_key} needed for the encryption as described in Section~\ref{sec:ssl2}. Using loops to move this data generates branches because there is always an if() inside a for() loop. To avoid these branches, which always slow down a GPU implementation, we manually shifted the input bytes into the 32-bit registers for MD5. This was possible since the \hashcomputation inputs,
$(mk_{clear} || mk_{secret} || ``0" || r_c || r_s)$,
have constant length.

\paragraph{Optimizing MD5 computation.} Our MD5 inputs have known input length and block structure, allowing us to use the so-called zero-based optimizations.  Given the known input length (49 bytes) and the fact that MD5 uses zero padding, in our case the MD5 input block included four 0x00 bytes. These \hex{00} bytes are read four times per MD5 computation which allowed us to drop in total 16 ADD operations per MD5 computation. In addition, we applied the Initial-step optimizations used in the Hashcat implementation~\cite{hashcat-talk}.

\paragraph{Skipping the second encryption block.} The input of the brute-force computation is a 16-byte client challenge $r_c$ and the resulting ciphertext from the \texttt{ServerVerify} message which is computed with an RC2 cipher. As RC2 is an 8-byte block cipher the RC2 input is split into two blocks and two RC2 encryptions are performed. In our verification algorithm, we skipped the second decryption step as soon as we saw the key candidate does not decrypt the first plaintext block correctly. This resulted in a speedup of about a factor of 1.5.

\paragraph{RC2 permutation table in constant memory.}
The RC2 algorithm uses a 256-byte permutation table which is constant for all RC2 computations. Hence, this table is a good candidate to be put into the constant memory, which is nearly as fast as registers and makes it easy to address the table elements. When finally using the values, we copied them into the even faster shared memory. Although this copy operation has to be repeated, it still led to a speed up of approximately a factor of 2.

\paragraph{RC2 key setup without keysize checks.}
The key used for RC2 encryption is generated using MD5, thus the key size is always 128 bits. Therefore, we do not have to check for the input key size, and can simply skip the size verification branch completely.


\fi

\ifext
\section{Amazon EC2 evaluation}
%\section{Brute Forcing Keys with Amazon EC2} 
\label{sec:ec2-details}

%Amazon Elastic Cloud Compute (EC2)~\cite{ec2} is a service that provides on-demand virtualized compute resources to customers. 
%This is an affordable alternative to provisioning one's own local cluster.

Amazon EC2 billing is based on the \textit{instance-hour}. An \textit{instance} represents a single virtualized machine and its associated cores, memory, and storage. For our experiments we used \texttt{g2} instances, which are equipped with high-performance NVIDIA GPUs, each with 1,536 CUDA cores. The two available models for this instance type are the \texttt{g2.2xlarge} and the \texttt{g2.8xlarge}, containing one and four GPUs, respectively.

It is possible to request instances at a fixed on-demand rate, or bid on instances at the discounted spot instance rate. Spot instances may be terminated depending on demand, but the savings in cost are significant compared to the on-demand rate. 
When we ran our experiments in January 2016, the on-demand rate for the \texttt{g2.2xlarge} model was \$0.65/hr and the rate for the \texttt{g2.8xlarge} model was \$2.65/hr, while the average spot rates we paid were \$0.09/hr and \$0.83/hr respectively.

We used a cluster composed of 200 spot instances: 150 \texttt{g2.2xlarge} which contain one GPU and 50 \texttt{g2.8xlarge}, each containing four GPUs, spread across multiple availability zones within the US-East region.
This distribution was determined by price: we were not able to launch more than 50 \texttt{g2.8xlarge} instances without a sharp spike in spot prices. We used the optimized Hashcat implementation on the same workload of key requests as the experiments run on the Hashcat servers.  We used Slurm~\cite{yoo2003slurm} to distribute jobs across compute nodes.

The GPU breaking experiment completed successfully, with two minor caveats. First, the 150 \texttt{g2.2xlarge} nodes completed their workloads at the 6h26m mark, while the other 50 \texttt{g2.8xlarge} nodes did not finish until the 7h41m mark. More careful job distribution would ensure that all nodes completed at approximately the same time, reducing the overall runtime. Second, in this particular run, $7.2\%$ of the jobs that we expected to complete were terminated early due to overheating GPUs.  The attack was successful despite the failed jobs, so we did not rerun them. In a more carefully engineered implementation, the unfinished jobs could have been reallocated to the unused GPU capacity without increasing the overall runtime.

The total cost of the experiment was \$440, and terminated in under 8 hours including startup and shutdown.


\fi

\ifext
\section{A brief history of obsolete cryptography}
% !TEX root = ../../../proposal.tex
A flaw was first observed in the MD5 hash function in 1996; the first
collision was discovered in 2004~\cite{Wang:2005:BMO:2154598.2154601}, but
MD5 was still in use by certificate authorities in 2009 when Stevens et
al.~\cite{Stevens2009} used a chosen-prefix MD5 attack to construct a
malicious TLS certificate with a valid CA signature. The RC4 stream cipher
was observed to be biased as early as 1995 and shown to be catastrophically
broken in the context of WEP in 2001~\cite{Fluhrer2001}; it was used by about
50\% of TLS connections in 2013 when AlFardan et~al.\ demonstrated
near-practical attacks against RC4 in TLS~\cite{RC4biases}. TLSv1.0 was
standardized in 1998 to replace SSLv3; before the POODLE attack~\cite{POODLE}
was shown to render all SSLv3 block cipher suites insecure in 2014, support
for SSLv3 was near 100\% for popular HTTPS sites, and most clients were
vulnerable to a downgrade attack from TLS to SSLv3~\cite{ssllabs}.
Export-grade cipher suites for TLS have been obsolete since 2000, when the
United States relaxed restrictions on commercial and open source software;
before the FREAK attack~\cite{SMACKTLS} demonstrated widespread
implementation flaws allowing a catastrophic downgrade attack exploiting
export RSA, 37\% of HTTPS sites with browser-trusted certificates supported
export-grade RSA\@. Three months later the Logjam attack~\cite{logjam-2015}
demonstrated a TLS protocol flaw downgrade attack exploiting export
Diffie-Hellman; 8.4\% of the Alexa top million sites were vulnerable at the
time.


\fi

\end{document}




\chapter{Conclusion and Future Work}
\label{chapter:conclusion}

\textit{This is an outline}

List of how the empirical methods discussed earlier had impact
(pull-out relevant results).

Did any of this research go anywhere beyond that? Yes, it had a clear impact
on TLS 1.3.

\section{TLS 1.3}

Point out that the ``informing future protocol design'' is effectively
``making TLS 1.3 better''. This is the ``improve the security of the Internet
in the future'' part.

Call out explicit TLS 1.3 design decisions based on results described in this
dissertation. Perhaps some of this should be interleaved into the earlier
sections?

\section{Engineering Challenges}

For not being a ``systems'' field, there is still an absurd amount of
engineering that largely goes unacknowledged, in order to write good
measurement papers. Discuss some of these examples.

Is this a fundamental state of being of the methodology, or are we doing something wrong?

\section{Broader applicability of empirical methods}

Security driven by data. Is any of this relevant to users who are just trying
to secure this own networks?

Extending measurement from aggregations and ecosystems to behavior of
individual hosts at global scale. Can we track individual hosts appearing and
disappearing, and map changes to configuration in real-time?

Should empiricism be part of more traditional cryptography education and research?

Some word of warning about how measurement doesn't solve all our problems,
and how measuring the wrong things makes things worse, \eg Robert McNamera's
use of body count as a metric during the Vietnam War.

% What the fuck does anything have to do with Vietnam?

\bibliographystyle{plain}
\bibliography{refs}

\end{document}
