\begin{landscape}
\begin{table*}[t]
\small
\centering
\begin{tabularx}{\linewidth}{lrrrr}
\toprule
              & \multicolumn{4}{c}{\emph{Vulnerable servers, if the attacker can precompute for \ldots}} \\
\cmidrule{2-5}
%\multicolumn{1}{r}{\emph{If the attacker can precompute \ldots.}}
              & all 512-bit groups & all 768-bit groups & one 1024-bit group & ten 1024-bit groups\\  
\midrule
HTTPS Top~1M w/ active downgrade\qquad\strut & 45,100 (8.4\%) & 45,100 (8.4\%) & 205,000 (37.1\%) & 309,000 (56.1\%) \\
HTTPS Top~1M & 118 (0.0\%) & 407 (0.1\%) & 98,500 (17.9\%) & 132,000 (24.0\%) \\
HTTPS Trusted w/ active downgrade & 489,000 (3.4\%) & 556,000 (3.9\%) & 1,840,000 (12.8\%) & 3,410,000 (23.8\%) \\ 
HTTPS Trusted &  1,000 (0.0\%) & 46,700 (0.3\%) & 939,000 (6.56\%) & 1,430,000 (10.0\%)\medskip\\
IKEv1 IPv4  & -- & 64,700 (2.6\%) & 1,690,000 (66.1\%)& 1,690,000 (66.1\%) \\
IKEv2 IPv4  & -- & 66,000 (5.8\%) & 726,000 (63.9\%) & 726,000 (63.9\%)\medskip \\
SSH   IPv4  & -- & -- & 3,600,000 (25.7\%) & 3,600,000 (25.7\%) \\
\bottomrule

\end{tabularx}
\caption{\textbf{Estimated impact of Diffie-Hellman attacks in early 2015}\,---\,%
We used Internet-wide scanning to estimate the number of real-world servers for which 
typical connections could be compromised by attackers with various levels of computational
resources. For HTTPS, we provide figures with and without downgrade attacks on the chosen ciphersuite. All others are passive attacks.
}
\end{table*}
\end{landscape}
