% !TEX root = ../../../proposal.tex

\section{IPsec}

%Description of IKEv1 and IKEv2.

IPsec is a set of Layer-3 protocols which add confidentiality, data protection,
sender authentication, and access control to IP traffic. IPsec is commonly used
to implement VPNs.
%IPsec provides two types of security service: Authentication Header (AH),
%which provides sender authentication, and Encapsulating Security Payload
%(ESP), which provides both sender authentication and payload encryption.  Each
%of these services requires the communicating parties to establish shared
%state, which includes the cryptographic algorithms used to provide the service
%and the keys used as input for the cryptographic algorithms.
IPsec uses the Internet Key Exchange (IKE) protocol to determine the keys used
to secure a session. IPsec may use IKEv1~\cite{rfc2409} or
IKEv2~\cite{rfc7296}. While IKEv2 is not backwards-compatible with IKEv1, the
two protocols are similar in message structure and purpose. Both versions use
Diffie-Hellman to negotiate shared secrets. The groups used are limited to a
fixed set of pre-determined choices, which include the DSA groups from
RFC~5114, each assigned a number by IANA~\cite{rfc3526,rfc5114,rfc7296}.

% IKE versions
\paragraph{IKEv1}
%IKEv1 is a hybrid protocol built upon three other protocols:
%ISAKMP~\cite{rfc2408}, which establishes a framework for authentication and
%key exchange; Oakley~\cite{rfc2412}, which defines a series of key exchanges
%and services based on the Diffie-Hellman key exchange; and
%SKEME~\cite{krawczyk1996skeme}, a key exchange protocol that provides
%anonymity, repudiability, and quick key refreshement.  IKEv1 is formally
%defined in RFCs 2407~\cite{rfc2407}, 2408~\cite{rfc2408}, and
%2409~\cite{rfc2409}. 
IKEv1~\cite{rfc2407,rfc2408,rfc2409} has two basic methods for authenticated
key exchange: Main Mode and Aggressive Mode. Main Mode requires six messages to
establish the requisite state. The initiator sends a Security Association
(\texttt{SA}) payload, containing a selection of cipher suites and
Diffie-Hellman groups they are willing to negotiate. The responder selects a
cipher and responds with its own \texttt{SA} payload. After the cipher suite is
selected, the initiator and responder both transmit Key Exchange (\texttt{KE})
payloads containing public Diffie-Hellman values for the chosen group. At this
point, both parties compute shared key materials, denoted \texttt{SKEYID}. When
using signatures for authentication, \texttt{SKEYID} is computed
$\texttt{SKEYID} = \operatorname{prf}(N_i | N_r, g^{x_ix_r})$.  For the other
two authentication modes, pre-shared key and public-key encryption,
\texttt{SKEYID} is derived from the pre-shared key and session cookies,
respectively, and does not depend on the negotiated Diffie-Hellman shared
secret.

Each party then in turn sends an authentication message (\texttt{AUTH}) derived
from a hash over \texttt{SKEYID} and the handshake. The authentication messages
are encrypted and authenticated using keys derived from the Diffie-Hellman
secret $g^{x_i x_r}$.  The responder only sends her \texttt{AUTH} message after
receiving and validating the initiator's \texttt{AUTH} message.

Aggressive Mode operates identically to Main Mode, but in order to reduce
latency, the initiator sends \texttt{SA} and \texttt{KE} messages together, and
the responder replies with its \texttt{SA}, \texttt{KE}, and \texttt{AUTH}
messages together. In aggressive mode, the responder sends an authentication
message first, and the authentication messages are not encrypted.


\paragraph{IKEv2}
%IKE Version 2 (IKEv2) was released in RFC 4306~\cite{rfc4306} to replace IKEv1
%and the plethora of RFCs that define it. RFC 7296~\cite{rfc7296} gives the
%current version of the IKEv2 specification.
IKEv2~\cite{rfc4306,rfc7296} combines the \texttt{SA} and \texttt{KE} messages
into a single message. The initiator provides a best guess ciphersuite for the
\texttt{KE} message. If the responder accepts that proposal and chooses not to
renegotiate, the responder replies with a single message containing both
\texttt{SA} and \texttt{KE} payloads. Both parties then send and verify
\texttt{AUTH} messages, starting with the initiator.  The authentication
messages are encrypted using session keys derived from the \texttt{SKEYSEED}
value which is derived from the negotiated Diffie-Hellman shared secret. The
standard authentication modes use public-key signatures over the handshake
values.

%IKE Group~23, the 2048-bit MODP group with a 224-bit subgroup, is particularly
%vulnerable as shown in Section~\ref{sec:ecm}.

\subsection{Small Subgroup Attacks in IPsec} There are several variants of
small subgroup attacks against IKEv1 and IKEv2.  We describe the attacks
against these protocols together in this section.

\paragraph{Small subgroup confinement attacks} First, consider attacks that can
be carried out by an attacking initiator or responder. In IKEv1 Main Mode and
in IKEv2, either peer can carry out a small subgroup confinement attack against
the other by sending a generator of a small subgroup as its key exchange value.
The attacking peer must then guess the other peer's view of the Diffie-Hellman
shared secret to compute the session keys to encrypt its authentication
message, leading to a mostly online attack. However, in IKEv1 Aggressive Mode,
the responder sends its \texttt{AUTH} message before the initiator, and this
value is not encrypted with a session key. If signature authentication is being
used, the \texttt{SKEYID} and resulting hashes are derived from the
Diffie-Hellman shared secret, so the initiator can perform an offline
brute-force attack against the responder's authentication message to learn
their exponent in the small subgroup.

Now, consider a man-in-the-middle attacker. Bhargavan, Delignat-Lavaud, and
Pironti~\cite{bhargavan-channel-bindings-2015} describe a transcript synchronization
attack against IKEv2 that relies on a small subgroup confinement attack.  A
man-in-the-middle attacker initiates simultaneous connections with an initiator
and a responder using identical nonces, and sends a generator $g_i$ for a
subgroup of small order $q_i$ to each as its \texttt{KE} message.  The two
sides have a $1/q_i$ chance of negotiating an identical shared secret, so an
authentication method depending only on nonces and shared secrets could be
forwarded, and the session keys would be identical.

If the attacker also has knowledge of the secrets used for authentication, more
attacks are possible.  Similar to the attack described for TLS, such an
attacker can use a small subgroup confinement attack to force a connection to
use weak encryption. The attacker only needs to rewrite a small number of
handshake messages; any further encrypted communications can then be decrypted
at leisure without requiring the man-in-the-middle attacker to continuously
rewrite the connection. We consider a man-in-the-middle attacker who modifies
the key exchange message from both the initiator and the responder to
substitute a generator $g_i$ of a subgroup of small order $q_i$.  The attacker
must then replace the handshake authentication messages, which would require
knowledge of the long-term authentication secret.  We describe this attack for
each of pre-shared key, signatures, and public-key authentication. 

For pre-shared key authentication in IKEv1 Main Mode, IKEv1 Aggressive Mode,
and IKEv2, the man-in-the-middle attacker must only know the pre-shared key to
construct the authentication hash; the authentication message does not depend
on the negotiated Diffie-Hellman shared secret. With probability $1/q_i$, the
two parties will agree on the Diffie-Hellman shared secret. The attacker can
then brute force this value after viewing messages encrypted with keys derived
from it.

For signature authentication in IKEv1 Main Mode and in IKEv2, the signed hash
transmitted from each side is derived from the nonces and the negotiated shared
secret, which is confined to one of $q_i$ possible values.  The attacker must
know the private signing keys for both initiator and responder and brute force
\texttt{SKEYID} from the received signature in order to forge the modified
authentication signatures on each side. The communicating parties will have a
$q_i$ chance of agreeing on the same value for the shared secret to allow the
attack to succeed. For IKEv1 Aggressive Mode, the attack can be made to succeed
every time. The responder's key exchange message is sent together with their
signature which depends on the negotiated shared secret, so the
man-in-the-middle attacker can brute force the $q_i$ possible values of the
responders private key $x_r$ and replace the responder's key exchange message
with $q_i^{x_r}$, forging an appropriate signature with their knowledge of the
signing key.

For public key authentication in IKEv1 Main Mode, IKEv1 Aggressive Mode, and
IKEv2, the attacker must know the private keys corresponding to the public keys
used to encrypt the ID and nonce values on both sides in order to forge a valid
authentication hash.  Since the authentication does not depend on the shared
Diffie-Hellman negotiated value, a man-in-the-middle attacker must then brute
force the negotiated shared key once they receives a message encrypted with the
derived key.  The two parties will agree on their view of the shared key with
probability $1/q_i$, allowing the attack to succeed.

\paragraph{Small subgroup key recovery attacks} Similar to TLS, an IKE
responder that reuses private exponents and does not verify that the initiator
key exchange values are in the correct subgroup is vulnerable to a small
subgroup key recovery attack. The most recent version of the IKEv2
specification has a section discussing reuse of Diffie-Hellman exponents,
and states that ``because computing Diffie-Hellman exponentials is
computationally expensive, an endpoint may find it advantageous to reuse those
exponentials for multiple connection setups''~\cite{rfc7296}. Following this
recommendation could leave a host open to a key recovery attack, depending on
how exponent reuse is implemented. A small subgroup key recovery attack on IKE
would be primarily offline for IKEv1 with signature authentication and for
IKEv2 against the initiator.

For each subgroup of order $q_i$, the attacker's goal is to obtain a responder
\texttt{AUTH} message, which depends on the secret chosen by the responder. If
an \texttt{AUTH} message can be obtained, the attacker can brute-force the
responder's secret within the subgroup offline. This is possible if the server
supports IKEv1 Aggressive Mode, since the server authenticates before the
client, and signature authentication produces a value dependent on the
negotiated secret.  In all other IKE modes, the client authenticates first,
leading to an online attack. The flow of the attack is identical to TLS; for
more details see Section~\ref{sec:tls}.

Ferguson and Schneier~\cite{ferguson2000cryptographic} describe a hypothetical
small-subgroup attack against the initiator where a man-in-the-middle attacker
abuses undefined behavior with respect to UDP packet retransmissions. A
malicious party could ``retransmit'' many key exchange messages to an initiator
and potentially receive a different authentication message in response to each,
allowing a mostly offline key recovery attack.

%The attacker must choose their key exchange value to be a generator of the
%selected subgroup, which we call $g_i$.  When the responder computes the
%shared secret, $g_i^a$, it will lie within the chosen subgroup.  The attacker
%must guess $g_i^a \bmod q_i$, and construct their \texttt{AUTH} message
%accordingly.  The responder will reply with an \texttt{AUTH} message in the
%event of a correct guess, and an error message otherwise.  With repeated
%guessing, the attacker can learn the value of $g_i^a$ for each subgroup, and
%take advantage of the Pollard labmda algorithm to solve for the rest of the
%secret offline within the reduced search space.

\subsection{Implementations}

We examined several open-source IKE implementations to understand server
behavior.  In particular, we looked for implementations that generate small
Diffie-Hellman exponents, repeat exponents across multiple connections, or do
not correctly validate subgroup order. Despite the suggestion in IKEv2 RFC 7296
to reuse exponents~\cite{rfc7296}, none of the implementations that we examined
reused secret exponents. 

% This is already included in the intro
%RFC 6989~\cite{rfc6989} (``Additional Diffie-Hellman Tests for IKEv2'')
%specifies additional checks that IKE implementations supporting MODP groups
%with small subgroups should perform. The RFC requries IKE implementations to
%choose between either checking that the peer's public value is in the correct
%subgroup, or it must never reuse Diffie-Hellman private values. 

All implementations we reviewed are based on FreeS/WAN~\cite{freeswan}, a
reference implementation of IPSec. The final release of FreeS/Wan, version
2.06, was released in 2004. Version 2.04 was forked into
Openswan~\cite{openswan} and strongSwan\cite{strongswan}, with a further fork
of Openswan into Libreswan~\cite{libreswan} in 2012.  The final release of
FreeS/WAN used constant length 256-bit exponents but did not support RFC~5114
DSA groups, offering only the Oakley 1024-bit and 1536-bit groups that use safe
primes.

\IKEGroupSupportAndValidationTable

Openswan does not generate keys with short exponents. By default, RFC~5114
groups are not supported, although there is a compile-time option that can be
explicitly set to enable support for DSA groups.  strongSwan both supports
RFC~5114 groups and has explicit hard-coded exponent sizes for each group. The
exponent size for each of the RFC~5114 DSA groups matches the subgroup size.
However, these exponent sizes are only used if the
\texttt{dh\_exponent\_ansi\_x9\_42} configuration option is set. It also
includes a routine inside an \texttt{\#ifdef} that validates subgroup order by
checking that $g^q \equiv 1 \bmod p$, but validation is not enabled by default.
Libreswan uses Mozilla Network Security Services (NSS)~\cite{nss-overview} to
generate Diffie-Hellman keys. As discussed in Section~\ref{subsec:nss}, NSS
generates short exponents for Diffie-Hellman groups. Libreswan was forked from
Openswan after support for RFC~5114 was added, and retains support for those
groups if it is configured to use them. 

Although none of the implementations we examined were configured to reuse
Diffie-Hellman exponents across connections, the failure to validate subgroup
orders even for the pre-specified groups renders these implementations fragile
to future changes and vulnerable to subgroup confinement attacks.

Several closed source implementations also provide support for RFC~5114
Group~24. These include Cisco's IOS~\cite{ciscogroup24}, Juniper's
Junos~\cite{junosgroup24}, and Windows Server 2012 R2~\cite{windowsgroup24}. We
were unable to examine the source code for these implementations to determine
whether or not they validate subgroup order.

%\IKEGroupSupportAndValidationTable

\subsection{Measurements}

We performed a series of Internet scans using ZMap to identify IKE responders.
In our analysis, we only consider hosts that respond to our ZMap scan probes.
Many IKE hosts that filter their connections based on IP are excluded from our
results.  We further note that, depending on VPN server configurations, some
responders may continue with a negotiation that uses weak parameters until they
are able to identify a configuration for the connecting initiator. At that
point, they might reject the connection. As an unauthenticated initiator, we
have no way of distinguishing this behavior from the behaviour of a VPN server
that legitimately accepts weak parameters. For a more detailed explanation of
possible IKE responder behaviors in response to scanning probes, see
Wouters~\cite{paul-wouters}.

In October 2016, we performed a series of scans offering the most common cipher
suites and group parameters we found in implementations to establish a baseline
population for IKEv1 and IKEv2 responses. For IKEv1, the baseline scan offered
Oakley groups 2 and 14 and RFC~5114 groups 22, 23, and 24 for the group
parameters; SHA1 or SHA256 for the hash function; pre-shared key or RSA
signatures for the authentication method; and AES-CBC, 3DES, and DES for the
encryption algorithm.  Our IKEv2 baseline scan was similar, but also offered
the 256-bit and 384-bit ECP groups and AES-GCM for authenticated encryption.

On top of the baseline scans, we performed additional scans to measure support
for the non-safe RFC~5114 groups and for key exchange parameter validation.
Table~\ref{tab:ikegroupsupportandvalidation} shows the results of the October
IKE scans.  For each RFC~5114 DSA group, we performed four handshakes with each
host; the first tested for support by sending a valid client key exchange
value, and the three others tested values that should be rejected by a
properly-validating host. We did not scan using the key exchange value $0$
because of a vulnerability present in unpatched Libreswan and Openswan
implementations that causes the IKE daemon to restart when it receives such a
value~\cite{cve-2015-3240}.

We considered a host to accept our key exchange value if after receiving the
value, it continued the handshake without any indication of an error. We found
that 33.2\% of IKEv1 hosts and 17.7\% of IKEv2 hosts that responded to our
baseline scans supported using one of the RFC 5114 groups, and that a
surprising number of hosts failed to validate key exchange values.  24.8\% of
IKEv1 hosts that accepted Group 23 with a valid key exchange value also
accepted $1 \bmod p$ or $-1 \bmod p$ as a key exchange value, even though this
is explicitly warned against in the RFC~\cite{rfc2412}.  This behavior leaves
these hosts open to a small subgroup confinement attack even for safe primes,
as described in Section~\ref{subsec:small-subgroup-attack}.

For safe groups, a check that the key exchange value is strictly between $1$
and $p-1$ is sufficient validation. However, when using non-safe DSA primes, it
is also necessary to verify that the key exchange value lies within the correct
subgroup (\ie, $y^q \equiv 1 \bmod p$). To test this case, we constructed a
generator of a subgroup that was not the intended DSA subgroup, and offered
that as our key exchange value. We did not find any IKEv1 hosts that rejected
this key exchange value after previously accepting a valid key exchange value
for the given group. For IKEv2, the results were similar with the exception of
Group 24, where still over 93\% of hosts accepted this key exchange value. This
suggests that almost no hosts supporting DSA groups are correctly validating
subgroup order. 

We observed that across all of the IKE scans, 109 IKEv1 hosts and 52 IKEv2
hosts repeated a key exchange value. This may be due to entropy issues in key
generation rather than static Diffie-Hellman exponents; we also found 15,891
repeated key exchange values across different IP addresses. We found no hosts
that used both repeated key exchange values and non-safe groups. We summarize
these results in Table~\ref{tab:scandata}. 

%The results of these scans are presented in Table~\ref{tab:scandata}.  To get
%a an estimate of the number of hosts that support each IKE version, we
%conducted scans of 1\% of the IPv4 address space using Zmap. If the host
%responded with a valid message for the IKE version we were scanning for, we
%considered it to support that version. The numbers present for IKE support in
%Table~\ref{tab:scandata} are extrapolated from these 1\% scans.

%To measure support for each of the RFC 5114 DSA groups, we conducted
%additional 1\% scans for IKEv1 Main Mode and IKEv2. We advertised a variety of
%common ciphersuite parameters, but only a single Diffie-Hellman group for each
%of these scans.  We considered a host willing to negotiate that group if they
%responded with a valid key exchange payload.  Table~\ref{tab:scandata} shows
%the number of hosts that were willing to negotiate any of the RFC 5114 DSA
%groups for IKEv1 Main Mode and IKEv2.  These hosts fit at least one of the
%four conditions for the small subgroup attack by using a prime $p$ where $p-1$
%has small factors.

%We performed additional scans to measure if any hosts repeated Diffie-Hellman
%key exchange values. First, we performed two simultaneous IKEv1 Main Mode
%scans proposing Group 23 only. In this scan, we did not observe any repeated
%key exchange values. However, across all the IKE scans that we performed, we
%observed that 109 hosts for IKEv1 and 52 hosts for IKEv2 repeated key exchange
%values at least once with Group 2 (a 1024-bit MODP group).
