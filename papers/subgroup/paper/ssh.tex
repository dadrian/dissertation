% !TEX root = ../../../proposal.tex

\section{SSH}

SSH contains three key agreement methods that make use of Diffie-Hellman. The
``Group 1'' and ``Group 14'' methods denote Oakley Group 2 and Oakley Group
14, respectively~\cite{rfc4253}. Both of these groups use safe primes. The
third method, ``Group Exchange'', allows server to select a custom
group~\cite{rfc4419}. The group exchange RFC specifies that all custom groups
should use safe primes. Despite this, RFC~5114 notes that group exchange method
allows for its DSA groups in SSH, and advocates for their immediate
inclusion~\cite{rfc5114}.

In all Diffie-Hellman key agreement methods, after negotiating cipher selection and group parameters, 
the SSH client generates a public
key exchange value $y_c = g^{x_c} \bmod p$ and sends it to the server. The
server computes its own Diffie-Hellman public value $y_s = g^{x_s} \bmod p$ and
sends it to the client, along with a signature from its host key over the
resulting shared secret $Y = g^{x_s x_c} \bmod p$ and the hash of the handshake
so far.  The client verifies the signature before continuing.

\subsection{Small Subgroup Attacks in SSH}

\paragraph{Small subgroup confinement attacks}
An SSH client could execute a small subgroup confinement attack against an SSH
server by sending a generator $g_i$ for a subgroup of small order $q_i$ as its
client key exchange, and immediately receive the server's key exchange $g^{x_s}
\bmod p$ together with a signature that depends on the server's view of the
shared secret $Y_s = g_i^{x_s} \bmod p$. For small $q_i$, this allows the
client to brute force the value of $x_s \bmod q_i$ offline and compare to the
server's signed handshake to learn the correct value of $x_s \bmod q_i$. To
avoid this, the SSH RFC specifically recommends using safe primes, and to use
exponents at least twice the length of key material derived from the shared
secret~\cite{rfc4419}. 

% LUKEV: Avoid describing the mitm attack in all it's detail again.
If client and server support Diffie-Hellman group exchange and the server uses
a non-safe prime, a man in the middle with knowledge of the server's long-term
private signing key can use a small subgroup confinement attack to
man-in-the-middle the connection without having to rewrite every message.  The
attack is similar to the case of TLS: the man in the middle modifies the server
group and key exchange messages, leaving the prime unchanged, but substituting
a generator $g_i$ of a subgroup of small order $q_i$ for the group generator
and $g_i$ for the server's key exchange value $y_s$.  The client then responds
with a client key exchange message $y_c = g_i^{x_c} \bmod p$, which the man in
the middle leaves unchanged.  The attacker then forges a correct signature for
the modified server group and key exchange messages and passes it to the
client.  The server's view of the shared secret is $g_i^{x_c x_s} \bmod p$, and
the client's view of the shared secret is $g_i^{x_c} \bmod p$.  As in the
attack described for TLS, these views are identical when $x_s \equiv 1 \bmod
q_i$, so this connection will succeed with probability $1/q_i$.  For a small
enough $q_i$, this enables a man in the middle to use a compromised server
signing key to decrypt traffic with a reasonable probability of success, while
only requiring tampering with the initial handshake messages, rather than
having to actively rewrite the entire connection for the duration of the
session.

\paragraph{Small subgroup key recovery attacks}
Since the server immediately sends a signature over the public values and the
Diffie-Hellman shared secret, an implementation using static exponents and non-safe primes that is
vulnerable to a small subgroup confinement attack would also be vulnerable to a
mostly offline key recovery attack, as a malicious client would only need to
send a single key exchange message per subgroup.


\SSHHostValidationTable

\ECMDistributionTable

\subsection{Implementations}

Censys~\cite{censys} SSH banner scans show that the two most common SSH server implementations
are Dropbear and OpenSSH. Dropbear group exchange uses hard-coded safe prime parameters from the Oakley groups and validates that client key exchange values are greater than 1 and less than $p - 1$. While OpenSSH
only includes safe primes by default, it does provide the ability to add additional primes and
does not provide the ability to specify subgroup orders. Both OpenSSH and Dropbear generate fresh
exponents per connection.

We find one SSH implementation, Cerberus SFTP server (FTP over SSH), repeating server exponents across connections. 
Cerberus uses OpenSSL, but fails to set \texttt{SSL\_OP\_SINGLE\-\_DH\_USE}, which was required
to avoid exponent reuse prior to OpenSSL 1.0.2f.

\subsection{Measurements}

Of the 15.2\,M SSH servers on Censys, of which 10.7\,M support Diffie-Hellman
group exchange, we found that 281 used a non-safe prime, and that
1.1\,K reused Diffie-Hellman exponents. All but 26 of the hosts that reused exponents had banners identifying the Cerberus SFTP server. We
encountered no servers that both reused exponents and used non-safe primes.

We performed a scan of 1\% of SSH hosts in February 2016 offering the key
exchange values of $y_c = 0 \bmod p, 1 \bmod p$ and $p-1 \bmod p$. As
Table~\ref{tab:sshvalidation} shows, 33\% of SSH hosts failed to validate group
order when we sent the key exchange value $p-1 \bmod p$. Even when safe groups
are used, this behaviour allows an attacker to learn a single bit of the
private exponent, violating the decisional Diffie-Hellman assumption and
leaving the implementation open to a small subgroup confinement attack
(Section~\ref{sec:tls-subgroup-attack}).

