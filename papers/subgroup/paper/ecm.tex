% !TEX root = ../../../proposal.tex

\section{Factoring Group Orders of Non-Safe Primes}
\label{sec:ecm}

Across all scans, we collected 41,847 unique groups with non-safe primes.
To measure the extent to which each group would facilitate a small subgroup
attack in a vulnerable implementation, we attempted to factor $(p-1)/2$. We
used the GMP-ECM~\cite{gmp-ecm-zimmerman-2012} implementation of the elliptic curve method for
integer factorization on a local cluster with 288 cores over a several-week
period to opportunistically find small factors of the group order for each of
the primes.


Given a group with prime $p$ and a generator $g$, we can check whether the
generator generates the entire group or generates a subgroup by testing whether
$g^{q_i} \equiv 1 \bmod p$ for each factor $q_i$ of $(p-1)/2$.  When $g^{q_i}
\equiv 1 \bmod p$, then if $q_i$ is prime, we know that $q_i$ is the exact
order of the subgroup generated by $g$; otherwise $q_i$ is a multiple of the
order of the subgroup. We show the distribution of group order for groups using
non-safe primes in Table~\ref{tab:ecm-distribution}.  We were able to
completely factor $p-1$ for 4,701 primes.  For the remaining primes, we
did not obtain enough factors of $(p-1)/2$ to determine the group order. 

%We show the number of groups for which the difference in size of $\lg(p)$ and
%$\lg(q_i)$ is at least 8 bits, since these subgroups are more likely to have
%been intentionally generated \todo{come up with better reason for why this is
%interesting}. 

Of the groups where we were able to deduce the exact subgroup orders, several
thousand had a generator for a subgroup that was either 8, 32, or 64 bits
shorter than the prime itself.  Most of these were generated by the Xlight FTP
server, a closed-source implementation supporting SFTP.  It is not clear
whether this behavior is intentional or a bug in an implementation intending to
generate safe primes.  Primes of this form would lead to a more limited
subgroup confinement or key recovery attack.

Given the factorization of $(p-1)/2$, and a limit for the amount of online and
offline work an attacker is willing to invest, we can estimate the
vulnerability of a given group to a hypothetical small subgroup key recovery
attack. For each subgroup of order $q_i$, where $q_i$ is less than the online
work limit, we can learn $q_i$ bits of the secret key via an online brute-force
attack over all elements of the subgroup. To recover the remaining bits of the
secret key, an attacker could use the Pollard lambda algorithm, which runs in
time proportional to the square root of the remaining search space. If this
runtime is less than the offline work limit, we can recover the entire secret
key. We give work estimates for the primes we were able to factor and the
number of hosts that would be affected by such a hypothetical attack in
Table~\ref{tab:ecm-breakable}.

The DSA groups introduced in RFC 5114~\cite{rfc5114} are of particular
interest. We were able to completely factor $(p-1)/2$ for both Group 22 and
Group 24, and found several factors for Group 23. We give these factorizations
in Table~\ref{tab:group-order-factorization}.
In Table~\ref{tab:ecm-rfc5114}, we show the amount of online and offline work
required to recover a secret exponent for each of the RFC 5114 groups. In
particular, an exponent of the recommended size used with Group 23 is fully
recoverable via a small subgroup attack with 33 bits of online work and 47 bits
of offline work.

\ECMBreakableGroups

\ECMRFCFiveFiveOneFourGroups

\GroupOrderFactorization
