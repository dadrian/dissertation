% !TEX root = ../../../proposal.tex



%\section{Introduction}
%
%Diffie-Hellman key exchange is one of the most common public-key cryptographic
%methods in use in the Internet. It is a fundamental building block for IPsec,
%SSH, and TLS\@. In the textbook presentation of finite field Diffie-Hellman,
%Alice and Bob agree on a large prime $p$ and an integer $g$ modulo $p$. Alice
%chooses a secret integer $x_a$ and transmits a public value $g^{x_a} \bmod p$;
%Bob chooses a secret integer $x_b$ and transmits his public value $g^{x_b}
%\bmod p$. Both Alice and Bob can reconstruct a shared secret $g^{x_a x_b} \bmod
%p$, but the best known way for a passive eavesdropper to reconstruct this
%secret is to compute the discrete log of either Alice or Bob's public value.
%Specifically, given $g$, $p$, and $g^x \bmod p$, an attacker must calculate
%$x$.

This chapter is adapted from a joint publication with Valenta et al.\ that
originally appeared in the proceedings of the 21st Network and Distributed
System Security Symposium (NDSS '17)~\cite{subgroup-2017}.

In order for the discrete log problem $\bmod p$ to be hard, Diffie-Hellman parameters
must be chosen carefully. A typical recommendation is that $p$ should be a
``safe'' prime, that is, that $p = 2q+1$ for some prime $q$, and that $g$
should generate the group of order $q$ modulo $p$. For $p$ that are not safe,
the group order $q$ can be much smaller than $p$. For security, $q$ must still
be large enough to thwart known attacks, which for prime $q$ run in time
$O(\sqrt{q})$. A common parameter choice is to use a 160-bit $q$ with a
1024-bit $p$ or a 224-bit $q$ with a 2048-bit $p$, to match the security level
under different cryptanalytic attacks. Diffie-Hellman parameters with $p$ and
$q$ of these sizes were suggested for use and standardized in DSA
signatures~\cite{fips186}. For brevity, we will refer to these non-safe primes as
DSA primes, and to groups using DSA primes with smaller values of $q$ as  DSA
groups.

A downside of using DSA primes instead of safe primes for Diffie-Hellman is
that implementations must perform additional validation checks to ensure the
key exchange values they receive from the other party are contained in the
correct subgroup modulo $p$. The validation consists of performing an extra
exponentiation step. If implementations fail to validate, a 1997 attack of Lim
and Lee~\cite{lim-1997} can allow an attacker to recover a static exponent by
repeatedly sending key exchange values that are in very small subgroups. We
describe several variants of small subgroup confinement attacks that allow an
attacker with access to authentication secrets to mount a much more efficient
man-in-the-middle attack against clients and servers that do not validate group
orders. Despite the risks posed by these well-known attacks on DSA groups, NIST SP 800-56A, ``Recommendations
for Pair-Wise Key Establishment Schemes Using Discrete Logarithm
Cryptography''~\cite{sp800} specifically recommends DSA group parameters
for Diffie-Hellman, rather than recommending using safe primes. RFC
5114~\cite{rfc5114} includes several DSA groups for use in IETF standards.

We observe
that few Diffie-Hellman implementations actually validate subgroup orders, in spite of the fact
that small subgroup attacks and countermeasures are well-known and specified
in every standard suggesting the use of DSA groups for Diffie-Hellman, and DSA
groups are commonly implemented and supported in popular protocols. For some protocols, including
TLS and SSH, that enable the server to unilaterally specify the group used for
key exchange, this validation step is not possible for clients to perform with
DSA primes---there is no way for the server to communicate to the client the
intended order of the group. Many standards involving DSA groups further
suggest that the order of the subgroup should be matched to the length of the
private exponent.  Using shorter private exponents yields faster exponentiation
times, and is a commonly implemented optimization. However, these standards
provide no security justification for decreasing the size of the subgroup to
match the size of the exponents, rather than using as large a subgroup as
possible. We discuss possible motivations for these recommendations
later in the paper.

We conclude that adopting the Diffie-Hellman group recommendations from RFC
5114 and NIST SP 800-56A may create vulnerabilities for organizations using
existing cryptographic implementations, as many libraries allow
user-configurable groups but have unsafe default behaviors.  This highlights the
need to consider developer usability and implementation fragility when designing
or updating cryptographic standards.

\paragraph{Our Contributions}
We study the implementation landscape of Diffie-Hellman from several
perspectives and measure the security impact of the widespread
failure of implementations to follow best security practices:
\begin{itemize}
\item We summarize the concrete impact of small-subgroup confinement attacks
    and small subgroup key recovery attacks on TLS, IKE, and SSH handshakes.
\item We examined the code of a wide variety of cryptographic libraries to
  understand their implementation choices. We find feasible full private
  exponent recovery vulnerabilities in OpenSSL and the Unbound DNS resolver,
  and a partial private exponent recovery vulnerability for the parameters used
  by the Amazon Elastic Load Balancer. We observe that \emph{no} implementation
  that we examined validated group order for subgroups of order larger than two
  by default prior to January 2016, leaving users potentially vulnerable to
  small subgroup confinement attacks.
  %In addition, we observed that nearly every implementation uses short
  %exponents by default, and several use ephemeral-static keys.
\item We performed Internet-wide scans of HTTPS, POP3S, SMTP with STARTTLS,
  SSH, IKEv1, and IKEv2, to provide a snapshot of the deployment of DSA groups
  and other non-``safe'' primes for Diffie-Hellman, quantify the incidence of
  repeated public exponents in the wild, and quantify the lack of validation
  checks even for safe primes.
  %Our work adds to the growing literature of empirical
  %studies of cryptographic implementation behavior on the Internet.
%\item We surveyed the protocol-level susceptibility to small subgroup attack
%    scenarios for TLS, IKE, and SSH.  While several of these attacks are well
%    known or have been described elsewhere, we are unaware of a comprehensive
%    reference that summarizes the state of protocol landscape from this
%    perspective. We describe variants of these attacks that we have not
%    seen elsewhere.
\item We performed a best-effort attempt to factor $p-1$ for all non-safe primes that we found in the
    wild, using \textasciitilde100,000 core-hours
    of computation. Group 23 from RFC 5114, a 2048-bit prime, is particularly vulnerable
    to small subgroup key recovery attacks; for TLS a full key recovery
    requires $2^{33}$ online work and $2^{47}$ offline work to recover a
    224-bit exponent.
\end{itemize}

\paragraph{Disclosure and Mitigations}
We reported the small subgroup key recovery vulnerability to OpenSSL in January
2016~\cite{asanso-unredacted}. OpenSSL issued a patch to add additional
validation checks and generate single-use private exponents by
default~\cite{cve-2016-0701}. We reported the Amazon load balancer
vulnerability in November 2015. Amazon responded to our report informing us
that they have removed Diffie-Hellman from their recommmended ELB security
policy, and have reached out to their customers to recommend that they use
these latest policies. Based on scans performed in February and May 2016, 88\%
of the affected hosts appear to have corrected their exponent generation
behavior. We found several libraries that had vulnerable combinations of
behaviours, including Unbound DNS, GnuTLS, LibTomCrypt, and Exim. We disclosed
to the developers of these libraries. Unbound issued a patch, GnuTLS
acknowledged the report but did not patch, and LibTomCrypt did not respond.
Exim responded to our bug report stating that they would use their own
generated Diffie-Hellman groups by default, without specifying subgroup order
for validation~\cite{exim-blog,exim-bug-report}.  We found products from Cisco,
Microsoft, and VMWare lacking validation that key exchange values were in
the range $(1,p-1)$. We informed these companies, and discuss their responses
in Section~\ref{sec:tls-measurements}.

%In May 2016, we submitted bug reports to the developers
%of several applications and libraries that had vulnerable combinations of
%behaviours, including the Unbound DNS resolver, the Exim mail server, and the
%LibTomCrypt and GnuTLS libraries.  Developers of the Unbound DNS resolver
%informed us that a fix would be included in the next release. The GnuTLS
%developers acknowledged the issue but have not yet applied patches. In October
%2016, Exim responded to our bug report stating that they would use their own
%generated Diffie-Hellman groups by default, without specifying subgroup order
%for validation~\cite{exim-blog,exim-bug-report}.  There was no response from
%LibTomCrypt. We found that products of several noteworthy companies, including
%Microsoft, Cisco, and VMWare, do not validate that a key exchange value is in
%the range $(1, p-1)$.  We informed these companies and others about this
%missing check, and discuss their responses in
%Section~\ref{sec:tls-measurements}.
