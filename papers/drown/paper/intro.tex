% !TEX root = ../../../proposal.tex
%\paragraph{Transport Layer Security (TLS).}
TLS~\cite{rfc5246} is one of the main protocols responsible for transport
security on the modern Internet. TLS and its precursor SSLv3 have been the
target of a large number of cryptographic attacks in the research community,
both on popular implementations and the protocol
itself~\cite{chronology-tls-attacks-2013}. Prominent recent examples include
attacks on outdated or deliberately weakened encryption in
RC4~\cite{rc4-biases-2013}, RSA~\cite{freak-attack-2015}, and
Diffie-Hellman~\cite{logjam-2015}, different side channels including
Lucky13~\cite{lucky-13-2013}, BEAST~\cite{beast-2011}, and
POODLE~\cite{poodle-2014}, and several attacks on invalid TLS protocol
flows~\cite{freak-attack-2015,protocol-state-fuzzing-2015,triple-handshake-2014}.

Comparatively little attention has been paid to the \ssltwo protocol,
likely because the known attacks are so devastating and the protocol
has long been considered obsolete.  Wagner and Schneier wrote in 1996 that
their attacks on \ssltwo ``will be irrelevant in the long term
when servers stop accepting SSL 2.0 connections''~\cite{ssl-v3-1996}.
Most modern TLS clients do not support \ssltwo at all.
Yet in 2016, our Internet-wide scans find that out of 36~million
HTTPS servers, 6~million (17\%) support \ssltwo.

%However, a surprisingly large number
%of servers, including 16\% of all HTTPS servers, 5\% of those HTTPS servers
%running browser-trusted certificates, and 36\% of SMTP servers
% still support \ssltwo.  

%\paragraph{Bleichenbacher's attack.}
%\vspace{-3pt}

\paragraph{A Bleichenbacher attack on \ssltwo.}
Bleichenbacher's padding oracle attack~\cite{bleichenbacher-1998} is an
adaptive chosen ciphertext attack against \PKCS, the RSA padding
standard used in SSL and TLS\@.  It enables decryption of RSA
ciphertexts if a server distinguishes between correctly and
incorrectly padded RSA plaintexts, and was termed the
``million-message attack'' upon its introduction in 1998, after the
number of decryption queries needed to deduce a plaintext.  All
widely used SSL/TLS servers include
countermeasures against Bleichenbacher attacks.

Our first result shows that the \ssltwo protocol is fatally vulnerable to a
form of Bleichenbacher attack that enables decryption of RSA ciphertexts. We
develop a novel application of the attack that allows us to use a server that
supports \ssltwo as an efficient padding oracle. This attack is a
protocol-level flaw in \ssltwo that results in a feasible attack for 40-bit
export cipher strengths, and in fact abuses the universally implemented
countermeasures against Bleichenbacher attacks to obtain a decryption oracle.
%\looseness=-1

We also discovered multiple implementation flaws in commonly deployed OpenSSL
versions that allow an extremely efficient and much more dangerous
instantiation of this attack.

\paragraph{Using \ssltwo to break TLS\@.}
Second, we present a novel \textit{cross-protocol attack} that allows an
attacker to break a passively collected RSA key exchange for any TLS server
if the RSA keys are also used for \ssltwo, possibly on a different server. We
call this attack DROWN (\emph{Decrypting RSA using Obsolete and Weakened
eNcryption}).

In its \emph{general} version, the attack exploits the protocol flaws in
\ssltwo, does not rely on any particular library implementation, and is
feasible to carry out in practice by taking advantage of commonly supported
export-grade ciphers. In order to decrypt one TLS session, the attacker must
passively capture about 1,000 TLS sessions using RSA key exchange, make
40,000 \ssltwo connections to the victim server, and perform $2^{50}$
symmetric encryption operations. We successfully carried out this attack
using an optimized GPU implementation and were able to decrypt a 2048-bit RSA
ciphertext in less than 18 hours on a GPU cluster and less than 8 hours using
Amazon EC2.

We found that 11.5~million HTTPS servers (33\%) are vulnerable to this
attack, because many HTTPS servers that do not directly support \ssltwo share
RSA keys with other services that do. Of servers offering HTTPS with
browser-trusted certificates, 22\% are vulnerable.

We also present a \emph{special} version of DROWN that exploits flaws in
OpenSSL for a more efficient oracle. It requires roughly the same number of
captured TLS sessions as the general attack, but only half as many
connections to the victim server and no large computations. This attack can
be completed on a single core on commodity hardware in less than a minute,
\ifext without GPUs or distributed computing, \fi and is limited primarily by
how fast the server can complete handshakes. It is fast enough that an
attacker can perform man-in-the-middle attacks on live TLS sessions before
the handshake times out, and {downgrade} a modern TLS client to RSA key
exchange with a server that prefers non-RSA cipher suites. Our Internet-wide
scans suggest that 79\% of HTTPS servers that are vulnerable to the general
attack, or 26\% of all HTTPS servers, are also vulnerable to real-time
attacks exploiting these implementation flaws.

Our results highlight the risk that continued support for SSLv2 imposes on
the security of much more recent TLS versions. This is an instance of a more
general phenomenon of insufficient domain separation, where older, vulnerable
security standards can open the door to attacks on newer versions. We
conclude that phasing out outdated and insecure standards should become a
priority for standards designers and practitioners.

\paragraph{Disclosure.}
DROWN was assigned CVE-2016-0800. We disclosed our attacks to OpenSSL and
worked with them to coordinate further disclosures. The specific OpenSSL
vulnerabilities we discovered have been designated CVE-2015-3197,
CVE-2016-0703, and CVE-2016-0704. In response to our findings, OpenSSL has
made it impossible to configure a TLS server in such a way that it is
vulnerable to DROWN\@. Microsoft had already disabled \ssltwo for all
supported versions of IIS\@. We also disclosed the attack to the NSS
developers, who have disabled \ssltwo on the last NSS tool that supported it
and have hastened efforts to entirely remove the protocol from their
codebase. In response to our disclosure, Google will disable QUIC support for
non-whitelisted servers and modify the QUIC standard. We also notified IBM,
Cisco, Amazon, the German CERT-Bund, and the Israeli CERT\@.
