%% In the following section, we experimentally evaluate the cost of brute forcing export \texttt{master\_key} values on CPU, GPU, FPGA, and cloud computing platforms. 
%% We then experimentally evaluate our general DROWN attack using the 
%% \texttt{SSL\_RC2\_128\_CBC\_EXPORT40\_WITH\_MD5}
%% cipher suite, which is the most suitable for this attack.

%% \subsubsection{Comparing hardware platforms}
\if0
The most computationally expensive part of our general DROWN attack is breaking the 40-bit symmetric key.  We wanted to find the platform that would have the best tradeoff of cost and speed for the attack, so we performed some preliminary experiments comparing performance of symmetric key breaking on CPUs, GPUs, and FPGAs.  These experiments used a na\"{\i}ve version of the attack using the OpenSSL implementation of MD5 and RC2.

The CPU machine contained four Intel Xeon E7-4820 CPUs with a total of 32 cores (64 concurrent threads). The GPU system was equipped with a ZOTAC GeForce GTX TITAN and an Intel Xeon E5-1620 host CPU\@. The FPGA setup consisted of 64 Spartan-6 LX150 FPGAs.

We benchmarked the performance of the CPU and GPU implementations over a large corpus of randomly generated keys, and then extrapolated to the full attack.
For the FPGAs, we tested the functionality in simulation and estimated the actual runtime by theoretically filling the FPGA up to 90\% with the design, including communication.
Table~\ref{perf_comparison} compares the three platforms.

While the FPGA implementation was the fastest in our test setup, the speed-to-cost ratio of GPUs was the most promising. Therefore, we decided to focus on optimizing the attack on the GPU platform.
\fi

\ifext
\subsubsection{Optimized GPU implementation}
\label{sec:gpu_brief}
We developed a highly optimized GPU implementation of our general DROWN brute force attack.  Our first na\"{\i}ve GPU implementation performed around 26MH/s, where MH measures the calculation of an MD5 hash and the RC2 decryption. The optimizations described below gave a final speed of 515MH/s, a speedup factor of 19.8.  
\fi

%\label{sec:gpu}

\GPUTable

The most computationally expensive part of our general DROWN attack is breaking the 40-bit symmetric key.  We wanted to find the platform that would have the best tradeoff of cost and speed for the attack, so we performed some preliminary experiments comparing performance of symmetric key breaking on CPUs, GPUs, and FPGAs.  These experiments used a na\"{\i}ve version of the attack using the OpenSSL implementation of MD5 and RC2.

The CPU machine contained four Intel Xeon E7-4820 CPUs with a total of 32 cores (64 concurrent threads). The GPU system was equipped with a ZOTAC GeForce GTX TITAN and an Intel Xeon E5-1620 host CPU\@. The FPGA setup consisted of 64 Spartan-6 LX150 FPGAs.

We benchmarked the performance of the CPU and GPU implementations over a large corpus of randomly generated keys, and then extrapolated to the full attack.
For the FPGAs, we tested the functionality in simulation and estimated the actual runtime by theoretically filling the FPGA up to 90\% with the design, including communication.
Table~\ref{perf_comparison} compares the three platforms.

While the FPGA implementation was the fastest in our test setup, the speed-to-cost ratio of GPUs was the most promising. Therefore, we decided to focus on optimizing the attack on the GPU platform.
We developed several optimizations:

\paragraph{Generating key candidates on GPUs.} Our na\"{\i}ve implementation generated key candidates on the CPUs. For each \hashcomputation, a key candidate was transmitted to the GPU, and the GPU responded with the key validity. The bottleneck in this approach was the PCI-E Bus. Even newer boards with PCI-E 3.0 or even PCI-E 4.0 are too slow to handle the large amount of data required to keep the GPUs busy. We solved this problem by generating the key candidates directly on the GPUs.
	
\paragraph{Generating memory blocks of keys.}
Our hash computation kernel had to access different candidate keys from the GPU memory. Accessing global memory is typically a slow operation and we needed to keep memory access as minimal as possible. Ideally we would be able to access the candidate keys on a register level or from a constant memory block, which is almost as fast as a register. However, there are not enough registers or constant memory available to store all the key values. 

We decided to divide each key value into two parts $k_H$ and $k_L$, where $|k_H|=1$ byte and $|k_L|=4$ bytes. We stored all possible $2^8$ $k_H$ values in the constant read-only memory, and all possible $2^{32}$  $k_L$ values in the global memory. 
Next we used an in-kernel loop. We loaded the latter 4 bytes from the slow global memory and stored it in registers. Inside the inner loop we iterated through our first byte $k_H$ by accessing the fast constant memory. The resulting key candidate was computed as $k=k_H||k_L$. 
	
\paragraph{Using 32-bit data types.}
Although modern GPUs support several data types ranging in size from 8 to 64 bits, many instructions are designed for 32-bit data types. This fits the design of MD5 perfectly, because it uses 32-bit data types. RC2, however, uses both 8-bit and 16-bit data types, which are not suitable for 32-bit instruction sets. This forced us to rewrite the original RC2 algorithm to use 32-bit instructions.
% to prevent casting from different datatypes which would typically involve a additional bitfield extraction call.
	
\paragraph{Avoiding loop branches.} Our kernel has to concatenate several inputs to generate the \texttt{server\_write\_key} needed for the encryption as described in Section~\ref{sec:ssl2}. Using loops to move this data generates branches because there is always an if() inside a for() loop. To avoid these branches, which always slow down a GPU implementation, we manually shifted the input bytes into the 32-bit registers for MD5. This was possible since the \hashcomputation inputs,
$(mk_{clear} || mk_{secret} || ``0" || r_c || r_s)$,
have constant length.

\paragraph{Optimizing MD5 computation.} Our MD5 inputs have known input length and block structure, allowing us to use the so-called zero-based optimizations.  Given the known input length (49 bytes) and the fact that MD5 uses zero padding, in our case the MD5 input block included four 0x00 bytes. These \hex{00} bytes are read four times per MD5 computation which allowed us to drop in total 16 ADD operations per MD5 computation. In addition, we applied the Initial-step optimizations used in the Hashcat implementation~\cite{hashcat-talk}.

\paragraph{Skipping the second encryption block.} The input of the brute-force computation is a 16-byte client challenge $r_c$ and the resulting ciphertext from the \texttt{ServerVerify} message which is computed with an RC2 cipher. As RC2 is an 8-byte block cipher the RC2 input is split into two blocks and two RC2 encryptions are performed. In our verification algorithm, we skipped the second decryption step as soon as we saw the key candidate does not decrypt the first plaintext block correctly. This resulted in a speedup of about a factor of 1.5.

\paragraph{RC2 permutation table in constant memory.}
The RC2 algorithm uses a 256-byte permutation table which is constant for all RC2 computations. Hence, this table is a good candidate to be put into the constant memory, which is nearly as fast as registers and makes it easy to address the table elements. When finally using the values, we copied them into the even faster shared memory. Although this copy operation has to be repeated, it still led to a speed up of approximately a factor of 2.

\paragraph{RC2 key setup without keysize checks.}
The key used for RC2 encryption is generated using MD5, thus the key size is always 128 bits. Therefore, we do not have to check for the input key size, and can simply skip the size verification branch completely.



\ifext
We obtained our improvements through a number of optimizations.  Our original implementation ran into a communication bottleneck in the PCI-E bus in transmitting candidate keys from CPU to GPU, so we removed this bottleneck by generating key candidates on the GPU itself.  We optimized memory management, including storing candidate keys and the RC2 permutation table in constant memory, which is almost as fast as a register, instead of slow global memory.  We optimized the cryptographic checks themselves by rewriting the RC2 implementation to use 32-bit instructions, removing unnecessary RC2 keysize checks, dropping unused ADD instructions during MD5, and manually shifting input bytes into the MD5 input registers to avoid loop branches.
\looseness=1
\fi

\ifext

\label{sec:ec2_results}
%In this section, we discuss attack performance and cost when a rented cloud compute cluster is used for the GPU breaking.
\fi

The most computationally expensive part of our general DROWN attack is breaking the 40-bit symmetric key, so we developed a highly optimized GPU implementation of this brute force attack.  Our first na\"{\i}ve GPU implementation performed around 26MH/s, where MH denotes the time required for testing one million possible values of $mk_{secret}$. Our optimized implementation runs at a final speed of 515MH/s, a speedup factor of 19.8.  
\label{sec:gpu_brief}

We obtained our improvements through a number of optimizations.  For example, our original implementation ran into a communication bottleneck in the PCI-E bus in transmitting candidate keys from CPU to GPU, so we removed this bottleneck by generating key candidates on the GPU itself.  We optimized memory management, including storing candidate keys and the RC2 permutation table in constant memory, which is almost as fast as a register, instead of slow global memory. 
\ifext  We optimized the cryptographic checks themselves by rewriting the RC2 implementation to use 32-bit instructions, removing unnecessary RC2 keysize checks, dropping unused ADD instructions during MD5, and manually shifting input bytes into the MD5 input registers to avoid loop branches.  We describe these optimizations in further detail in Appendix~\ref{sec:gpu}. \fi

We experimentally evaluated our optimized implementation on a local cluster and in the cloud.
We used it to execute a full attack of $2^{49.6}$ tested keys on each platform.
The required number of keys to test during the attack is a random variable, distributed geometrically, with an expectation that ranges between $2^{49.6}$ and $2^{52.5}$ depending on the choice of optimization parameters.
We treat a full attack as requiring $2^{49.6}$ tested keys overall.

\paragraph{Hashcat.}
Hashcat~\cite{hashcat} is an open source optimized password-recovery tool.
The Hashcat developers allowed us to use their GPU servers for our attack evaluation. 
The servers contain a total of 40 GPUs: 32 Nvidia GTX 980 cards, and 8 AMD R9 290X cards.
The value of this equipment is roughly \$18,040.
Our full attack took less than 18 hours to complete on the Hashcat servers, with the longest single instance taking 17h9m.

% Nimrod: Generally, we can't report times which are different from wallclock times without further explanation,
% like "retroactively assuming a perfecly balanced distribution..."
% The instance that took longest took 1029m41.176, which is 17.16 hours,
% so I think any number smaller than 18 hours would need to be discussed.

\paragraph{Amazon EC2.}
\label{sec:ec2_results}
% AH: Suggest trimming this as in the submitted version
\ifext
% !TEX root = ../../../proposal.tex
%\section{Brute Forcing Keys with Amazon EC2} 
\label{sec:ec2-details}

%Amazon Elastic Cloud Compute (EC2)~\cite{ec2} is a service that provides on-demand virtualized compute resources to customers. 
%This is an affordable alternative to provisioning one's own local cluster.

Amazon EC2 billing is based on the \textit{instance-hour}. An \textit{instance} represents a single virtualized machine and its associated cores, memory, and storage. For our experiments we used \texttt{g2} instances, which are equipped with high-performance NVIDIA GPUs, each with 1,536 CUDA cores. The two available models for this instance type are the \texttt{g2.2xlarge} and the \texttt{g2.8xlarge}, containing one and four GPUs, respectively.

It is possible to request instances at a fixed on-demand rate, or bid on instances at the discounted spot instance rate. Spot instances may be terminated depending on demand, but the savings in cost are significant compared to the on-demand rate. 
When we ran our experiments in January 2016, the on-demand rate for the \texttt{g2.2xlarge} model was \$0.65/hr and the rate for the \texttt{g2.8xlarge} model was \$2.65/hr, while the average spot rates we paid were \$0.09/hr and \$0.83/hr respectively.

We used a cluster composed of 200 spot instances: 150 \texttt{g2.2xlarge} which contain one GPU and 50 \texttt{g2.8xlarge}, each containing four GPUs, spread across multiple availability zones within the US-East region.
This distribution was determined by price: we were not able to launch more than 50 \texttt{g2.8xlarge} instances without a sharp spike in spot prices. We used the optimized Hashcat implementation on the same workload of key requests as the experiments run on the Hashcat servers.  We used Slurm~\cite{yoo2003slurm} to distribute jobs across compute nodes.

The GPU breaking experiment completed successfully, with two minor caveats. First, the 150 \texttt{g2.2xlarge} nodes completed their workloads at the 6h26m mark, while the other 50 \texttt{g2.8xlarge} nodes did not finish until the 7h41m mark. More careful job distribution would ensure that all nodes completed at approximately the same time, reducing the overall runtime. Second, in this particular run, $7.2\%$ of the jobs that we expected to complete were terminated early due to overheating GPUs.  The attack was successful despite the failed jobs, so we did not rerun them. In a more carefully engineered implementation, the unfinished jobs could have been reallocated to the unused GPU capacity without increasing the overall runtime.

The total cost of the experiment was \$440, and terminated in under 8 hours including startup and shutdown.


\else
We also ran our optimized GPU code on the Amazon Elastic Compute Cloud (EC2) service.  We used a cluster composed of 200 variable-price ``spot'' instances: 150 \texttt{g2.2xlarge} instances, each containing one high-performance NVIDIA GPU with 1,536 CUDA cores and 50 \texttt{g2.8xlarge} instances, each containing four of these GPUs.  
When we ran our experiments in January 2016, the average spot rates we paid were \$0.09/hr and \$0.83/hr respectively.  
%The 150 \texttt{g2.2xlarge} nodes finished after 6h26m, while the \texttt{g2.8xlarge} finished after 7h41m.  $7.2\%$ of the jobs that we expected to complete failed due to overheating GPUs.  The attack was successful despite the failed jobs, so we did not rerun them.  
Our full attack finished in under 8 hours including startup and shutdown for a cost of \$440.  
\ifext See Appendix~\ref{sec:ec2-details} for more details. \fi
\fi


\subsection{OpenSSL SSLv2 cipher suite selection bug}

General DROWN is a protocol flaw, but the population of vulnerable hosts is
increased due to a bug in OpenSSL that causes many servers to erroneously
support \ssltwo and export ciphers even when configured not to. The OpenSSL
team intended to disable \ssltwo by default in 2010, with a change that removed
all \ssltwo cipher suites from the default list of ciphers offered by the
server~\cite{opensslchangelog}.  However, the code for the protocol itself was
not removed in standard builds and \ssltwo itself remained enabled. We
discovered a bug in OpenSSL's \ssltwo cipher suite negotiation logic that
allows clients to select \ssltwo cipher suites even when they are not
explicitly offered by the server. We notified the OpenSSL team of this
vulnerability, which was assigned CVE-2015-3197.  The problem was fixed in
OpenSSL releases 1.0.2f and 1.0.1r~\cite{opensslchangelog}.
\looseness=1
