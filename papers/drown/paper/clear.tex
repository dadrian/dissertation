\label{sec:special}

We discovered multiple vulnerabilities in recent
(but not current) versions of the OpenSSL SSLv2 handshake code that
create even more powerful Bleichenbacher oracles, and drastically reduce the amount
of computation required to implement our attacks.  
The vulnerabilities, designated CVE-2016-0703 and CVE-2016-0704, were
present in the OpenSSL codebase from at least the start of the repository,
in 1998, until they were unknowingly fixed on March 4, 2015 by a
patch~\cite{openssl-clear-patch} designed to correct an unrelated
problem~\cite{CVE-2015-0293}.
By adapting DROWN to exploit this special case, we can significantly cut both
the number of connections and the computational work required.

%To distinguish the two, we call the attack developed so far
%\emph{general} DROWN and refer to the variant that
%exploits the OpenSSL bug as \emph{special} DROWN\@.  General DROWN is
%a protocol-level attack that makes few assumptions about the SSLv2
%server, other than that it allows export cipher handshakes.  Special
%DROWN exploits a specific implementation bug, but it is highly practical.
%It may be the first widespread Bleichenbacher vulnerability within reach of ``script-kiddies'' and other low-resource
%attackers.

%% This dramatic discovery came too late to fully incorporate into the
%% body of our submission, so for now we confine the bulk of the
%% discussion to this section.

% !TEX root = ../../../proposal.tex
\subsection{The OpenSSL ``extra clear'' oracle}

%\subsubsection{A key recovery attack on SSLv2 handshakes}

\label{sec:clear-key-vuln}

Prior to the fix, OpenSSL servers improperly allowed the \texttt{ClientMasterKey} message to contain
\texttt{clear\_key\_data} bytes for \emph{non-export} ciphers.  When such bytes are present,
the server substitutes them for bytes from the
encrypted key. For example, consider the case that the client chooses a 128-bit cipher and sends a 16-byte
encrypted key $k\pos{1}, k\pos{2}, \ldots, k\pos{16}$ but, contrary to the protocol specification, includes 4
null bytes of \texttt{clear\_key\_data}. Vulnerable OpenSSL versions will
construct the following \texttt{master\_key}:

\small $[00\ 00\ 00\ 00\ k\pos{1}\ k\pos{2}\ k\pos{3}\ k\pos{4}\ \dots\ k\pos{9}\ k\pos{10}\ k\pos{11}\ k\pos{12}]$\normalsize

This enables a straightforward key recovery attack against such versions.
An attacker that has intercepted an \ssltwo connection takes the RSA
ciphertext of the encrypted key and replays it in non-export handshakes to
the server with varying lengths of \texttt{clear\_key\_data}. For a 16-byte
encrypted key, the attacker starts with 15 bytes of clear key, causing the server to use the \texttt{master\_key}:

\small$[00\ 00\ 00\ 00\ 00\ 00\ 00\ 00\ 00\ 00\ 00\ 00\ 00\ 00\ 00\ k\pos{1}]$\normalsize

The attacker can brute force the first byte of the encrypted key by
finding the matching \texttt{ServerVerify} message among 256
possibilities. Knowing $k\pos{1}$, the attacker makes another
connection with the same RSA ciphertext but 14 bytes of clear key,
resulting in the \texttt{master\_key}:

\small $[00\ 00\ 00\ 00\ 00\ 00\ 00\ 00\ 00\ 00\ 00\ 00\ 00\ 00\ k\pos{1}\ k\pos{2}]$\normalsize

% Note: The number below is 15, since you the original intercepted connection
% servers as the case with 0 bytes of clear_key_data.
The attacker can now easily brute force $k\pos{2}$. With only 15 probe
connections and an expected $15 \cdot 128 = 1,920$
trial encryptions, the attacker learns the entire \texttt{master\_key} for the
recorded session.

%\subsubsection{An improved oracle for DROWN}

As this oracle is obtained by improperly sending unexpected clear-key bytes,
we call it the Extra Clear oracle.

This session key-recovery attack can be directly converted to a Bleichenbacher oracle. Given a candidate ciphertext and symmetric key length $\ell_k$, the attacker sends the ciphertext with $\ell_k$ known bytes of \texttt{clear\_key\_data}. The oracle decision is simple:
\begin{itemize}
\item If the ciphertext is valid, the \texttt{ServerVerify} message will reflect a \texttt{master\_key} consisting of those $\ell_k$ known bytes.
\item If the ciphertext is invalid, the \texttt{master\_key} will be replaced with $\ell_k$ random bytes (by following the countermeasure against the Bleichenbacher attack), resulting in a different \texttt{ServerVerify} message.
\end{itemize}

This oracle decision requires one connection to the server and one \texttt{ServerVerify} computation. After the attacker has found a valid ciphertext corresponding to a $\ell_k$-byte encrypted key, they recover the $\ell_k$ plaintext bytes by repeating the key recovery attack from above.  Thus our oracle $\OracleSSLclear(c)$ requires one connection to determine whether $c$ is valid.  After $\ell_k$ connections, the attacker additionally learns the $\ell_k$ least significant bytes of $m$.  We model this as a single oracle call, but the number of server connections will vary depending on the response.


%\tabDrownAll

\subsection{MITM attack against TLS}
\label{sec:special_mitm_tls}

Special DROWN is fast enough that it can decrypt a TLS premaster
secret \emph{online}, during a connection handshake.  A
man-in-the-middle attacker can use it to compromise connections
between modern browsers and TLS servers---even those configured to
prefer non-RSA cipher suites.

%\subsubsection{Attack scenario}
The MITM attacker impersonates the server and sends a
\texttt{ServerHello} message that selects a cipher suite with RSA as
the key-exchange method.  Then, the attacker uses special DROWN to
decrypt the \pms.  The main difficulty is completing the decryption and producing a valid
\texttt{ServerFinished} message before the client's connection times
out.  Most browsers will allow the handshake to last up to one minute~\cite{LogJam}.

The attack requires targeting
an average of 100 connections, only one of
which will be attacked, probabilistically.  The simplest
way for the attacker to facilitate this is to use JavaScript to cause
the client to connect repeatedly to the victim server, as described in
Section~\ref{sec:attack-scenario}.  Each connection is tested
against the oracle with only small number of fractions, and the attacker can discern
immediately when he receives a positive response from the oracle.

Note that since the decryption must be completed online, the Leaky Export oracle
cannot be used, and the attack uses only the Extra Clear oracle.

\subsubsection{Constructing the attack}
We will use \texttt{SSL\_DES\_192\_EDE3\_CBC\_WITH\_MD5} as the cipher suite, allowing the attacker to recover 24 bytes of key at a time.
The attack works as follows:

\begin{enumerate}
 \setcounter{enumi}{-1}
 \item The attacker causes the victim client to connect repeatedly to the
  victim server, with at least 100 connections.
 \item The attacker uses the fractional trimmers as described in
  Section~\ref{sec:trimmers} to convert one of the TLS ciphertexts
  into an \sslconform ciphertext $c_0$.
 \item Once the attacker has obtained a valid \ssltwo ciphertext $c_1$,
  they repeatedly use the shifting technique described in
  Section~\ref{sec:rotations} to rotate the message by 25 bytes each iteration,
  learning 27 bytes with each shift.  After several iterations, they have learned
  the entire plaintext.
 \item The attacker then transforms the decrypted \ssltwo plaintext into the decrypted TLS plaintext. 
\end{enumerate}

Using 100 fractional trimmers, this more efficient oracle attack allows
the attacker to recover one in 100 TLS session keys using only about
27,000 connections to the server, as described in
Appendix~\ref{sec:special-performance}.  The computation cost is so low that
we can complete the full attack on a single workstation in under one
minute.

\ifext
The ability of the victim server to perform 27,000 handshakes in less than a
minute is not an impediment for modern hardware.  An RSA
private key operation with a 2048-bit modulus requires on the order of
1~ms using OpenSSL on a recent-generation CPU, so the cryptographic
portion of the attacker's queries induces additional server load of
roughly 22~core-seconds.  In tests with a nearby server running Apache
2.4, we could easily complete 10,000 HTTPS requests in under 10
seconds.
\fi
