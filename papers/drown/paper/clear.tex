\label{sec:special}

We discovered a vulnerability in recent
(but not current) versions of the OpenSSL SSLv2 handshake code that
creates a powerful Bleichenbacher oracle, and drastically reduces the amount
of computation required to implement our attack.  
The vulnerability, which has been designated CVE-2016-0703, was
present in the OpenSSL codebase from at least the start of the repository,
in 1998, until it was unknowingly fixed on March 4, 2015 by a
patch~\cite{openssl-clear-patch} designed to correct an unrelated
problem~\cite{CVE-2015-0293}.
By adapting DROWN to
exploit this special case, we can cut the number of connections
required by more than 50\% and reduce the computational work to a negligible amount.

%To distinguish the two, we call the attack developed so far
%\emph{general} DROWN and refer to the variant that
%exploits the OpenSSL bug as \emph{special} DROWN\@.  General DROWN is
%a protocol-level attack that makes few assumptions about the SSLv2
%server, other than that it allows export cipher handshakes.  Special
%DROWN exploits a specific implementation bug, but it is highly practical.
%It may be the first widespread Bleichenbacher vulnerability within reach of ``script-kiddies'' and other low-resource
%attackers.

%% This dramatic discovery came too late to fully incorporate into the
%% body of our submission, so for now we confine the bulk of the
%% discussion to this section.

% !TEX root = ../../../proposal.tex
\subsection{The OpenSSL ``extra clear'' oracle}

%\subsubsection{A key recovery attack on SSLv2 handshakes}

\label{sec:clear-key-vuln}

Prior to the fix, OpenSSL servers improperly allowed the \texttt{ClientMasterKey} message to contain
\texttt{clear\_key\_data} bytes for \emph{non-export} ciphers.  When such bytes are present,
the server substitutes them for bytes from the
encrypted key. For example, consider the case that the client chooses a 128-bit cipher and sends a 16-byte
encrypted key $k\pos{1}, k\pos{2}, \ldots, k\pos{16}$ but, contrary to the protocol specification, includes 4
null bytes of \texttt{clear\_key\_data}. Vulnerable OpenSSL versions will
construct the following \texttt{master\_key}:

\small $[00\ 00\ 00\ 00\ k\pos{1}\ k\pos{2}\ k\pos{3}\ k\pos{4}\ \dots\ k\pos{9}\ k\pos{10}\ k\pos{11}\ k\pos{12}]$\normalsize

This enables a straightforward key recovery attack against such versions.
An attacker that has intercepted an \ssltwo connection takes the RSA
ciphertext of the encrypted key and replays it in non-export handshakes to
the server with varying lengths of \texttt{clear\_key\_data}. For a 16-byte
encrypted key, the attacker starts with 15 bytes of clear key, causing the server to use the \texttt{master\_key}:

\small$[00\ 00\ 00\ 00\ 00\ 00\ 00\ 00\ 00\ 00\ 00\ 00\ 00\ 00\ 00\ k\pos{1}]$\normalsize

The attacker can brute force the first byte of the encrypted key by
finding the matching \texttt{ServerVerify} message among 256
possibilities. Knowing $k\pos{1}$, the attacker makes another
connection with the same RSA ciphertext but 14 bytes of clear key,
resulting in the \texttt{master\_key}:

\small $[00\ 00\ 00\ 00\ 00\ 00\ 00\ 00\ 00\ 00\ 00\ 00\ 00\ 00\ k\pos{1}\ k\pos{2}]$\normalsize

% Note: The number below is 15, since you the original intercepted connection
% servers as the case with 0 bytes of clear_key_data.
The attacker can now easily brute force $k\pos{2}$. With only 15 probe
connections and an expected $15 \cdot 128 = 1,920$
trial encryptions, the attacker learns the entire \texttt{master\_key} for the
recorded session.

%\subsubsection{An improved oracle for DROWN}

As this oracle is obtained by improperly sending unexpected clear-key bytes,
we call it the Extra Clear oracle.

This session key-recovery attack can be directly converted to a Bleichenbacher oracle. Given a candidate ciphertext and symmetric key length $\ell_k$, the attacker sends the ciphertext with $\ell_k$ known bytes of \texttt{clear\_key\_data}. The oracle decision is simple:
\begin{itemize}
\item If the ciphertext is valid, the \texttt{ServerVerify} message will reflect a \texttt{master\_key} consisting of those $\ell_k$ known bytes.
\item If the ciphertext is invalid, the \texttt{master\_key} will be replaced with $\ell_k$ random bytes (by following the countermeasure against the Bleichenbacher attack), resulting in a different \texttt{ServerVerify} message.
\end{itemize}

This oracle decision requires one connection to the server and one \texttt{ServerVerify} computation. After the attacker has found a valid ciphertext corresponding to a $\ell_k$-byte encrypted key, they recover the $\ell_k$ plaintext bytes by repeating the key recovery attack from above.  Thus our oracle $\OracleSSLclear(c)$ requires one connection to determine whether $c$ is valid.  After $\ell_k$ connections, the attacker additionally learns the $\ell_k$ least significant bytes of $m$.  We model this as a single oracle call, but the number of server connections will vary depending on the response.


\tabDrownAll

\subsection{TLS decryption with special DROWN}
\label{sec:clear_analysis}

Using our oracle $\OracleSSLclear$, we can construct an extremely efficient version of our TLS decryption attack.  The OpenSSL \tOracleSSLclear provides three significant advantages over our export oracle $\OracleSSLexp$: (1) It no longer requires an export cipher suite, and, in fact, we gain efficiency by exploiting regular SSLv2 ciphers; (2) It requires only one handshake per oracle query; and (3) Computation is reduced to one \texttt{ServerVerify} decryption per oracle query, versus $2^{40}$.

\subsubsection{Attack scenario}

As before, we consider a server that accepts TLS connections, and a client that negotiates a secure, state-of-the-art TLS version with a \texttt{TLS\_RSA} cipher suite.  The same RSA key pair used for TLS is also used on a server that is running a vulnerable version of OpenSSL.

\subsubsection{Constructing the attack}

The attacker can exploit the OpenSSL extra clear vulnerability to efficiently decrypt a TLS ciphertext as follows.  We will use the cipher suite \texttt{SSL\_DES\_192\_EDE3\_CBC\_WITH\_MD5} as the cipher suite, allowing the attacker to recover 24 bytes of key at a time from the oracle.
We first present a straightforward adaptation of the general DROWN attack to the \tOracleSSLclear,
before later applying a few additional optimizations made possible by this new oracle.

\begin{enumerate}
 \setcounter{enumi}{-1}
 \item The attacker intercepts several hundred TLS handshakes using RSA key exchange.
 \item The attacker uses the fractional trimmers as described in Section~\ref{sec:trimmers} to convert the TLS ciphertexts into an \sslconform ciphertext $c_0$.
 \item Once the attacker has obtained a valid \ssltwo ciphertext $c_1$, he repeatedly uses the shifting technique described in Section~\ref{sec:rotations} to rotate the message by 25 bytes each iteration, learning 27 bytes with each shift.  After several iterations, he has learned the entire plaintext.
 \item The attacker then transforms the decrypted \ssltwo plaintext into the decrypted TLS plaintext. 
 \end{enumerate}

\paragraph{Attack costs}
Using 40 fractional trimmers, this more efficient oracle attack allows
the attacker to recover one in 260 TLS session keys using only about
17,000 connections to the server.  The computation cost is so low that
we can complete the full attack on a single workstation in under one
minute. Appendix~\ref{sec:special-performance} gives more details.

Mounting the attack using the optimized version of Special DROWN
described in Appendix~\ref{sec:special-performance} allows the
attacker to target one of 100 connections, at the expense of
increasing the number of queries to 27,000.

\subsection{MITM attack against TLS}

Special DROWN is fast enough that it can decrypt a TLS premaster
secret \emph{online}, during a connection handshake.  A
man-in-the-middle attacker can use it to compromise connections
between modern browsers and TLS servers---even those configured to
prefer non-RSA cipher suites.

\paragraph{Attack scenario.}
The MITM attacker impersonates the server and sends a
\texttt{ServerHello} message that selects a cipher suite with RSA as
the key-exchange method.  Then, the attacker uses special DROWN to
decrypt the \pms.  The main difficulty is completing the decryption and producing a valid
\texttt{ServerFinished} message before the client's connection times
out.  Most browsers will allow the handshake to last up to one minute~\cite{LogJam}.

Using the fully optimized version of special DROWN, the attack still requires intercepting 
an average of 100 ciphertexts, only one of
which will be decrypted, probabilistically.  The simplest
way for the attacker to facilitate this is to use JavaScript to cause
the client to connect repeatedly to the victim server, as described in
Section~\ref{sec:attack-scenario}.  Each connection is tested
against the oracle with only small number of fractions, and the attacker can discern
immediately when he receives a positive response from the oracle.

Once the attacker has obtained a positive response, he
can proceed to the final phase of the special DROWN attack described above, 
which employs 200-bit rotation 10 times to fully decrypt the
plaintext.   Our current implementation requires under
30 seconds for this phase on a single PC.

% AH: The 34 microsecond number for 2048-bit RSA doesn't seem right.
% Here's a trial with openssl on a fairly modern server:
%
%% $ openssl speed rsa
%% Doing 512 bit private rsa's for 10s: 176064 512 bit private RSA's in 10.00s
%% Doing 512 bit public rsa's for 10s: 2092137 512 bit public RSA's in 10.00s
%% Doing 1024 bit private rsa's for 10s: 53091 1024 bit private RSA's in 10.00s
%% Doing 1024 bit public rsa's for 10s: 779785 1024 bit public RSA's in 10.00s
%% Doing 2048 bit private rsa's for 10s: 7199 2048 bit private RSA's in 10.00s
%% Doing 2048 bit public rsa's for 10s: 234671 2048 bit public RSA's in 10.00s
%% Doing 4096 bit private rsa's for 10s: 1005 4096 bit private RSA's in 10.01s
%% Doing 4096 bit public rsa's for 10s: 63173 4096 bit public RSA's in 10.01s

% Nimrod: Let's go with your number then.
% Just for the record, I got the number from the QUIC Crypto document:
% https://docs.google.com/document/d/1g5nIXAIkN_Y-7XJW5K45IblHd_L2f5LTaDUDwvZ5L6g/edit#heading=h.bzxklo2i5w6k

The ability of the victim server to perform 17,000 handshakes in less than a
minute is not an impediment for modern hardware.  An RSA
private key operation with a 2048-bit modulus requires on the order of
1~ms using OpenSSL on a recent-generation CPU, so the cryptographic
portion of the attacker's queries induces additional server load of
roughly 14~core-seconds.  In tests with a nearby server running Apache
2.4, we could easily complete 10,000 HTTPS requests in under 10
seconds.

% AH: Here's further data in support of this fact, tested against a local
% server.  10k TLS/RSA connections took about 7 seconds to complete.

%% $ ab -n 10000 -c 100 -Z AES128-SHA https://www.eecs.umich.edu/x
%% This is ApacheBench, Version 2.3 <$Revision: 1528965 $>
%% Copyright 1996 Adam Twiss, Zeus Technology Ltd, http://www.zeustech.net/
%% Licensed to The Apache Software Foundation, http://www.apache.org/

%% Benchmarking www.eecs.umich.edu (be patient)

%% Server Software:        Apache/2.2.15
%% Server Hostname:        www.eecs.umich.edu
%% Server Port:            443
%% SSL/TLS Protocol:       TLSv1.2,AES128-SHA,2048,128

%% Document Path:          /x
%% Document Length:        285 bytes

%% Concurrency Level:      100
%% Time taken for tests:   7.130 seconds
%% Complete requests:      10000
%% Failed requests:        0
%% Non-2xx responses:      10000
%% Total transferred:      4660000 bytes
%% HTML transferred:       2850000 bytes
%% Requests per second:    1402.45 [#/sec] (mean)
%% Time per request:       71.304 [ms] (mean)
%% Time per request:       0.713 [ms] (mean, across all concurrent requests)
%% Transfer rate:          638.22 [Kbytes/sec] received

%% Connection Times (ms)
%%               min  mean[+/-sd] median   max
%% Connect:       28   57  47.7     52    1069
%% Processing:     7   13   7.2     13     234
%% Waiting:        7   12   7.2     12     230
%% Total:         47   70  48.1     65    1082





