% !TEX root = ../../../proposal.tex
\label{sec:scans}

We performed Internet-wide scans to analyze the number of systems vulnerable to
DROWN\@. A host is directly vulnerable to general DROWN if it supports \ssltwo.
Similarly, a host is directly vulnerable to special DROWN if it supports
\ssltwo and has the extra clear bug (which also implies the leaky export bug).
These directly vulnerable hosts can be
used as oracles to attack any other host with the same key. Hosts that do not
support \ssltwo are still vulnerable to general or special DROWN if their RSA
key pair is exposed by any general or special DROWN oracle, respectively. The
oracles may be on an entirely different host or port.  Additionally, any host
serving a browser-trusted certificate is vulnerable to a special DROWN
man-in-the-middle if any name on the certificate appears on any other
certificate containing a key that is exposed by a special DROWN oracle.

We used ZMap~\cite{zmap-2013} to perform full IPv4 scans on eight different ports
during late January and February 2016.  We examined port 443 (HTTPS), and
common email ports 25 (SMTP with STARTTLS), 110 (POP3 with STARTTLS), 143 (IMAP
with STARTTLS), 465 (SMTPS), 587 (SMTP with STARTTLS), 993 (IMAPS), and 995
(POP3S).  For each open port, we attempted three complete handshakes: one
normal handshake with the highest available SSL/TLS version; one \ssltwo
handshake requesting an export RC2 cipher suite; and one \ssltwo handshake with
a non-export cipher and sixteen bytes of plaintext key material sent during key
exchange, which we used to detect if a host has the extra clear bug.

We summarize our general DROWN results in Table~\ref{table:general}. The
fraction of SSL/TLS hosts that directly supported \ssltwo varied substantially
across ports. 28\% of SMTP servers on port 25 supported \ssltwo, likely due to
the opportunistic encryption model for email transit. Since SMTP fails-open to
plaintext, many servers are configured with support for the largest possible
set of protocol versions and cipher suites, under the assumption that even bad
or obsolete encryption is better than plaintext~\cite{better-crypto}. The other
email ports ranged from 8\% for SMTPS to 20\% for POP3S and IMAPS. We found
17\% of all HTTPS servers, and 10\% of those with a browser-trusted
certificate, are directly vulnerable to general DROWN\@.

\tabSpecialAll


\paragraph{OpenSSL SSLv2 cipher suite selection bug.}
\label{sec:openssl-selection}

We discovered that OpenSSL servers do not respect the cipher suites advertised
in the \ssltwo \texttt{ServerHello} message. That is, a malicious client can
select an \textit{arbitrary} cipher suite in the \texttt{ClientMasterKey}
message, regardless of the contents of the \texttt{ServerHello}, and force the
use of export cipher suites even if they are explicitly disabled in the server
configuration.  To fully detect \ssltwo oracles, we configured our scanner to
ignore the \texttt{ServerHello} cipher list. The cipher selection bug helps
explain the wide support for \ssltwo---the protocol appeared disabled, but 
non-standard clients could still complete handshakes.

%(this was not necessarily the case when OpenSSL was used as a plugin in Apache or other webservers).
%In addition to verifying this vulnerability in our lab, we have encountered several SSLv2 servers on the Internet which have apparently disabled export cipher suites (as judged by their \texttt{ServerHello} message), where we could indeed force the use of these cipher suites on those servers.

%In addition, these versions by default disabled \ssltwo support.

%\todo{Mention POP3 is likely vulnerable without any active attacks involving the client, since we expect to have a handshake every few minutes anyway}


\paragraph{Widespread public key reuse.}
Reuse of RSA key material across hosts and certificates is
widespread~\cite{mail-tls-holz-2016,weak-keys-2012}. Often this is benign:
organizations may issue multiple TLS certificates for distinct domains with
the same public key in order to simplify use of TLS acceleration hardware and
load balancing. However, there is also evidence that system administrators
may not entirely understand the role of the public key in certificates. For
example, in the wake of the Heartbleed vulnerability, a substantial fraction
of compromised certificates were reissued with the same public
key~\cite{heartbleed-2014}.

There are many reasons why the same public key or certificate would be reused
across different ports and services within an organization. For example a
mail server that serves SMTP, POP3, and IMAP from the same daemon would
likely share the same TLS configuration. Additionally, an organization might
choose to purchase a single wildcard TLS certificate, and use it on both web
servers and mail servers. Public keys have also been observed to be widely
shared across independent organizations due to default certificates and
public keys that are shipped with networked devices and software, improperly
configured virtual machine images, and random number generation flaws.

The number of hosts vulnerable to DROWN rises significantly when we take RSA
key reuse into account. For HTTPS, 17\% of hosts are vulnerable to general
DROWN because they support both TLS and \ssltwo on the HTTPS port, but 33\%
are vulnerable when considering RSA keys used by another service.

\paragraph{Special DROWN\@.}
As shown in Table~\ref{table:special},
9.1\,M HTTPS servers (26\%) are
vulnerable to special DROWN, as are 2.5\,M HTTPS servers with browser-trusted
certificates~(14\%). 66\% as many HTTPS hosts are vulnerable to special DROWN
as to general DROWN\@ (70\% for browser-trusted servers). While 2.7\,M public
keys are vulnerable to general DROWN, only 1.1\,M are vulnerable to special DROWN
(41\% as many). Vulnerability among Alexa Top Million domains is also lower, with
only 9\% of domains vulnerable (7\% for browser-trusted domains).

Since special DROWN enables active man-in-the-middle attacks, any host serving
a browser-trusted certificate with at least one name that appears on any
certificate with an RSA key exposed by a special DROWN oracle is vulnerable to an
impersonation attack. Extending our search to account for certificates with
shared names, we find that 3.8\,M~(22\%) hosts with browser-trusted certificates
are vulnerable to man-in-the-middle attacks, as well as 19\% of the
browser-trusted domains in the Alexa Top Million.
