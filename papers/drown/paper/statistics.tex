\label{sec:scans}

\tabSpecialAll

We performed Internet-wide scans to analyze the number of systems vulnerable to
DROWN\@. A host is directly vulnerable to general DROWN if it supports \ssltwo.
Similarly, a host is directly vulnerable to special DROWN if it supports
\ssltwo and has the extra clear bug. These directly vulnerable hosts can be
used as oracles to attack any other host with the same key. Hosts that do not
support \ssltwo are still vulnerable to general or special DROWN if their RSA
key pair is exposed by any general or special DROWN oracle, respectively. The
oracles may be on an entirely different host or port.  Additionally, any host
serving a browser-trusted certificate is vulnerable to a special DROWN
man-in-the-middle if any name on the certificate appears on any other
certificate containing a key that is exposed by a special DROWN oracle.

We used ZMap~\cite{zmap13} to perform full IPv4 scans on eight different ports
during late January and February 2016.  We examined port 443 (HTTPS), and
common email ports 25 (SMTP with STARTTLS), 110 (POP3 with STARTTLS), 143 (IMAP
with STARTTLS), 465 (SMTPS), 587 (SMTP with STARTTLS), 993 (IMAPS), and 995
(POP3S).  For each open port, we attempted three complete handshakes: one
normal handshake with the highest available SSL/TLS version; one \ssltwo
handshake requesting an export RC2 cipher suite; and one \ssltwo handshake with
a non-export cipher and sixteen bytes of plaintext key material sent during key
exchange, which we used to detect if a host has the extra clear bug.

We summarize our general DROWN results in Table~\ref{table:general}. The
fraction of SSL/TLS hosts that directly supported \ssltwo varied substantially
across ports. 28\% of SMTP servers on port 25 supported \ssltwo, likely due to
the opportunistic encryption model for email transit. Since SMTP fails-open to
plaintext, many servers are configured with support for the largest possible
set of protocol versions and cipher suites, under the assumption that even bad
or obsolete encryption is better than plaintext~\cite{better-crypto}. The other
email ports ranged from 8\% for SMTPS to 20\% for POP3S and IMAPS. We found
17\% of all HTTPS servers, and 10\% of those with a browser-trusted
certificate, are directly vulnerable to General DROWN\@.

\paragraph{Widespread public key reuse.}
Reuse of RSA key material across hosts and certificates is widespread, as has
been documented
in~\cite{2016holz_analysis_tls-based_protocols_electronic_communication,DBLP:journals/corr/MayerZSH15}. In many
cases this is benign: many organizations issue multiple TLS certificates for
distinct domains (e.g.~one for each TLD) with the same public key; reusing the
same key simplifies the use of SSL acceleration hardware and load balancing.
However, there is also evidence that system administrators may not entirely
understand the role of the public key in certificates. For example, in the wake
of the Heartbleed vulnerability, a substantial fraction of compromised
certificates were reissued with the same public
key~\cite{Durumeric:2014:MH:2663716.2663755}.  

There are many reasons why the same public key or certificate would be reused
across different ports and services within an organization. For example a mail
server that serves SMTP, POP3, and IMAP from the same daemon would likely share
the same TLS configuration.  Additionally, an organization might choose to
purchase a single wildcard TLS certificate, and use it on both web servers and
mail servers. Public keys have also been observed to be widely shared across
independent organizations due to default certificates and public keys that are
shipped with networked devices and software, improperly configured virtual
machine images, and random number generation flaws.

The number of hosts vulnerable to DROWN rises significantly when we take RSA
key reuse into account. For HTTPS, 17\% of hosts are vulnerable to general
DROWN because they support both TLS and \ssltwo on the HTTPS port, but the
number of vulnerable hosts rises to 33\% when considering RSA keys used by
another service that is vulnerable to DROWN\@. Appendix~\ref{sec:pub_key_reuse}
gives more detailed statistics on the reuse of RSA key material across hosts
and ports.

\paragraph{Special DROWN\@.}
As shown in Table~\ref{table:special}, 9.1\,M HTTPS servers (26\%) are
vulnerable to special DROWN, as are 2.5\,M HTTPS servers with browser-trusted
certificates~(14\%). 66\% as many HTTPS hosts are vulnerable to special DROWN
as to general DROWN\@ (70\% for browser-trusted servers). While there are 
2.7\,M public keys that are vulnerable to general DROWN, we find 1.1\,M
vulnerable to special DROWN (41\% as many). Vulnerability among Alexa Top Million
domains is lower, with only 9\% of Alexa domains vulnerable (7\% for
browser-trusted domains).

Since special DROWN enables active man-in-the-middle attacks, any host serving
a browser-trusted certificate with at least one name that appears on any
certificate with a key exposed by a special DROWN oracle is vulnerable to a
impersonation attacks. Extending our search to account for shared names, we
find 3.8\,M~(22\%) of hosts with browser-trusted certificates are vulnerable to
man-in-the-middle, as well as 19\% of the browser-trusted Alexa Top Million.
