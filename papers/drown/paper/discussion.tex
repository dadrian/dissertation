% !TEX root = ../../../proposal.tex
\ifext
\subsection{Lessons for protocol design}
A natural question is to ask whether SSLv3 or later versions of TLS could also be vulnerable.
Our attack exploits two properties of the \ssltwo protocol:

\paragraph{Server authenticates first.} 
First, the fact that in \ssltwo the server responds to the \texttt{ClientMasterKey} message before the client proves it has knowledge of the RSA plaintext, provides a direct message side channel. In SSLv3 and later, the client must demonstrate knowledge of the RSA plaintext first via a valid \texttt{ClientFinished} message before the server sends a message derived from the RSA plaintext.  In order to perform a similar attack in this case, the client would need to perform an online brute-force attack\ifext, significantly increasing the workload\fi.

We characterize this behavior of \ssltwo as a protocol vulnerability and not an implementation vulnerability, although this behavior is not rigorously determined by the standard itself.  The standard's presentation of message ordering is contradictory: the prose states that the \texttt{ServerVerify} message is sent immediately after the server receives the \texttt{ClientMasterKey} message, while the diagrams in Section 5.2, "Typical Protocol Message Flow", depict the server waiting for \texttt{ClientFinished} message before sending its own \texttt{ServerVerify}.  The three widely-used implementations of the protocol that we examined, OpenSSL, Microsoft IIS, and NSS, all took the former interpretation, and responded immediately with a \texttt{ServerVerify} message after the \texttt{ClientMasterKey}, rendering them vulnerable in this respect.

\paragraph{Short secrets.} Second, \ssltwo allows RSA plaintexts that are short enough to be vulnerable to a feasible-time brute force search.  For export ciphers, the unpadded RSA plaintext is five bytes long.  In SSLv3 and later versions of TLS, the RSA plaintexts and \pms length is 48 bytes, even for export ciphers with 40-bit strength.  For later protocol versions, an attacker can perform a brute-force search over the derived 40-bit key if a client negotiates an export cipher suite, but the 48-byte \pms length appears to prevent an attacker from escalating the weakness of the export cipher strength into a similar protocol vulnerability.
\fi

\subsection{Implications for modern protocols}
Although the protocol flaws in SSLv2 enabling DROWN are not present in recent TLS versions, many modern protocols meet a subset of the requirements to be vulnerable to a DROWN-style attack. For example:
\begin{enumerate}
	\item RSA key exchange. TLS 1.2~\cite{rfc5246} allows this.
	\item Reuse of server-side nonce by the client. QUIC~\cite{quic-langley-2014} allows this.
	\item Server sends a message encrypted with the derived key before the client. QUIC, TLS 1.3~\cite{rfc8446}, and TLS False Start~\cite{rfc7918} do this.
	\item Deterministic cipher parameters are generated from the \pms and nonces. This is the case for all TLS stream ciphers and TLS 1.0 block ciphers.
\end{enumerate}

\if0
When all three properties are combined, a natural adaptation of our attack presents itself.
The attacker obtains a Bleichenbacher oracle by connecting to the server twice with the same RSA ciphertext and the same server-side nonce, and comparing the messages sent by the server.
If the RSA ciphertext is PKCS conformant, the two messages will be identical.
Otherwise, they will differ.
Note that we also assumed that all symmetric cipher parameters, including IVs for block ciphers, are deterministically generated from the \pms and nonces; this is the case for TLS 1.0.
If that is not the case, the attacker can choose a stream cipher.
\fi

DROWN has a natural adaptation when all three properties are present. The attacker exposes a Bleichenbacher oracle by connecting to the server twice with the identical RSA ciphertexts and server-side nonces. If the RSA ciphertext is PKCS conformant, the server will respond with identical messages across both connections; otherwise they will differ.

\looseness=-1

\ifext
% NA: I'm pretty sure the following argument is wrong, after discussing this with Juraj.
% Basically, even if the attacker correctly guesses the encryption and MAC keys,
% he needs to also guess the *contents* of the ClientFinished message.
% The length of this content is independent of the chosen MAC and cipher, and is usually unfeasibly long to guess.
%
% Even if the third property above does not hold, an attacker may reduce the session strength to the weakest symmetric cipher plus the weakest MAC supported by the server.
% The attacker proceeds as follows:
% \begin{itemize}
%	\item Choose arbitrary server and client nonces, $r_s$ and $r_c$\ifext, which will be used throughout the attack\fi.
	%\item Connect to the server with $c$ as the RSA ciphertext, $r_s$ and $r_c$ as the nonces, and choose the symmetric cipher and MAC with the smallest key size out of those supported. Denote these key sizes $L_1$ and $L_2$ respectively.
%	\item Generate random symmetric encryption and MAC keys of these sizes, denoted by $k_1$ and $k_2$ respectively, and hope they are identical to the correct keys computed by the server.  The probability that both $k_1$ and $k_2$ are identical to the correct keys is $2^{-(L_1 + L_2)}$.
%	\item Send a \texttt{Finished} message encrypted and MACed using $k_1$ and $k_2$.
%	If the RSA ciphertext was valid, the same keys $k_1$ and $k_2$ will produce two identical \texttt{ServerFinished} messages in two TLS handshakes. Otherwise, the two \texttt{ServerFinished} message will be invalid.
%\end{itemize}
\fi

\if0
% Really not sure what this has to do with anything...
An attacker can use False Start to cause a victim client to perform TLS handshakes using RSA for key exchange\ifext, and send secret application layer data after these handshakes\fi, even if the server supports other key exchange methods which provide Perfect Forward Secrecy. The attacker masquerades as the server and indicates support for RSA key exchange only. The client will then handshake using RSA, and send application layer data, before the server authenticates by sending the \texttt{Finished} message. The False Start standard indeed discourages the use of RSA for key exchange, but does not explicitly forbid it, leaving the security of the protocol dependent on correct choices in the client configuration. Our attacks show that relying on such assumptions is extremely brittle protocol design.
% nothing in DROWN is a client flaw...
\fi

\subsection{Lessons for key reuse}

DROWN illustrates the cryptographic principle that keys should be single use.
Often, this principle is primarily applied to keys that are used to both sign
and decrypt, but DROWN illustrates that using keys \emph{for different protocol
versions} can also be a serious security risk.
Unfortunately, there is no widely supported way to pin X.509 certificates to specific
protocols. While using per-protocol certificates may help defend against
passive attacks, an active attacker could still leverage any certificate with a
matching name.

\subsection{Harms from obsolete cryptography}

Recent years have seen a significant number of serious attacks exploiting
outdated and obsolete cryptography. Many protocols and cryptographic primitives
that were demonstrated to be weak decades ago are surprisingly common in
real-world systems.

% BEAST Lucky13 TLS truncation

DROWN exploits a modification of an 18-year-old attack against a combination of protocols and ciphers that have long been superseded by better options: the \ssltwo protocol, export cipher suites, and PKCS \#1 v1.5 RSA padding. In fact, support for RSA as a key exchange method, including the use of PKCS \#1 v1.5, is mandatory even for TLS 1.2. The attack is made more severe by implementation flaws in rarely used code.

Our work serves as yet another reminder of the importance of removing
deprecated technologies before they become exploitable vulnerabilities. In
response to many of the vulnerabilities listed above, browser vendors have
been aggressively warning end users when TLS connections are negotiated with
unsafe cryptographic parameters, including SHA-1 certificates, small RSA and
Diffie-Hellman parameters, and SSLv3 connections. This process is currently
happening in a piecemeal fashion, primitive by primitive. Vendors and
developers rightly prioritize usability and backward compatibility in
standards, and are willing to sacrifice these only for practical attacks.
This approach works less well for cryptographic vulnerabilities, where the
first sign of a weakness, while far from being practically exploitable, can
signal trouble in the future. Communication issues between academic
researchers and vendors and developers have been voiced by many in the
community, including Green~\cite{green-2015} and Jager
et~al.\@~\cite{jager-2013}.

The long-term solution is to proactively remove these obsolete technologies.
There is movement towards this already: TLS 1.3 has entirely removed RSA key
exchange and has restricted Diffie-Hellman key exchange to a few groups large
enough to withstand cryptanalytic attacks long in the future. The CA/Browser
forum will remove support for SHA-1 certificates this year. Resources such as
the SSL Labs SSL Reports have gathered information about best practices and
vulnerabilities in one place, in order to encourage administrators to make
the best choices.

\subsection{Harms from weakening cryptography}

Export-grade cipher suites for TLS deliberately weakened three primitives to
the point that they are now broken even to enthusiastic amateurs: 512-bit RSA
key exchange, 512-bit Diffie-Hellman key exchange, and 40-bit symmetric
encryption. All three deliberately weakened primitives have been cornerstones
of high-profile attacks: FREAK exploits export RSA, Logjam exploits export
Diffie-Hellman, and now DROWN exploits export symmetric encryption.

Like FREAK and Logjam, our results illustrate the continued harm that a
legacy of deliberately weakened export-grade cryptography inflicts on the
security of modern systems, even decades after the regulations influencing
the original design were lifted. The attacks described in this paper are
fully feasible against export cipher suites today. The technical debt induced
by cryptographic ``front doors'' has left implementations vulnerable for
decades. With the slow rate at which obsolete protocols and primitives fade
away, we can expect some fraction of hosts to remain vulnerable for years to
come.

