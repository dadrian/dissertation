\subsection{The OpenSSL ``extra clear'' oracle}

%\subsubsection{A key recovery attack on SSLv2 handshakes}

\label{sec:clear-key-vuln}

Prior to the fix, OpenSSL servers improperly allowed the \texttt{ClientMasterKey} message to contain
\texttt{clear\_key\_data} bytes for \emph{non-export} ciphers.  When such bytes are present,
the server substitutes them for bytes from the
encrypted key. For example, consider the case that the client chooses a 128-bit cipher and sends a 16-byte
encrypted key $k\pos{1}, k\pos{2}, \ldots, k\pos{16}$ but, contrary to the protocol specification, includes 4
null bytes of \texttt{clear\_key\_data}. Vulnerable OpenSSL versions will
construct the following \texttt{master\_key}:\smallskip

\small\noindent\quad\quad $[00\ 00\ 00\ 00\ k\pos{1}\ k\pos{2}\ k\pos{3}\ k\pos{4}\ \dots\ k\pos{9}\ k\pos{10}\ k\pos{11}\ k\pos{12}]$\normalsize\vspace{-12pt}
\smallskip

This enables a straightforward key recovery attack against such versions.
An attacker that has intercepted an \ssltwo connection takes the RSA
ciphertext of the encrypted key and replays it in non-export handshakes to
the server with varying lengths of \texttt{clear\_key\_data}. For a 16-byte
encrypted key, the attacker starts with 15 bytes of clear key, causing the server to use the \texttt{master\_key}:
\smallskip

\small\noindent\quad\quad $[00\ 00\ 00\ 00\ 00\ 00\ 00\ 00\ 00\ 00\ 00\ 00\ 00\ 00\ 00\ k\pos{1}]$\normalsize
\smallskip

The attacker can brute force the first byte of the encrypted key by
finding the matching \texttt{ServerVerify} message among 256
possibilities. Knowing $k\pos{1}$, the attacker makes another
connection with the same RSA ciphertext but 14 bytes of clear key,
resulting in the \texttt{master\_key}:
\smallskip

\small\noindent\quad\quad $[00\ 00\ 00\ 00\ 00\ 00\ 00\ 00\ 00\ 00\ 00\ 00\ 00\ 00\ k\pos{1}\ k\pos{2}]$\normalsize
\smallskip

% Note: The number below is 15, since you the original intercepted connection
% servers as the case with 0 bytes of clear_key_data.
The attacker can now easily brute force $k\pos{2}$. With only 15 probe
connections and an expected $15 \cdot 128 = 1,920$
trial encryptions, the attacker learns the entire \texttt{master\_key} for the
recorded session.

%\subsubsection{An improved oracle for DROWN}

As this oracle is obtained by improperly sending unexpected clear-key bytes,
we call it the Extra Clear oracle.

This session key-recovery attack can be directly converted to a Bleichenbacher oracle. Given a candidate ciphertext and symmetric key length $\ell_k$, the attacker sends the ciphertext with $\ell_k$ known bytes of \texttt{clear\_key\_data}. The oracle decision is simple:
\begin{itemize}
\item If the ciphertext is valid, the \texttt{ServerVerify} message will reflect a \texttt{master\_key} consisting of those $\ell_k$ known bytes.
\item If the ciphertext is invalid, the \texttt{master\_key} will be replaced with $\ell_k$ random bytes (by following the countermeasure against the Bleichenbacher attack), resulting in a different \texttt{ServerVerify} message.
\end{itemize}

This oracle decision requires one connection to the server and one \texttt{ServerVerify} computation. After the attacker has found a valid ciphertext corresponding to a $\ell_k$-byte encrypted key, they recover the $\ell_k$ plaintext bytes by repeating the key recovery attack from above.  Thus our oracle $\OracleSSLclear(c)$ requires one connection to determine whether $c$ is valid.  After $\ell_k$ connections, the attacker additionally learns the $\ell_k$ least significant bytes of $m$.  We model this as a single oracle call, but the number of server connections will vary depending on the response.
