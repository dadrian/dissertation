% !TEX root = ../../../proposal.tex
A flaw was first observed in the MD5 hash function in 1996; the first
collision was discovered in 2004~\cite{Wang:2005:BMO:2154598.2154601}, but
MD5 was still in use by certificate authorities in 2009 when Stevens et
al.~\cite{Stevens2009} used a chosen-prefix MD5 attack to construct a
malicious TLS certificate with a valid CA signature. The RC4 stream cipher
was observed to be biased as early as 1995 and shown to be catastrophically
broken in the context of WEP in 2001~\cite{Fluhrer2001}; it was used by about
50\% of TLS connections in 2013 when AlFardan et~al.\ demonstrated
near-practical attacks against RC4 in TLS~\cite{RC4biases}. TLSv1.0 was
standardized in 1998 to replace SSLv3; before the POODLE attack~\cite{POODLE}
was shown to render all SSLv3 block cipher suites insecure in 2014, support
for SSLv3 was near 100\% for popular HTTPS sites, and most clients were
vulnerable to a downgrade attack from TLS to SSLv3~\cite{ssllabs}.
Export-grade cipher suites for TLS have been obsolete since 2000, when the
United States relaxed restrictions on commercial and open source software;
before the FREAK attack~\cite{SMACKTLS} demonstrated widespread
implementation flaws allowing a catastrophic downgrade attack exploiting
export RSA, 37\% of HTTPS sites with browser-trusted certificates supported
export-grade RSA\@. Three months later the Logjam attack~\cite{logjam-2015}
demonstrated a TLS protocol flaw downgrade attack exploiting export
Diffie-Hellman; 8.4\% of the Alexa top million sites were vulnerable at the
time.

