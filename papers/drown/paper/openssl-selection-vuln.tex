\paragraph{OpenSSL SSLv2 cipher suite selection bug.}
\label{sec:openssl-selection}

We discovered that OpenSSL servers do not respect the cipher suites advertised
in the \ssltwo \texttt{ServerHello} message. That is, a malicious client can
select an \textit{arbitrary} cipher suite in the \texttt{ClientMasterKey}
message, regardless of the contents of the \texttt{ServerHello}, and force the
use of export cipher suites even if they are explicitly disabled in the server
configuration.  To fully detect \ssltwo oracles, we configured our scanner to
ignore the \texttt{ServerHello} cipher list. The cipher selection bug helps
explain the wide support for \ssltwo---the protocol appeared disabled, but 
non-standard clients could still complete handshakes.

%(this was not necessarily the case when OpenSSL was used as a plugin in Apache or other webservers).
%In addition to verifying this vulnerability in our lab, we have encountered several SSLv2 servers on the Internet which have apparently disabled export cipher suites (as judged by their \texttt{ServerHello} message), where we could indeed force the use of these cipher suites on those servers.

%In addition, these versions by default disabled \ssltwo support.

%\todo{Mention POP3 is likely vulnerable without any active attacks involving the client, since we expect to have a handshake every few minutes anyway}
