\subsection{OpenSSL SSLv2 cipher suite selection bug}
\label{sec:openssl-selection}
The \ssltwo protocol is supported in OpenSSL by default in all versions under 1.1.0.
OpenSSL removed \ssltwo cipher suites from the default cipher string in 2010 between versions 0.9.8n and 1.0.0; the changelog discusses this as being equivalent to disabling support for \ssltwo by default~\cite{opensslchangelog}.   Unfortunately, during our experiments we discovered that OpenSSL servers do not respect the cipher suites advertised in the \texttt{ServerHello} message.
That is, the client can select an \textit{arbitrary} cipher suite in the \texttt{ClientMasterKey} message and force the use of export cipher suites even if they are explicitly disabled in the server configuration.
The \ssltwo protocol itself was still enabled by default in the OpenSSL standalone server for the most recent OpenSSL versions prior to our disclosure. %(this was not necessarily the case when OpenSSL was used as a plugin in Apache or other webservers).
%In addition to verifying this vulnerability in our lab, we have encountered several SSLv2 servers on the Internet which have apparently disabled export cipher suites (as judged by their \texttt{ServerHello} message), where we could indeed force the use of these cipher suites on those servers.

We notified the OpenSSL team of this vulnerability, which was assigned CVE ID CVE-2015-3197. We have cooperated to develop a fix, which was included in OpenSSL releases 1.0.2f and 1.0.1r~\cite{opensslchangelog}. 
%In addition, these versions by default disabled \ssltwo support.

%\todo{Mention POP3 is likely vulnerable without any active attacks involving the client, since we expect to have a handshake every few minutes anyway}
