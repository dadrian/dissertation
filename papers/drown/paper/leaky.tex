%In addition to the \tOracleSSLclear $\OracleSSLclear$ described in Section~\ref{sec:special},
%the same set of OpenSSL versions allow a different oracle,
%which we term \tOracleSSLleaky $\OracleSSLleaky$.
%This additional oracle is powerful in a different way than
%\tOracleSSLclear --- it still requires roughly $2^{40}$ offline computations
%per query, but it is more permissive when checking for conformant messages.

\subsection{The OpenSSL ``leaky export'' oracle}
In addition to the extra clear implementation bug, the same set of OpenSSL versions
also contain a separate bug, where they do not follow the correct algorithm
for their implementation of the Bleichenbacher countermeasure.
We now describe this faulty implementation:
\begin{itemize}
	\item The \ssltwo \texttt{ClientKeyExchange} message contains the
	$mk_{clear}$ bytes immediately before the ciphertext $c$. Let $p$
	be the buffer starting at the first $mk_{clear}$ byte.

	\item Decrypt $c$ in place. If the decryption operation succeeds,
	and $c$ decrypted to a plaintext of a correct padded length,
	$p$ now contains the 11 $mk_{clear}$ bytes followed by the 5
	$mk_{secret}$ bytes.

	\item If $c$ decrypted to an unpadded plaintext $k$ of incorrect length,
	the decryption operation overwrites the first $j = min(|k|, 5)$ bytes
	of $c$ with the first $j$ bytes of $k$.

	\item If $c$ is not \sslconform and the decryption operation failed,
	randomize the first five bytes of $p$, which are the first
	five bytes of $mk_{clear}$.
\end{itemize}

This behavior allows the attacker to distinguish between these three cases.
Suppose the attacker sends 11 null bytes as $mk_{clear}$.
Then these are the possible cases:

\begin{enumerate}
\item $c$ decrypts to a correctly padded plaintext $k$ of the expected length, 5
	bytes. Then the following \texttt{master\_key} will be constructed:\smallskip\small\\
	$[00\ 00\ 00\ 00\ 00\ 00\ 00\ 00\ 00\ 00\ 00\ k\pos{1}\ k\pos{2}\ k\pos{3}\ k\pos{4}\ k\pos{5}]$
	\normalsize
\item $c$ decrypts to a correctly padded plaintext $k$ of a wrong length.
	Let $r$ be the five random bytes the server generated.
	The yielded \texttt{master\_key} will be:\smallskip\small\\
	\hspace*{-12pt}$[r\pos{1}\ r\pos{2}\ r\pos{3}\ r\pos{4}\ r\pos{5}\ 00\ 00\ 00\ 00\ 00\ 00\ k\pos{1}\ k\pos{2}\ k\pos{3}\ k\pos{4}\ k\pos{5}]$\medskip\\
	\normalsize
    when $|k| \ge 5$. If $|k| < 5$, the server substitutes the
	first $|k|$ bytes of $c$
	with the first $|k|$ bytes of $k$.
	Using $|k| = 3$ as an example, 
	the \texttt{master\_key} will be:\smallskip\small\\
	\hspace*{-12pt}$[r\pos{1}\ r\pos{2}\ r\pos{3}\ r\pos{4}\ r\pos{5}\ 00\ 00\ 00\ 00\ 00\ 00\ k\pos{1}\ k\pos{2}\ k\pos{3}\ c\pos{4}\ c\pos{5}]$\normalsize\vspace{-11pt}
%\ifext	where $c\pos{1}, \ldots, c\pos{5}$ denote the first five bytes of the RSA ciphertext $c$. \fi
\item $c$ is not \sslconform, and hence the decryption operation failed.
	The resulting \texttt{master\_key} will be:\medskip\small\\
	\hspace*{-12pt}$[r\pos{1}\ r\pos{2}\ r\pos{3}\ r\pos{4}\ r\pos{5}\ 00\ 00\ 00\ 00\ 00\ 00\ c\pos{1}\ c\pos{2}\ c\pos{3}\ c\pos{4}\ c\pos{5}]$
\end{enumerate}
The attacker detects case (3) by performing an exhaustive search over the
$2^{40}$ possibilities for $r$, and checking whether any of the resulting
values for the \texttt{master\_key} correctly decrypts the observed
\texttt{ServerVerify} message. If no $r$ value satisfies this property, then
$c^d$ starts with bytes \hexb{00}{02}. The attacker then distinguishes between
cases (1) and (2) by performing an exhaustive search over the five bytes of $k$,
and checking whether any of the resulting values for $mk$ correctly 
decrypts the observed \texttt{ServerVerify} message.

As this oracle leaks information when using export ciphers,
we have named it the Leaky Export oracle.

In conclusion, $\OracleSSLleaky$ allows an attacker to obtain a valid oracle response
for all ciphertexts which decrypt to a correctly-padded plaintext of \textit{any} length. This is in contrary to the previous oracles $\OracleSSLclear$ and $\OracleSSLexp$, which required the plaintext to be of a specific length.
Each oracle query to $\OracleSSLleaky$ requires one connection to the server
and $2^{41}$ offline work.


\paragraph{Combining the two oracles.}
\label{sec:special_drown_summary}

The attacker can use the Extra Clear and Leaky Export oracles
together in order to reduce the number of queries required for the TLS decryption attack.
They first test a \tlsconform ciphertext for divisors using the Leaky Export oracle, then use fractions dividing the plaintext with both oracles.
Once the attacker has obtained a valid \ssltwo ciphertext $c_1$, they repeatedly
use the shifting technique described in Section~\ref{sec:rotations} to rotate
the message by 25 bytes each iteration while choosing 3DES as the
symmetric cipher, learning 27 bytes with each shift.  After several iterations,
they have learned the entire plaintext, using 6,300 queries (again for a 2048-bit
modulus).
This brings the overall number of queries for this variant of the attack to
$ 900 + 16 * 4 + 6,300 = 7,264 $.
These parameter choices are not necessarily optimal.  We give more details in Appendix~\ref{sec:special-both}.
