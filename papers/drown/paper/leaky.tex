In addition to the \tOracleSSLclear $\OracleSSLclear$ described in Section~\ref{sec:special},
the same set of OpenSSL versions allow a different oracle,
which we term \tOracleSSLleaky $\OracleSSLleaky$.
This additional oracle is powerful in a different way than
\tOracleSSLclear --- it still requires roughly $2^{40}$ offline computations
per query, but it is more permissive when checking for conformant messages.

\subsection{The OpenSSL \tOracleSSLleaky}
In addition to the Extra clear implementation bug, these OpenSSL versions
also contain a separate bug, where they do not follow the correct algorithm
for their implementation of the Bleichenbacher countermeasure.

We now describe this faulty implementation:
\begin{itemize}
	\item Observe the received buffer for the \texttt{ClientKeyExchange} message, $p$.
	As per the \ssltwo standard, $p$ happens to contain the $mk_{clear}$
	bytes immediately before the RSA ciphertext, $c$.
	We slightly abuse notation and use $p$ to also refer to the buffer starting at the first
	$mk_{clear}$ byte.

	\item Decrypt $c$ in-place. If the decryption operation succeeds,
	and $c$ decrypted to a plaintext of a correct padded length,
	$p$ now contains the 11 $mk_{clear}$ bytes followed by the 5
	$mk_{secret}$ bytes.

	\item If $c$ decrypted to an unpadded plaintext $k$ of incorrect length,
	the decryption operation overwrites the first $j = min(|k|, 5)$ bytes
	of $c$ with the first $j$ bytes of $k$.

	\item If $c$ decrypted to a plaintext of an incorrect unpadded length,
	or if $c$ is not \sslconform and the decryption operation failed,
	randomize the first five bytes of $p$, which are \textit{the first
	five bytes of $mk_{clear}$}.
\end{itemize}

This behavior allows the attacker to distinguish between three cases.
Suppose the attacker sends 11 null bytes as $mk_{clear}$.
Then these are the possible cases:
\begin{enumerate}
	\item $c$ decrypts to a correctly padded plaintext $k$ of the expected length, 5 bytes.
		In this case, the \texttt{master\_key} becomes:
		$$[00\ 00\ 00\ 00\ 00\ 00\ 00\ 00\ 00\ 00\ 00\ k_{[1]}\ k_{[2]}\ k_{[3]}\ k\pos{4}\ k\pos{5}]$$
%   		$$ mk = \hex{00}^{\left[11\right]} || k $$
%		where $|k| = 5$.
	\item $c$ decrypts to a correctly padded plaintext $k$ of a wrong length.
		Let $r$ be the five random bytes the server generated.
		If $|k| > 5$,
		then the vulnerable OpenSSL version constructs the following \texttt{master\_key}:
		$$[r\pos{1}\ r\pos{2}\ r\pos{3}\ r\pos{4}\ r\pos{5}\ 00\ 00\ 00\ 00\ 00\ 00\ k\pos{1}\ k\pos{2}\ k\pos{3}\ k\pos{4}\ k\pos{5}]$$
		and if $|k| < 5$, using $|k| = 3$ as an example,
		then the vulnerable OpenSSL version constructs the following \texttt{master\_key},

		\todo{Change from example to a formal phrasing}

		$$[r\pos{1}\ r\pos{2}\ r\pos{3}\ r\pos{4}\ r\pos{5}\ 00\ 00\ 00\ 00\ 00\ 00\ k\pos{1}\ k\pos{2}\ k\pos{3}\ c\pos{4}\ c\pos{5}]$$

		where $c\pos{1}, \ldots, c\pos{5}$ denote the first five bytes of the RSA ciphertext $c$.
		\todo{Fix indentation}
%		$$ mk = r || \hex{00}^{\left[6\right]} || k $$
	\item $c$ is not \sslconform,
		and hence the decryption operation failed. In this case, the \texttt{master\_key} becomes:
		$$[r\pos{1}\ r\pos{2}\ r\pos{3}\ r\pos{4}\ r\pos{5}\ 00\ 00\ 00\ 00\ 00\ 00\ c\pos{1}\ c\pos{2}\ c\pos{3}\ c\pos{4}\ c\pos{5}]$$
%		$$ mk = r || \hex{00}^{\left[6\right]} || c_{1..5} $$
\end{enumerate}

The attacker can detect case (3) by performing an exhaustive search over the $2^{40}$ possibilities for $r$,
and checking whether any of the resulting values for the \texttt{master\_key} correctly 
decrypts the observed \texttt{ServerVerify} message.
If no $r$ value satisfies this property, then
the decrypted plaintext $c^d$ starts with bytes \hexb{00}{02}.
The attacker can then distinguish between cases (1) and (2) by 
performing an exhaustive search over the five bytes of $k$,
and checking whether any of the resulting values for $mk$ correctly 
decrypts the observed \texttt{ServerVerify} message.

In conclusion, $\OracleSSLleaky$ allows an attacker to obtain a valid oracle response
for all ciphertexts which decrypt to a correctly-padded plaintext of \textit{any} length. This is in contrary to the previous oracles $\OracleSSLclear$ and $\OracleSSLexp$, which required the plaintext to be of a specific length.
Each oracle query requires one \ssltwo connection to the server,
and at most $2^{40}$ offline work.


\subsection{Combining the two oracles}
When combining the \tOracleSSLclear and \tOracleSSLleaky, a new technique
presents itself.
The attacker can now first test a \tlsconform ciphertext for divisors
using the \tOracleSSLleaky,
and then use only fractions for which the denominator divides the plaintext
with both oracles.

Using the \tOracleSSLleaky, the probability that a fraction $u/t$ will result
in a positive response, is $P = P_0 * P_3$, where the formula for
computing $P_0 = P((m \cdot u/t)[1,2] = 00||02)$ was provided in Section~\ref{sec:trimmers},
and $P_3$ is, for a 2048-bit modulo:
\begin{equation}
\begin{aligned}
P_3 = P(
\hex{00} \text{ } \not \in \text{ } \{m_3, \ldots,m_{10}\} \wedge \\
\hex{00} \text{ } \in \text{ } \{m_{11}, \ldots,m_{\ell}\}) \\
 = (1 - 1/256)^{8} * (1 - (1 - 1/256)^{246}) = 0.60
\end{aligned}
\end{equation}

Since $P_0$ is also non-negligible for small fractions, the overall probability of
obtaining a positive response is now also non-negligible.

\paragraph{Phase 1.}
Our goal for this phase is to obtain a divisor $t$ as large as possible,
such that $t|m$. We generate a list of small
fractions, sorted descendingly by their probability of
resulting in a positive response from $\OracleSSLleaky$. For a given ciphertext $c$, we then query with
the 50 fractions in the list with the highest probability,
until we obtain a first positive response for a fraction $u_0/t_0$.
We can now deduce that $t_0|m$.
We then generate a list of fractions $u/t$ where $t$ is a multiple of $t_0$, sort them again
by success probability, and again query with the 50 most probable fractions, until a positive answer is obtained,
or the list is exhausted.
If a positive answer is obtained, we iteratively re-apply this process,
until the list is exhausted, resulting in a final fraction $u^*/t^*$.

\paragraph{Phase 2.}
We then query with all fractions
denominated by $t^*$, and hope the ciphertext decrypts to a plaintext of
one of seven possible lengths: $\{2, 3, 4, 5, 8, 16, 24\}$.
These lengths all result in knowledge of at least three least significant bytes,
which allows us to use the shifting technique in order to continue the attack.
Detecting plaintext lengths 8, 16 and 24 can be accomplished using three \tOracleSSLclear
queries, as explained above.
Detecting plaintext lengths 2, 3, 4 and 5 can be accomplishing by using a single
\tOracleSSLleaky query, which requires at most $2^{40}$ offline computation.
In fact, the optimization over the key search space described in
Section~\ref{sec:trimmers} is applicable here and can slightly reduce the required computation.
Therefore, by initiating four SSLv2 connections and performing at most $2^{40}$ offline
work, the attacker can test for this much more probable property.

In practice, choosing 50 fractions per iteration as described above,
results in a success probability of 0.066 for a single ciphertext.
Hence, the expected number of required ciphertexts is merely $1/0.066 = 15$.
The expected number of fractions per ciphertext for phase 1 is 60, as in most cases, phase 1
consists of just a few successful iterations.
Since each fraction requires a single query to $\OracleSSLleaky$,
the overall number of queries for this stage is
$15 * 60 = 900$, and the required offline computation is at most
$900 * 2^{40} \approx 2^{50}$, which is similar to General DROWN\@.
The expected number of queries for phase 2 is 16.
Each query consists of three queries to $\OracleSSLclear$, and one query
to $\OracleSSLleaky$, which requires at most $2^{40}$ computation.
therefore the attacker has to perform at most $2^{44}$ offline computation for phase 2.

The attack can then continue in the same manner as the \tOracleSSLclear attack,
using 6,300 queries. This brings the overall numer of queries to
$ 900 + 16 * 4 + 6,300 = 7,264 $.
The parameter choices presented above for this attack variant are not necessarily optimal.

