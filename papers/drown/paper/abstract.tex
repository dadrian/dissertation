We present DROWN, a novel cross-protocol attack on TLS that
uses a server supporting SSLv2 as an oracle to decrypt modern TLS connections.

We introduce two versions of the
attack. The more general form exploits multiple unnoticed protocol flaws in
SSLv2 to develop a new and stronger variant of the Bleichenbacher RSA
padding-oracle attack.  To decrypt a 2048-bit RSA TLS ciphertext, an attacker
must observe 1,000 TLS handshakes, initiate 40,000 \ssltwo connections, and
perform $2^{50}$ offline work. The victim client never initiates \ssltwo
connections. We implemented the attack and can decrypt a TLS 1.2 handshake using
2048-bit RSA in under 8 hours, at a cost of \$440 on Amazon EC2\@. Using
Internet-wide scans, we find that 33\% of all HTTPS servers and 22\% of those
with browser-trusted certificates are vulnerable to this protocol-level attack
due to widespread key and certificate reuse.

For an even cheaper attack, we apply our new techniques together with a newly
discovered vulnerability in OpenSSL that was present in releases from 1998 to
early 2015.  Given an unpatched SSLv2 server to use as an oracle, we can
decrypt a TLS ciphertext in one minute on a single CPU---fast enough to enable
man-in-the-middle attacks against modern browsers. We find that 26\% of HTTPS servers are
vulnerable to this attack.

We further observe that the QUIC protocol is
vulnerable to a variant of our attack that allows an attacker to impersonate a
server indefinitely after performing as few as 
$2^{17}$ \ssltwo connections and $2^{58}$ offline work. %% Updated to match 6.3
%$2^{25}$ \ssltwo connections and $2^{65}$ offline work. %% Leaky's in, so restoring previous line

We conclude that \ssltwo is not only weak, but actively harmful to the TLS ecosystem.

