% !TEX root = ../../../proposal.tex
\documentclass{article}

\begin{document}

\section{Logistics Stuff - temporary}
This is just a temporary place to scribble down todo items.

\todo{Paper submissions should be at most 13 typeset pages, excluding bibliography and well-marked appendices. These appendices may be included to assist reviewers who may have questions that fall outside the stated contribution of the paper on which your work is to be evaluated or to provide details that would only be of interest to a small minority of readers. There is no limit on the length of the bibliography and appendices, but reviewers are not required to read any appendices so the paper should be self-contained without them. Once accepted, papers must be reformatted to fit in 18 pages, including bibliography and any appendices.}

\subsection{Paper todos}
\begin{itemize}
	\item Move ciphersuite selection bug to Measurements - LGTM after first pass from me.
	\item Describe the Leaky Export oracle in the special drown section - made a first review pass, I'm still not thrilled about the state of the new text.
	\item Describe the improvement of Leaky Export on QUIC in the signature forgery section - LGTM after first pass from me.
	\item Shorten the paper to 13 pages for the body, and 18 pages overall.
	\item If space permits, include a table summarizing the different attacks,
	with the oracles they require, and the required amounts of server connections
	and offline work - DONE. I'd really like to keep it.
\end{itemize}

Paper structure:
0 Abstract
1 Intro
2 Background
  2.1 PKCS
  2.2 SSL and TLS
  2.3 Bleichenbacher
3 Attack
  3.1 Attack scenario
  3.2 A generic SSLv2 oracle
  3.3 DROWN attack template
  3.3.1 Phase 1
  3.3.2 Phase 2 - Rotations
  3.3.3 Later phases - Adapted Bleichenbacher iteration
4 General DROWN
  4.1 The SSLv2 export padding oracle
  4.2 TLS decryption attack
    4.2.1 Attack scenario
    4.2.2 Constructing the attack
    4.2.3 Attack performance
  4.3 Implementing general DROWN with GPUs
5 Special DROWN
  5.1 The Extra Clear oracle
  5.2 The OpenSSL Leaky export oracle
  5.3 Combining the two oracles
6 Measurements
  Cipher suite selection bug
  Widespread public key reuse
  Special DROWN
7 MITM attacks and QUIC
  7.1 MITM attack against TLS
    7.1.1 Attack scenario
    7.1.2 Constructing the attack
  7.2 Extending the attack to QUIC
    Unauthenticated QUIC discovery
    Signature forgery details
    Attack cost
  7.3 QUIC signature forgery using special DROWN
  7.4 SSLv2 servers with CA certificates
8 Related work
9 Discussion
  9.2 Implications for modern protocols
  9.3 Lessons for key reuse
  9.4 Harms from obsolete cryptography
  9.5 Harms from deliberately weakening cryptography
Acknowledgements
Appendices:
A public key reuse
B Adaptations to Bleichenbacher’s attack
  B.1 Calculating the success probability of a fraction
  B.2 Optimizing the chosen set of fractions
  B.3 Efficiently computing rotations and multipliers
  B.4 Rotations in the general DROWN attack
  B.5 General DROWN Bleichenbacher iterations
  B.6 General DROWN attack performance
  B.7 Performance of the MITM attack against TLS using special DROWN
    Using multiple queries per fraction
C Highly optimized GPU implementation
D Amazon EC2 evaluation
E A brief history of obsolete cryptography

Stuff we might cut:
- Figure 1: SSLv2 handshake
- 3.1 "attacker's position in the network".
  We could probably make this shorter, currently it's roughly a 1/3 of a page.
  Similarly for 4.2.1 "Attack scenario" under "General DROWN"
- 5.1 The Extra Clear oracle - we can probably make this shorter.
  The visual master_keys in particular can be replaced with a 0^ notation, saving a lot of space.
  Similarly for 5.2 The OpenSSL Leaky export oracle.
- Table 5 - Summary of the attacks.
- 9.1 Lessons for protocol design
- Appendix B: It's not at all clear if/what parts overlap with the body.
- Appendix A.3: The lattices part.
  As cool as it is, if we're short on space, we can allow the reader to assume
  we just used more CPU, the entire attack before this optimization took 10
  minutes of CPU on my machine.

Shortened:
- Widespread public key reuse - it's an important point,
  but I'm not sure we can afford a 1/3 of a page to it

Removed:
- Appendix A: Public key reuse.
- Appendix C: Highly optimized GPU implementation.
- Appendix E: A brief history of obsolete cryptography:
  Personally I'd like to hammer this point home even in the body,
  but realistically we give it due space in 9.4.
- Table 4 - hosts vulnerable to special DROWN.
  Ideally we should include it, but with our space limitations we might consider
  merging the HTTPS and Alexa rows into the general DROWN table?
- Figure 2: Visual overview of the attack
- 7.4 SSLv2 servers with CA certificates
- D Amazon EC2 evaluation
- 9.1 Lessons for protocol design

Comparison between Section 3/4 and Appendix B:
--------------------------------------------
3 Attack
  3.1 Attack scenario
  3.2 A generic SSLv2 oracle
  3.3 DROWN attack template
  3.3.1 Phase 1 - just a primer about trimmers, with a concrete example.
        No probability computation.
  3.3.2 Phase 2 - Rotations
  3.3.3 Later phases - Adapted Bleichenbacher iteration
4.2.2 Constructing the attack: Does explain the rationale behind the different phases.

Appendix should:
V Compute random probability.
V A.4 should contain further explanations on the shifting technique?
  I'm not sure any more explanations are needed.
  (Yes, they're needed to tie together the rationale,
  i.e. explain with formulas the need for m_3$.
V A.5 should further describe Bleichenbacher iteration.
V Should not include the rationale behind the different phases.
  (It only formalizes the need, LGTM).
V A.6 should explain the computation given a fraction set F.
  (A.1 does this, fixed the ref and deleted A.6).

Things we might add back if space permits:
------------------------------------------
- False Start paragraph.
- Table 4 (summary of attacks).

\end{document}

