TLS has had a long history of implementation flaws and protocol
attacks~\cite{poodle-2014,crime-attack,rc4-biases-2013,lucky-13-2013,beast-2011,sloth-2016,heartbleed-2014}.
We discuss relevant Bleichenbacher and cross-protocol attacks below.

\paragraph{Bleichenbacher's attack.}
Bleichenbacher's adaptive chosen ciphertext attack against SSL was first
published in 1998~\cite{bleichenbacher-1998}. Several works have adapted his
attack to different
scenarios~\cite{klima-rsa-tls-2003,efficient-padding-oracle-2012,jager-2012}.
The TLS standard explicitly introduces countermeasures against the
attack~\cite{rfc5246}, but several modern implementations have been
discovered to be vulnerable to timing-attack variants in recent
years~\cite{meyer-2014,zhang-2014}. These side-channel
attacks are implementation failures and only apply when the attacker is
co-located with the victim.

\paragraph{Cross-protocol attacks.}
Jager et al.\@ \cite{tls-quic-pkcs-2015} showed that a cross-protocol
Bleichenbacher RSA padding oracle attack is possible against the proposed TLS
1.3 standard, in spite of the fact that TLS 1.3 does not include RSA key
exchange, if server implementations use the same certificate for previous
versions of TLS and TLS 1.3. Wagner and Schneier~\cite{ssl-v3-1996} developed
a cross-cipher suite attack for SSLv3, in which an attacker could reuse a
signed server key exchange message in a later exchange with a different
cipher suite. Mavrogiannopoulos et al.\@~\cite{mvvp-2012} developed a
cross-cipher suite attack allowing an attacker to use elliptic curve
Diffie-Hellman as prime field Diffie-Hellman.

\paragraph{Attacks on export-grade cryptography.}
Recently, the FREAK~\cite{smack-tls-2015} and Logjam~\cite{logjam-2015}
attacks allowed an active attacker to downgrade a connection to export-grade
RSA and Diffie-Hellman, respectively. DROWN exploits export-grade symmetric
ciphers, completing the export-grade cryptography attack trifecta.
