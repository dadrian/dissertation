% !TEX root = ../../../proposal.tex
\subsection{TLS decryption with special DROWN}
\label{sec:clear_analysis}

Using our oracle $\OracleSSLclear$, we can construct an extremely efficient version of our TLS decryption attack.  The OpenSSL \tOracleSSLclear provides three significant advantages over our export oracle $\OracleSSLexp$: (1) It no longer requires an export cipher suite, and, in fact, we gain efficiency by exploiting regular SSLv2 ciphers; (2) It requires only one handshake per oracle query; and (3) Computation is reduced to one \texttt{ServerVerify} decryption per oracle query, versus $2^{40}$.

\subsubsection{Attack scenario}

As before, we consider a server that accepts TLS connections, and a client that negotiates a secure, state-of-the-art TLS version with a \texttt{TLS\_RSA} cipher suite.  The same RSA key pair used for TLS is also used on a server that is running a vulnerable version of OpenSSL.

\subsubsection{Constructing the attack}

The attacker can exploit the OpenSSL extra clear vulnerability to efficiently decrypt a TLS ciphertext as follows.  We will use the cipher suite \texttt{SSL\_DES\_192\_EDE3\_CBC\_WITH\_MD5} as the cipher suite, allowing the attacker to recover 24 bytes of key at a time from the oracle.
We first present a straightforward adaptation of the general DROWN attack to the \tOracleSSLclear,
before later applying a few additional optimizations made possible by this new oracle.

\begin{enumerate}
 \setcounter{enumi}{-1}
 \item The attacker intercepts several hundred TLS handshakes using RSA key exchange.
 \item The attacker uses the fractional trimmers as described in Section~\ref{sec:trimmers} to convert the TLS ciphertexts into an \sslconform ciphertext $c_0$.
 \item Once the attacker has obtained a valid \ssltwo ciphertext $c_1$, he repeatedly uses the shifting technique described in Section~\ref{sec:rotations} to rotate the message by 25 bytes each iteration, learning 27 bytes with each shift.  After several iterations, he has learned the entire plaintext.
 \item The attacker then transforms the decrypted \ssltwo plaintext into the decrypted TLS plaintext. 
 \end{enumerate}

\paragraph{Attack costs}
Using 40 fractional trimmers, this more efficient oracle attack allows
the attacker to recover one in 260 TLS session keys using only about
17,000 connections to the server.  The computation cost is so low that
we can complete the full attack on a single workstation in under one
minute. Appendix~\ref{sec:special-performance} gives more details.

Mounting the attack using the optimized version of Special DROWN
described in Appendix~\ref{sec:special-performance} allows the
attacker to target one of 100 connections, at the expense of
increasing the number of queries to 27,000.

