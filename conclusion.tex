\textit{This is an outline}

List of how the empirical methods discussed earlier had impact
(pull-out relevant results).

Did any of this research go anywhere beyond that? Yes, it had a clear impact
on TLS 1.3.

\section{TLS 1.3}

Point out that the ``informing future protocol design'' is effectively
``making TLS 1.3 better''. This is the ``improve the security of the Internet
in the future'' part.

Call out explicit TLS 1.3 design decisions based on results described in this
dissertation. Perhaps some of this should be interleaved into the earlier
sections?

\section{Engineering Challenges}

For not being a ``systems'' field, there is still an absurd amount of
engineering that largely goes unacknowledged, in order to write good
measurement papers. Discuss some of these examples.

Is this a fundamental state of being of the methodology, or are we doing something wrong?

\section{Broader applicability of empirical methods}

Reusing TLS keys across multiple protocols, such as HTTPS, SMTP, and IMAP,
leads to an increased attack surface. Empirical methods allow us to understand
the attack surface increase from key reuse. Furthermore, specific
vulnerabilities in the TLS protocol and older implementations can be utilized
in a cross-protocol context to attack users of a web service without explicitly
compromising the private key. This is best shown by the DROWN vulnerability, in
which the mere existence of an \ssltwo host that shared a key with a TLS host
enabled decryption of otherwise secure TLS connections using modern
cryptography.

The DROWN attack further exploited export-grade cryptography with an additional
novel insight: Bleichenbacher oracles need not be present in the target
protocol under attack, so long as the key is shared between the two protocols.
Specifically, DROWN shows how to use protocol vulnerabilities in \ssltwo to
attack TLS 1.2. The \ssltwo protocol includes support export symmetric ciphers
which are seeded via only five bytes of key material encrypted using RSA \PKCS.

Beyond DROWN, the TLS protocol has a fundamentally cross-protocol attack
surface. X.509 certificates are not bound to any particular protocol or port.
Furthermore, even if distinct services, such as mail and web servers, use
different keys, so long as they share any name on the certificate, the
transport-layer security of all connections to that name are limited to the
security of the weakest TLS implementation or configuration. Even traditionally
web-based padding oracle attacks, such as POODLE, or the AES-NI padding-oracle
in OpenSSL, non-web servers can be exploited by active attackers targeting web
users. The attacker can rewrite the TCP connection to an alternative port, and
fill-in any pre-handshake protocol dialogue (e.g. by sending an EHLO or
STARTTLS command in SMTP). Ignoring vulnerabilities in TLS itself, an unpatched
piece of software with a known RCE using the same key as a well-configured and
up to date web server places web clients, should the key be stolen via
traditional software exploitation. We can place an upper bound on the increased
attack surface, by measuring key and name reuse across TLS in different
application-layer protocols on different ports.

Security driven by data. Is any of this relevant to users who are just trying
to secure this own networks?

Extending measurement from aggregations and ecosystems to behavior of
individual hosts at global scale. Can we track individual hosts appearing and
disappearing, and map changes to configuration in real-time?

Should empiricism be part of more traditional cryptography education and research?

Some word of warning about how measurement doesn't solve all our problems,
and how measuring the wrong things makes things worse, \eg Robert McNamera's
use of body count as a metric during the Vietnam War.

% What the fuck does anything have to do with Vietnam?


\section{Weaknesses from Export Cryptography}

As shown by Freak, Logjam, and DROWN, the security of TLS and export
cryptography are fundamentally linked. Export cryptography is a unique
constraint with a fundamentally dangerous goal: weaken cryptography, without
weakening cryptography. Internet measurement techniques show us that the export
regulations weakened protocol design to the point where the regulations are
directly harmful to the security of the Internet today. These empirical
techniques show that these attacks are not theoretical, leveraging protocols
that have long-since disappeared, but instead are a dark side of backwards
compatibility, harming real users today. Although the regulations went out of
effect by 1999, the cryptography remains. At their respective times of
disclosure, 36.7\% of IPv4 HTTPS hosts were vulnerable to FREAK, 4.9\% were
vulnerable to Logjam, and 26\% were vulnerable to DROWN. All forms of export
cryptography have been broken: export RSA key exchange was broken by FREAK,
export Diffie-Hellman key exchange was broken by Logjam, and export symmetric
ciphers were broken by DROWN. In all cases, empirical research enabled the full
understanding of the effects and impacts of these issues.

