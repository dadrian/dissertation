\textit{This is an outline}

List of how the empirical methods discussed earlier had impact
(pull-out relevant results).

Did any of this research go anywhere beyond that? Yes, it had a clear impact
on TLS 1.3.

\section{TLS 1.3}

Point out that the ``informing future protocol design'' is effectively
``making TLS 1.3 better''. This is the ``improve the security of the Internet
in the future'' part.

Call out explicit TLS 1.3 design decisions based on results described in this
dissertation. Perhaps some of this should be interleaved into the earlier
sections?

\section{Engineering Challenges}

For not being a ``systems'' field, there is still an absurd amount of
engineering that largely goes unacknowledged, in order to write good
measurement papers. Discuss some of these examples.

Is this a fundamental state of being of the methodology, or are we doing something wrong?

\section{Broader applicability of empirical methods}

Security driven by data. Is any of this relevant to users who are just trying
to secure this own networks?

Extending measurement from aggregations and ecosystems to behavior of
individual hosts at global scale. Can we track individual hosts appearing and
disappearing, and map changes to configuration in real-time?

Should empiricism be part of more traditional cryptography education and research?

Some word of warning about how measurement doesn't solve all our problems,
and how measuring the wrong things makes things worse, \eg Robert McNamera's
use of body count as a metric during the Vietnam War.

% What the fuck does anything have to do with Vietnam?


\section{Weaknesses from Export Cryptography}

As shown by Freak, Logjam, and DROWN, the security of TLS and export
cryptography are fundamentally linked. Export cryptography is a unique
constraint with a fundamentally dangerous goal: weaken cryptography, without
weakening cryptography. Internet measurement techniques show us that the export
regulations weakened protocol design to the point where the regulations are
directly harmful to the security of the Internet today. These empirical
techniques show that these attacks are not theoretical, leveraging protocols
that have long-since disappeared, but instead are a dark side of backwards
compatibility, harming real users today. Although the regulations went out of
effect by 1999, the cryptography remains. At their respective times of
disclosure, 36.7\% of IPv4 HTTPS hosts were vulnerable to FREAK, 4.9\% were
vulnerable to Logjam, and 26\% were vulnerable to DROWN. All forms of export
cryptography have been broken: export RSA key exchange was broken by FREAK,
export Diffie-Hellman key exchange was broken by Logjam, and export symmetric
ciphers were broken by DROWN. In all cases, empirical research enabled the full
understanding of the effects and impacts of these issues.

