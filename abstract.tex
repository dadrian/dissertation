% !TEX root = proposal.tex
\abstract{
Cryptography is a key component of the security of the Internet.
Unfortunately, the process of using cryptography to secure the Internet is
fraught with failure. Cryptography is often fragile, as a single mistake can
have devastating consequences on security, and this fragility is further
complicated by the diverse and distributed nature of the Internet. This
dissertation shows how to use empirical methods in the form of Internet-wide
scanning to study how cryptography is deployed on the Internet, and shows
this methodology can discover vulnerabilities and gain insights into fragile
cryptographic ecosystems that are not possible without an empirical approach.
I introduce improvements to ZMap, the fast Internet-wide scanner, that allow
it to fully utilize a 10\,GigE connection, and then use Internet-wide
scanning to measure cryptography on the Internet.

First, I study how Diffie-Hellman is deployed, and show that implementations
are fragile and not resilient to small subgroup attacks. Next, I measure the
prevalence of ``export-grade'' cryptography. Although regulations limiting
the strength of cryptography that could be exported from the United States
were lifted in 1999, Internet-wide scanning shows that support for various
forms of export cryptography remains widespread. I show how purposefully
weakening TLS to comply with these export regulations led to the FREAK,
Logjam, and DROWN vulnerabilities, each of which exploits obsolete
export-grade cryptography to attack modern clients. I conclude by discussing
how empirical cryptography improved protocol design, and I present further
opportunities for empirical research in cryptography.

\paragraph{Thesis Statement.} Large-scale empirical methods allow us to
observe fragility in how cryptography is being used on the Internet, identify
new vulnerabilities, and better secure the Internet in the future.
}
